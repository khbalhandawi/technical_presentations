
%==============================================================================%
%% Mathematical symbols
\definecolor{design}{HTML}{D0A31D}
\definecolor{spec}{HTML}{FF7575}
\definecolor{margin}{HTML}{CC99FF}
\definecolor{perf}{HTML}{87C061}
\newcommand\design[1]{{\color{design}#1}}
\newcommand\spec[1]{{\color{spec}#1}}
\newcommand\margin[1]{{\color{margin}#1}}
\newcommand\perf[1]{{\color{perf}#1}}
%==============================================================================%
%%%%%%%%%%%%%%%%%%%%%%%%%%%%%%%%%%%%%%%%%%%%%%%%%%%%%%%
%%              Supplementary Content                %%
%%%%%%%%%%%%%%%%%%%%%%%%%%%%%%%%%%%%%%%%%%%%%%%%%%%%%%%
%------------------------------------------------
\section{Supplementary content}
%------------------------------------------------
\begin{frame}[c,noframenumbering]
	\centering
	% \setlength\fboxsep{0pt}
	\begin{titleblock}{}
		~\\%
		{\centering\LARGE Supplementary content: Miscellaneous\\}%
		~\\%
	\end{titleblock}
\end{frame}
%------------------------------------------------
\subsection{Example problem: Design of TRS strut}
%------------------------------------------------
\begin{frame}[t,label=ap2_2]
	\frametitle{Overview of turbine rear frame design}
	\vspace{-1em}
	\only<1-2>{The turbine rear structure (TRS) sits aft of the turbine}%
	\only<3>{We consider the design of a single strut}%
	\only<4>{Variable thermal loads are expected}%
	\only<5-7>{Which cause thermal expansion \uncover<6-7>{and stress on the strut}}%
	\only<8>{The entire problem can be represented compactly as an MDM}%
	~\\
	\begin{columns}[t] % The "c" option specifies centered vertical alignment while the "t" option is used for top vertical alignment
			
		\begin{column}{.45\textwidth} % Left column and width
			\footnotesize
			Design parameters:
			\design{
				\uncover<3->{
				\begin{itemize}\itemsep0em 
					\item \only<8>{$D1$} vane width $w$
					\item \only<8>{$D2$} vane height $h$
					\item \only<8>{$D3$} lean angle $\theta$
					\item \only<8>{$D4$} Material - yield stress $\sigma_y$ (off-the shelf)
				\end{itemize}
			}}%
			Input Specifications:
			\spec{
				\uncover<5->{
				\begin{itemize}\itemsep0em 
					\item \only<8>{$S1$} Nacelle temperature $T_1$ ($\downarrow$)
					\item \only<8>{$S2$} Gas surface temperature $T_2$ ($\uparrow$)
				\end{itemize}
			}}%
			Design margins:
			\margin{
				\uncover<6->{
				\begin{itemize}\itemsep0em 
					\item \only<8>{$E1$} compression $\sigma_y - \sigma_a \ge 0$
					\item \only<8>{$E2$} bending $\sigma_y - \sigma_m \ge 0$
					\item \only<8>{$E3$} buckling $F_\text{buckling} - F_a \ge 0$
				\end{itemize}
			}}%
			Performance parameters:
			\perf{
				\uncover<7->{
				\begin{itemize}\itemsep0em 
					\item \only<8>{$P1$}  Weight
				\end{itemize}
			}}%
		\end{column}
	
		\begin{column}{.4\textwidth} % Left column and width
			\begin{figure}
				\only<1>{
					\includegraphics[width=6.0cm]{design_space_exploration/TRS/engine.png}
					\vspace{-0.2em}
					\caption{Trent 900 aeroengine}
				}%
				\only<2>{
					\includegraphics[width=5.0cm]{design_space_exploration/TRS/rear_view.pdf}
					\vspace{-0.2em}
					\caption{Turbine rear structure}
				}%
				\only<3>{
					\includegraphics[width=7.0cm]{design_space_exploration/TRS/cross_section_view.pdf}
					\vspace{-0.2em}
					\caption{Turbine rear structure: cross-sectional view}
				}%
				\only<4>{
					\includegraphics[width=6.0cm]{design_space_exploration/TRS/expansion_1.pdf}
					\vspace{-0.2em}
					\caption{Thermal expansion}
				}%
				\only<5>{
					\includegraphics[width=6.0cm]{design_space_exploration/TRS/expansion_2.pdf}
					\vspace{-0.2em}
					\caption{Thermal expansion}
				}%
				\only<6-7>{
					\includegraphics[width=7.0cm]{design_space_exploration/TRS/loads.pdf}
					\vspace{-0.2em}
					\caption{Load cases}
				}%
			\end{figure}
		\end{column}
		
	\end{columns}
	
\end{frame}
%------------------------------------------------
\subsection{Implementation}
%------------------------------------------------
\begin{frame}[fragile,t]
	\frametitle{Pseudo-code of MVM}
    %
	\vspace{-1em}
	\begin{algorithm}[H] % must use [H]
		\tiny
		\DontPrintSemicolon % Some LaTeX compilers require you to use \dontprintsemicolon instead
		\KwIn{
			set of possible choices $\mathcal{D}$, chosen design $\mathbf{c}\in\mathcal{D}$,
			the MAN given by analytic functions, models, and surrogates,
			$n_\mathrm{samples}$, Joint PDF $F_\mathbf{S}(\mathbf{s})$
		}
		\KwOut{estimated mean impact and change absorption $\bar{\bar{\mathbf{a}}}$, $\bar{\bar{\mathbf{i}}}$}
		%
		\textbf{[0] Initialization}: 
			set counter $i_\mathrm{sample} \gets 0$, initialize aggregate absorption and impact $\bar{\bar{\mathbf{a}}} \in \mathbb{R}^e$ and $\bar{\bar{\mathbf{i}}} \in \mathbb{R}^e$\;
		%
		\textbf{[1] Monte-carlo simulation}: 
			draw sample input specifications $\mathbf{S}\in\mathbb{R}^n \sim F_\mathbf{S}(\mathbf{s})$\;
		%
		\textbf{[2] Calculate threshold, decided, excess, and performance values}: 
			$\mathbf{t} = \mathbf{f}(\mathbf{S},\mathbf{c})$, 
			$\mathbf{d} = \mathbf{g}(\mathbf{c})$, 
			$\mathbf{e} = \left(\mathbf{t} - \mathbf{d}\right){\mathbf{i}^\mathrm{excess}}^\mathsf{T}$, 
			$\mathbf{p} =  \mathbf{h}(\mathbf{S},\mathbf{c})$\;
		%
		\textbf{[3] Calculate allowable deterioration}\;
		\qquad loop over input specifications \For{$k = 1, 2, ..., n$} {
			\qquad perform a line search along the $k$th direction to identify the root of: 
				$\min{((\mathbf{f}(\mathbf{x},\mathbf{c}) - \mathbf{d}){\mathbf{i}^\mathrm{excess}}^\mathsf{T})} = \mathbf{0}$, $s_k' \gets x^*$, 
				$s_k^\mathrm{limit} \gets s_k'$\;
		}
		%
		\textbf{[4] Calculate margin value metrics}\;
		\qquad initialization tensors and vectors, 
			$\mathbf{I} \in \mathbb{R}^{m\times e} \gets \mathbf{0}$ and $\mathbf{A} \in \mathbb{R}^{n\times e} \gets \mathbf{0}$, 
			$\mathbf{v} \in \mathbb{R}^{n} \gets \mathbf{0}$, $\bar{\mathbf{a}} \in \mathbb{R}^{e} \gets \mathbf{0}$, and $\bar{\mathbf{i}} \in \mathbb{R}^{e} \gets \mathbf{0}$\;
		\qquad loop over margin nodes \For{$i = 1, 2, ..., e$} {  
			\qquad \textbf{[a] Calculate impact on performance}\;
			\qquad $\mathbf{d}^\mathrm{threshold} \gets \mathbf{d}$, $d_i^\mathrm{threshold} \gets t_i$, 
				$\mathbf{c}_i^\mathrm{threshold} = \mathbf{g}^{-1}(\mathbf{d}^\mathrm{threshold})$, 
				$\mathbf{p}_i^\mathrm{threshold} = \mathbf{h}(\mathbf{S},\mathbf{c}_i^\mathrm{threshold})$\;
			\qquad loop over performance parameters \For{$j = 1, 2, ..., m$} {
				\qquad $i_{i,j} \gets (p_j - p_{j,i}^\mathrm{threshold})i_j^p/{p_{j,i}^\mathrm{threshold}}$\;
			}
			\qquad \textbf{[b] Calculate change absorption capability}\;
			\qquad loop over input specifications \For{$k = 1, 2, ..., n$} {
				\qquad calculate deterioration $v_k=(s_k^\mathrm{limit} - s_k)/{s_k}$\;
				\qquad $a_{i,k} \gets (f_i(\mathbf{s}_k^\mathrm{limit},\mathbf{c}) - f_i(\mathbf{S},\mathbf{c}))/(f_i(\mathbf{S},\mathbf{c})v_k)$\;
			}
			\qquad \textbf{[c] Aggregate metrics}: 
				$\bar{a}_i \gets \frac{1}{n}\sum_{k=1}^{n} a_{i,k}$, 
				$\bar{i}_i \gets \frac{1}{m}\sum_{k=1}^{m} i_{i,j}$
		}
		%
		\textbf{[5] sum of monte-carlo samples}: 
			$\bar{\bar{\mathbf{a}}} \gets \bar{\bar{\mathbf{a}}} + \bar{\mathbf{a}}$, 
			$\bar{\bar{\mathbf{i}}} \gets \bar{\bar{\mathbf{i}}} + \bar{\mathbf{i}}$\;
		%
		\textbf{[6] Termination}, 
		\If{$i_\mathrm{sample} < n_\mathrm{samples}$} {
			set $i_\mathrm{sample} \gets i_\mathrm{sample}+1$, go to \textbf{[1]}\;
			otherwise $\bar{\bar{\mathbf{a}}} \gets \bar{\bar{\mathbf{a}}}/n_\mathrm{samples}$ and $\bar{\bar{\mathbf{i}}} \gets \bar{\bar{\mathbf{i}}}/n_\mathrm{samples}$, \textbf{stop}
		}
	\end{algorithm}
\end{frame}
%------------------------------------------------
\begin{frame}[t,noframenumbering]
	\frametitle{UML diagram of \texttt{mvmlib}\ifshowcitations\footpartcite{Alhandawi2022}\fi}
    %
	\begin{figure}
        \centering
		\only<1>{\includegraphics[width=1.0\textwidth]{design_space_exploration/classes_mvmlib.pdf}}%
		\only<2>{\includegraphics[width=0.6\textwidth]{design_space_exploration/packages_mvmlib.pdf}}%
    \end{figure}%
\end{frame}
\addtocounter{footnote}{-1}
%------------------------------------------------
\subsection{Overview of mesh adaptive direct search}
%------------------------------------------------
\begin{frame}[t,label=abm_5]
	\frametitle{Overview of mesh adaptive direct search}
	\tikzstyle{background grid}=[draw, black!50,step=.5cm]
	%
	\uncover<1->{No gradient information available, blackbox is expensive\ifshowcitations\footpartcite{Audet2017}\fi}\\
	%
	\begin{columns}[t] % The "c" option specifies centered vertical alignment while the "t" option is used for top vertical alignment
		\begin{column}{.42\textwidth} % Left column and width
			\vspace{-1.2em}
			% Optimization problem
			\begin{exampleblock}{Objective and constraints}
				\begin{equation*}
					\begin{aligned}
						& \underset{\mathbf{x}}{\text{min}}
						& & f(\mathbf{x})\\
						& \text{subject to}
						& & {c}(\mathbf{x}) \le 0\\
						& \text{where}
						& & \mathbf{x}:\text{ variables}\\
					\end{aligned}
				\end{equation*}
			\end{exampleblock}

			\small
			\begin{itemize}\itemsep0em
				\item SEARCH step and POLL step
				\item<2-> we focus on the POLL step
				\item<3-> incumbent solution: $x_k$
				\item<4-> trail points: $\left\{p_1,p_2,\cdots,p_k\right\}$
				\item<10-> successful trail point $p_7$
			\end{itemize}
		\end{column}
		%
		\begin{column}{.5\textwidth} % Left column and width
			\tikzstyle{background grid}=[draw, black!50,step=.5cm]
			
			\only<2>{\def\frameopt{-2}}%
			\only<3>{\def\frameopt{-1}}%
			\only<4>{\def\frameopt{0}}%
			\only<5>{\def\frameopt{1}}%
			\only<6>{\def\frameopt{2}}%
			\only<7>{\def\frameopt{3}}%
			\only<8>{\def\frameopt{4}}%
			\only<9>{\def\frameopt{5}}%
			\only<10>{\def\frameopt{6}}%
			\only<11>{\def\frameopt{7}}%
			\only<12>{\def\frameopt{8}}%
			\only<13>{\def\frameopt{9}}%

			\tikzstyle{background grid}=[draw, black!50,step=.1cm]
			\hspace*{5em}\raisebox{-1em}{%
				\begin{tikzpicture}[scale=0.45, every node/.style={scale=0.45}, remember picture, overlay] %show background grid, 
					\only<2->{\ifnum\frameopt < 2
    \def\unit{5}
    \def\meshsize{2}
    \def\pollX{0}
    \def\pollY{0}
    \def\pollsize{2}
    %
    \def\pAX{-1}
    \def\pAY{\pollsize/2}
    \def\pBX{\pollsize/2}
    \def\pBY{0}
    \def\pCX{0}
    \def\pCY{-\pollsize/2}
\fi
%
\ifnum\frameopt=-2
    % https://tex.stackexchange.com/a/183633
\newcommand{\Grid}[6]{%
    \def\startX{#1}
    \def\startY{#2}
    \def\maxX{#3}
    \def\maxY{#4}
    \def\size{#5}
    \def\borderthick{#6}
    %
    \draw[line width=\borderthick] (\size*0+\startX,\size*0+\startY) rectangle (\size*\maxX+\startX,\size*-\maxY+\startY);
    \foreach \x in {0,1,...,\maxX}{
        \draw (\size*\x+\startX,\size*0+\startY) -- (\size*\x+\startX,\size*-\maxY+\startY);
    }
    %
    \foreach \y in {0,1,...,\maxY}{
        \draw (\size*0+\startX,\size*-\y+\startY) -- (\size*\maxX+\startX,\size*-\y+\startY);
    }
}
%
\pgfmathsetmacro{\centerX}{\meshsize/2}
\pgfmathsetmacro{\centerY}{-\meshsize/2}
%
\pgfmathsetmacro{\pollcenterX}{\centerX+\pollX}
\pgfmathsetmacro{\pollcenterY}{\centerY+\pollY}
\pgfmathsetmacro{\pollcornerX}{\pollcenterX - \pollsize/2}
\pgfmathsetmacro{\pollcornerY}{\pollcenterY + \pollsize/2}
%
\pgfmathsetmacro{\pAcoordX}{\pollcenterX + \pAX}
\pgfmathsetmacro{\pAcoordY}{\pollcenterY - \pAY}
\pgfmathsetmacro{\pBcoordX}{\pollcenterX + \pBX}
\pgfmathsetmacro{\pBcoordY}{\pollcenterY - \pBY}
\pgfmathsetmacro{\pCcoordX}{\pollcenterX + \pCX}
\pgfmathsetmacro{\pCcoordY}{\pollcenterY - \pCY}
%
\pgfmathsetmacro{\deltameshn}{\pollcenterX-\pollcornerX}
\pgfmathsetmacro{\deltameshd}{\centerX}
\pgfmathsetmacro{\deltamesh}{(\pollcenterX-\pollcornerX)/\centerX}
\pgfmathsetmacro{\deltaframen}{\deltameshn * \deltameshn}
\pgfmathsetmacro{\deltaframed}{\deltameshd * \deltameshd}
\pgfmathsetmacro{\deltaframe}{\deltamesh * \deltamesh}
%
\Grid{0}{0}{\meshsize}{\meshsize}{\unit}{1pt}
\Grid{\pollcornerX*\unit}{\pollcornerY*\unit}{\pollsize}{\pollsize}{\unit}{3pt}
%
\ifdim \deltamesh pt = 1 pt
    \dimline[label style={above=0.5ex,}, line style = {line width=0.7},extension start length=+0.3,extension end length=+0.3] {(\pollcornerX*\unit,1)}{(\pollcenterX*\unit,1)}{\huge $\delta^{k}_\mathrm{poll}=1$};
    
    \dimline[label style={above=0.5ex,}, line style = {line width=0.7},extension start length=+0.3,extension end length=+0.3] {(\meshsize*\unit - 1*\unit,1)}{(\meshsize*\unit,1)}{\huge $\delta^{k}_\mathrm{mesh}=1$};
\else
    \dimline[label style={above=0.5ex,}, line style = {line width=0.7},extension start length=+0.3,extension end length=+0.3] {(\pollcornerX*\unit,0.5)}{(\pollcenterX*\unit,0.5)}{\huge $\delta^{k}_\mathrm{poll}=\addfraction{\deltameshn}{\deltameshd}$};
    
    \dimline[label style={above=0.5ex,}, line style = {line width=0.7},extension start length=+0.3,extension end length=+0.3] {(\meshsize*\unit - 1*\unit,0.5)}{(\meshsize*\unit,0.5)}{\huge $\delta^{k}_\mathrm{mesh}=\addfraction{\deltaframen}{\deltaframed}$};
\fi
\fi
%
\ifnum\frameopt=-1
    % https://tex.stackexchange.com/a/183633
\newcommand{\Grid}[6]{%
    \def\startX{#1}
    \def\startY{#2}
    \def\maxX{#3}
    \def\maxY{#4}
    \def\size{#5}
    \def\borderthick{#6}
    %
    \draw[line width=\borderthick] (\size*0+\startX,\size*0+\startY) rectangle (\size*\maxX+\startX,\size*-\maxY+\startY);
    \foreach \x in {0,1,...,\maxX}{
        \draw (\size*\x+\startX,\size*0+\startY) -- (\size*\x+\startX,\size*-\maxY+\startY);
    }
    %
    \foreach \y in {0,1,...,\maxY}{
        \draw (\size*0+\startX,\size*-\y+\startY) -- (\size*\maxX+\startX,\size*-\y+\startY);
    }
}
%
\pgfmathsetmacro{\centerX}{\meshsize/2}
\pgfmathsetmacro{\centerY}{-\meshsize/2}
%
\pgfmathsetmacro{\pollcenterX}{\centerX+\pollX}
\pgfmathsetmacro{\pollcenterY}{\centerY+\pollY}
\pgfmathsetmacro{\pollcornerX}{\pollcenterX - \pollsize/2}
\pgfmathsetmacro{\pollcornerY}{\pollcenterY + \pollsize/2}
%
\pgfmathsetmacro{\pAcoordX}{\pollcenterX + \pAX}
\pgfmathsetmacro{\pAcoordY}{\pollcenterY - \pAY}
\pgfmathsetmacro{\pBcoordX}{\pollcenterX + \pBX}
\pgfmathsetmacro{\pBcoordY}{\pollcenterY - \pBY}
\pgfmathsetmacro{\pCcoordX}{\pollcenterX + \pCX}
\pgfmathsetmacro{\pCcoordY}{\pollcenterY - \pCY}
%
\pgfmathsetmacro{\deltameshn}{\pollcenterX-\pollcornerX}
\pgfmathsetmacro{\deltameshd}{\centerX}
\pgfmathsetmacro{\deltamesh}{(\pollcenterX-\pollcornerX)/\centerX}
\pgfmathsetmacro{\deltaframen}{\deltameshn * \deltameshn}
\pgfmathsetmacro{\deltaframed}{\deltameshd * \deltameshd}
\pgfmathsetmacro{\deltaframe}{\deltamesh * \deltamesh}
%
\Grid{0}{0}{\meshsize}{\meshsize}{\unit}{1pt}
\Grid{\pollcornerX*\unit}{\pollcornerY*\unit}{\pollsize}{\pollsize}{\unit}{3pt}
%
\ifdim \deltamesh pt = 1 pt
    \dimline[label style={above=0.5ex,}, line style = {line width=0.7},extension start length=+0.3,extension end length=+0.3] {(\pollcornerX*\unit,1)}{(\pollcenterX*\unit,1)}{\huge $\delta^{k}_\mathrm{poll}=1$};
    
    \dimline[label style={above=0.5ex,}, line style = {line width=0.7},extension start length=+0.3,extension end length=+0.3] {(\meshsize*\unit - 1*\unit,1)}{(\meshsize*\unit,1)}{\huge $\delta^{k}_\mathrm{mesh}=1$};
\else
    \dimline[label style={above=0.5ex,}, line style = {line width=0.7},extension start length=+0.3,extension end length=+0.3] {(\pollcornerX*\unit,0.5)}{(\pollcenterX*\unit,0.5)}{\huge $\delta^{k}_\mathrm{poll}=\addfraction{\deltameshn}{\deltameshd}$};
    
    \dimline[label style={above=0.5ex,}, line style = {line width=0.7},extension start length=+0.3,extension end length=+0.3] {(\meshsize*\unit - 1*\unit,0.5)}{(\meshsize*\unit,0.5)}{\huge $\delta^{k}_\mathrm{mesh}=\addfraction{\deltaframen}{\deltaframed}$};
\fi
    \node[circle, minimum size = 1mm, fill=red, label={[fill=white,above right,label distance=0.2cm]\huge $x_k$}] (center) at (\pollcenterX*\unit,\pollcenterY*\unit) {};
\fi
%
\ifnum\frameopt=0
    % https://tex.stackexchange.com/a/183633
\newcommand{\Grid}[6]{%
    \def\startX{#1}
    \def\startY{#2}
    \def\maxX{#3}
    \def\maxY{#4}
    \def\size{#5}
    \def\borderthick{#6}
    %
    \draw[line width=\borderthick] (\size*0+\startX,\size*0+\startY) rectangle (\size*\maxX+\startX,\size*-\maxY+\startY);
    \foreach \x in {0,1,...,\maxX}{
        \draw (\size*\x+\startX,\size*0+\startY) -- (\size*\x+\startX,\size*-\maxY+\startY);
    }
    %
    \foreach \y in {0,1,...,\maxY}{
        \draw (\size*0+\startX,\size*-\y+\startY) -- (\size*\maxX+\startX,\size*-\y+\startY);
    }
}
%
\pgfmathsetmacro{\centerX}{\meshsize/2}
\pgfmathsetmacro{\centerY}{-\meshsize/2}
%
\pgfmathsetmacro{\pollcenterX}{\centerX+\pollX}
\pgfmathsetmacro{\pollcenterY}{\centerY+\pollY}
\pgfmathsetmacro{\pollcornerX}{\pollcenterX - \pollsize/2}
\pgfmathsetmacro{\pollcornerY}{\pollcenterY + \pollsize/2}
%
\pgfmathsetmacro{\pAcoordX}{\pollcenterX + \pAX}
\pgfmathsetmacro{\pAcoordY}{\pollcenterY - \pAY}
\pgfmathsetmacro{\pBcoordX}{\pollcenterX + \pBX}
\pgfmathsetmacro{\pBcoordY}{\pollcenterY - \pBY}
\pgfmathsetmacro{\pCcoordX}{\pollcenterX + \pCX}
\pgfmathsetmacro{\pCcoordY}{\pollcenterY - \pCY}
%
\pgfmathsetmacro{\deltameshn}{\pollcenterX-\pollcornerX}
\pgfmathsetmacro{\deltameshd}{\centerX}
\pgfmathsetmacro{\deltamesh}{(\pollcenterX-\pollcornerX)/\centerX}
\pgfmathsetmacro{\deltaframen}{\deltameshn * \deltameshn}
\pgfmathsetmacro{\deltaframed}{\deltameshd * \deltameshd}
\pgfmathsetmacro{\deltaframe}{\deltamesh * \deltamesh}
%
\Grid{0}{0}{\meshsize}{\meshsize}{\unit}{1pt}
\Grid{\pollcornerX*\unit}{\pollcornerY*\unit}{\pollsize}{\pollsize}{\unit}{3pt}
%
\ifdim \deltamesh pt = 1 pt
    \dimline[label style={above=0.5ex,}, line style = {line width=0.7},extension start length=+0.3,extension end length=+0.3] {(\pollcornerX*\unit,1)}{(\pollcenterX*\unit,1)}{\huge $\delta^{k}_\mathrm{poll}=1$};
    
    \dimline[label style={above=0.5ex,}, line style = {line width=0.7},extension start length=+0.3,extension end length=+0.3] {(\meshsize*\unit - 1*\unit,1)}{(\meshsize*\unit,1)}{\huge $\delta^{k}_\mathrm{mesh}=1$};
\else
    \dimline[label style={above=0.5ex,}, line style = {line width=0.7},extension start length=+0.3,extension end length=+0.3] {(\pollcornerX*\unit,0.5)}{(\pollcenterX*\unit,0.5)}{\huge $\delta^{k}_\mathrm{poll}=\addfraction{\deltameshn}{\deltameshd}$};
    
    \dimline[label style={above=0.5ex,}, line style = {line width=0.7},extension start length=+0.3,extension end length=+0.3] {(\meshsize*\unit - 1*\unit,0.5)}{(\meshsize*\unit,0.5)}{\huge $\delta^{k}_\mathrm{mesh}=\addfraction{\deltaframen}{\deltaframed}$};
\fi
    \node[circle, minimum size = 1mm, fill=red, label={[fill=white,above right,label distance=0.2cm]\huge $x_k$}] (center) at (\pollcenterX*\unit,\pollcenterY*\unit) {};
    \node[circle, minimum size = 1mm, fill=red, label={[fill=white,below,label distance=-0.5cm]\huge $p_1$}] (p1) at (\pAcoordX*\unit,\pAcoordY*\unit) {};
    \node[circle, minimum size = 1mm, fill=red, label={[fill=white,right,label distance=0.5cm]\huge $p_2$}] (p2) at (\pBcoordX*\unit,\pBcoordY*\unit) {};
    \node[circle, minimum size = 1mm, fill=red, label={[fill=white]\huge $p_3$}] (p3) at (\pCcoordX*\unit,\pCcoordY*\unit) {};
\fi
%
\ifnum\frameopt=1
    % https://tex.stackexchange.com/a/183633
\newcommand{\Grid}[6]{%
    \def\startX{#1}
    \def\startY{#2}
    \def\maxX{#3}
    \def\maxY{#4}
    \def\size{#5}
    \def\borderthick{#6}
    %
    \draw[line width=\borderthick] (\size*0+\startX,\size*0+\startY) rectangle (\size*\maxX+\startX,\size*-\maxY+\startY);
    \foreach \x in {0,1,...,\maxX}{
        \draw (\size*\x+\startX,\size*0+\startY) -- (\size*\x+\startX,\size*-\maxY+\startY);
    }
    %
    \foreach \y in {0,1,...,\maxY}{
        \draw (\size*0+\startX,\size*-\y+\startY) -- (\size*\maxX+\startX,\size*-\y+\startY);
    }
}
%
\pgfmathsetmacro{\centerX}{\meshsize/2}
\pgfmathsetmacro{\centerY}{-\meshsize/2}
%
\pgfmathsetmacro{\pollcenterX}{\centerX+\pollX}
\pgfmathsetmacro{\pollcenterY}{\centerY+\pollY}
\pgfmathsetmacro{\pollcornerX}{\pollcenterX - \pollsize/2}
\pgfmathsetmacro{\pollcornerY}{\pollcenterY + \pollsize/2}
%
\pgfmathsetmacro{\pAcoordX}{\pollcenterX + \pAX}
\pgfmathsetmacro{\pAcoordY}{\pollcenterY - \pAY}
\pgfmathsetmacro{\pBcoordX}{\pollcenterX + \pBX}
\pgfmathsetmacro{\pBcoordY}{\pollcenterY - \pBY}
\pgfmathsetmacro{\pCcoordX}{\pollcenterX + \pCX}
\pgfmathsetmacro{\pCcoordY}{\pollcenterY - \pCY}
%
\pgfmathsetmacro{\deltameshn}{\pollcenterX-\pollcornerX}
\pgfmathsetmacro{\deltameshd}{\centerX}
\pgfmathsetmacro{\deltamesh}{(\pollcenterX-\pollcornerX)/\centerX}
\pgfmathsetmacro{\deltaframen}{\deltameshn * \deltameshn}
\pgfmathsetmacro{\deltaframed}{\deltameshd * \deltameshd}
\pgfmathsetmacro{\deltaframe}{\deltamesh * \deltamesh}
%
\Grid{0}{0}{\meshsize}{\meshsize}{\unit}{1pt}
\Grid{\pollcornerX*\unit}{\pollcornerY*\unit}{\pollsize}{\pollsize}{\unit}{3pt}
%
\ifdim \deltamesh pt = 1 pt
    \dimline[label style={above=0.5ex,}, line style = {line width=0.7},extension start length=+0.3,extension end length=+0.3] {(\pollcornerX*\unit,1)}{(\pollcenterX*\unit,1)}{\huge $\delta^{k}_\mathrm{poll}=1$};
    
    \dimline[label style={above=0.5ex,}, line style = {line width=0.7},extension start length=+0.3,extension end length=+0.3] {(\meshsize*\unit - 1*\unit,1)}{(\meshsize*\unit,1)}{\huge $\delta^{k}_\mathrm{mesh}=1$};
\else
    \dimline[label style={above=0.5ex,}, line style = {line width=0.7},extension start length=+0.3,extension end length=+0.3] {(\pollcornerX*\unit,0.5)}{(\pollcenterX*\unit,0.5)}{\huge $\delta^{k}_\mathrm{poll}=\addfraction{\deltameshn}{\deltameshd}$};
    
    \dimline[label style={above=0.5ex,}, line style = {line width=0.7},extension start length=+0.3,extension end length=+0.3] {(\meshsize*\unit - 1*\unit,0.5)}{(\meshsize*\unit,0.5)}{\huge $\delta^{k}_\mathrm{mesh}=\addfraction{\deltaframen}{\deltaframed}$};
\fi
    \node[align=center, minimum width = 100mm, above right] at (0,3) {\huge Poll \textbf{\color{red}failure}};
    \node[star, star points=8, minimum size = 10mm, fill=blue, label={[fill=white,above right,label distance=0.2cm]\huge $x_k$}] (center) at (\pollcenterX*\unit,\pollcenterY*\unit) {};
    \node[circle, minimum size = 1mm, fill=red, label={[fill=white,below,label distance=-0.5cm]\huge $p_1$}] (p1) at (\pAcoordX*\unit,\pAcoordY*\unit) {};
    \node[circle, minimum size = 1mm, fill=red, label={[fill=white,right,label distance=0.5cm]\huge $p_2$}] (p2) at (\pBcoordX*\unit,\pBcoordY*\unit) {};
    \node[circle, minimum size = 1mm, fill=red, label={[fill=white]\huge $p_3$}] (p3) at (\pCcoordX*\unit,\pCcoordY*\unit) {};
\fi
%
\ifnum\frameopt < 4
    \def\unit{1.25}
    \def\meshsize{8}
    \def\pollX{0}
    \def\pollY{0}
    \def\pollsize{4}
    %
    \def\pAX{-\pollsize/2}
    \def\pAY{-1}
    \def\pBX{\pollsize/2}
    \def\pBY{\pollsize/2}
    \def\pCX{\pollsize/2}
    \def\pCY{-1}
\fi
%
\ifnum\frameopt=2
    % https://tex.stackexchange.com/a/183633
\newcommand{\Grid}[6]{%
    \def\startX{#1}
    \def\startY{#2}
    \def\maxX{#3}
    \def\maxY{#4}
    \def\size{#5}
    \def\borderthick{#6}
    %
    \draw[line width=\borderthick] (\size*0+\startX,\size*0+\startY) rectangle (\size*\maxX+\startX,\size*-\maxY+\startY);
    \foreach \x in {0,1,...,\maxX}{
        \draw (\size*\x+\startX,\size*0+\startY) -- (\size*\x+\startX,\size*-\maxY+\startY);
    }
    %
    \foreach \y in {0,1,...,\maxY}{
        \draw (\size*0+\startX,\size*-\y+\startY) -- (\size*\maxX+\startX,\size*-\y+\startY);
    }
}
%
\pgfmathsetmacro{\centerX}{\meshsize/2}
\pgfmathsetmacro{\centerY}{-\meshsize/2}
%
\pgfmathsetmacro{\pollcenterX}{\centerX+\pollX}
\pgfmathsetmacro{\pollcenterY}{\centerY+\pollY}
\pgfmathsetmacro{\pollcornerX}{\pollcenterX - \pollsize/2}
\pgfmathsetmacro{\pollcornerY}{\pollcenterY + \pollsize/2}
%
\pgfmathsetmacro{\pAcoordX}{\pollcenterX + \pAX}
\pgfmathsetmacro{\pAcoordY}{\pollcenterY - \pAY}
\pgfmathsetmacro{\pBcoordX}{\pollcenterX + \pBX}
\pgfmathsetmacro{\pBcoordY}{\pollcenterY - \pBY}
\pgfmathsetmacro{\pCcoordX}{\pollcenterX + \pCX}
\pgfmathsetmacro{\pCcoordY}{\pollcenterY - \pCY}
%
\pgfmathsetmacro{\deltameshn}{\pollcenterX-\pollcornerX}
\pgfmathsetmacro{\deltameshd}{\centerX}
\pgfmathsetmacro{\deltamesh}{(\pollcenterX-\pollcornerX)/\centerX}
\pgfmathsetmacro{\deltaframen}{\deltameshn * \deltameshn}
\pgfmathsetmacro{\deltaframed}{\deltameshd * \deltameshd}
\pgfmathsetmacro{\deltaframe}{\deltamesh * \deltamesh}
%
\Grid{0}{0}{\meshsize}{\meshsize}{\unit}{1pt}
\Grid{\pollcornerX*\unit}{\pollcornerY*\unit}{\pollsize}{\pollsize}{\unit}{3pt}
%
\ifdim \deltamesh pt = 1 pt
    \dimline[label style={above=0.5ex,}, line style = {line width=0.7},extension start length=+0.3,extension end length=+0.3] {(\pollcornerX*\unit,1)}{(\pollcenterX*\unit,1)}{\huge $\delta^{k}_\mathrm{poll}=1$};
    
    \dimline[label style={above=0.5ex,}, line style = {line width=0.7},extension start length=+0.3,extension end length=+0.3] {(\meshsize*\unit - 1*\unit,1)}{(\meshsize*\unit,1)}{\huge $\delta^{k}_\mathrm{mesh}=1$};
\else
    \dimline[label style={above=0.5ex,}, line style = {line width=0.7},extension start length=+0.3,extension end length=+0.3] {(\pollcornerX*\unit,0.5)}{(\pollcenterX*\unit,0.5)}{\huge $\delta^{k}_\mathrm{poll}=\addfraction{\deltameshn}{\deltameshd}$};
    
    \dimline[label style={above=0.5ex,}, line style = {line width=0.7},extension start length=+0.3,extension end length=+0.3] {(\meshsize*\unit - 1*\unit,0.5)}{(\meshsize*\unit,0.5)}{\huge $\delta^{k}_\mathrm{mesh}=\addfraction{\deltaframen}{\deltaframed}$};
\fi
    \node[circle, minimum size = 1mm, fill=red, label={[fill=white,above right,label distance=0.2cm]\huge $x_k$}] (center) at (\pollcenterX*\unit,\pollcenterY*\unit) {};
    \node[circle, minimum size = 1mm, fill=red, label={[fill=white,below,label distance=-0.5cm]\huge $p_4$}] (p1) at (\pAcoordX*\unit,\pAcoordY*\unit) {};
    \node[circle, minimum size = 1mm, fill=red, label={[fill=white,right,label distance=0.5cm]\huge $p_5$}] (p2) at (\pBcoordX*\unit,\pBcoordY*\unit) {};
    \node[circle, minimum size = 1mm, fill=red, label={[fill=white]\huge $p_6$}] (p3) at (\pCcoordX*\unit,\pCcoordY*\unit) {};
\fi
%
\ifnum\frameopt=3
    % https://tex.stackexchange.com/a/183633
\newcommand{\Grid}[6]{%
    \def\startX{#1}
    \def\startY{#2}
    \def\maxX{#3}
    \def\maxY{#4}
    \def\size{#5}
    \def\borderthick{#6}
    %
    \draw[line width=\borderthick] (\size*0+\startX,\size*0+\startY) rectangle (\size*\maxX+\startX,\size*-\maxY+\startY);
    \foreach \x in {0,1,...,\maxX}{
        \draw (\size*\x+\startX,\size*0+\startY) -- (\size*\x+\startX,\size*-\maxY+\startY);
    }
    %
    \foreach \y in {0,1,...,\maxY}{
        \draw (\size*0+\startX,\size*-\y+\startY) -- (\size*\maxX+\startX,\size*-\y+\startY);
    }
}
%
\pgfmathsetmacro{\centerX}{\meshsize/2}
\pgfmathsetmacro{\centerY}{-\meshsize/2}
%
\pgfmathsetmacro{\pollcenterX}{\centerX+\pollX}
\pgfmathsetmacro{\pollcenterY}{\centerY+\pollY}
\pgfmathsetmacro{\pollcornerX}{\pollcenterX - \pollsize/2}
\pgfmathsetmacro{\pollcornerY}{\pollcenterY + \pollsize/2}
%
\pgfmathsetmacro{\pAcoordX}{\pollcenterX + \pAX}
\pgfmathsetmacro{\pAcoordY}{\pollcenterY - \pAY}
\pgfmathsetmacro{\pBcoordX}{\pollcenterX + \pBX}
\pgfmathsetmacro{\pBcoordY}{\pollcenterY - \pBY}
\pgfmathsetmacro{\pCcoordX}{\pollcenterX + \pCX}
\pgfmathsetmacro{\pCcoordY}{\pollcenterY - \pCY}
%
\pgfmathsetmacro{\deltameshn}{\pollcenterX-\pollcornerX}
\pgfmathsetmacro{\deltameshd}{\centerX}
\pgfmathsetmacro{\deltamesh}{(\pollcenterX-\pollcornerX)/\centerX}
\pgfmathsetmacro{\deltaframen}{\deltameshn * \deltameshn}
\pgfmathsetmacro{\deltaframed}{\deltameshd * \deltameshd}
\pgfmathsetmacro{\deltaframe}{\deltamesh * \deltamesh}
%
\Grid{0}{0}{\meshsize}{\meshsize}{\unit}{1pt}
\Grid{\pollcornerX*\unit}{\pollcornerY*\unit}{\pollsize}{\pollsize}{\unit}{3pt}
%
\ifdim \deltamesh pt = 1 pt
    \dimline[label style={above=0.5ex,}, line style = {line width=0.7},extension start length=+0.3,extension end length=+0.3] {(\pollcornerX*\unit,1)}{(\pollcenterX*\unit,1)}{\huge $\delta^{k}_\mathrm{poll}=1$};
    
    \dimline[label style={above=0.5ex,}, line style = {line width=0.7},extension start length=+0.3,extension end length=+0.3] {(\meshsize*\unit - 1*\unit,1)}{(\meshsize*\unit,1)}{\huge $\delta^{k}_\mathrm{mesh}=1$};
\else
    \dimline[label style={above=0.5ex,}, line style = {line width=0.7},extension start length=+0.3,extension end length=+0.3] {(\pollcornerX*\unit,0.5)}{(\pollcenterX*\unit,0.5)}{\huge $\delta^{k}_\mathrm{poll}=\addfraction{\deltameshn}{\deltameshd}$};
    
    \dimline[label style={above=0.5ex,}, line style = {line width=0.7},extension start length=+0.3,extension end length=+0.3] {(\meshsize*\unit - 1*\unit,0.5)}{(\meshsize*\unit,0.5)}{\huge $\delta^{k}_\mathrm{mesh}=\addfraction{\deltaframen}{\deltaframed}$};
\fi
    \node[align=center, minimum width = 100mm, above right] at (0,2) {\huge Poll \textbf{\color{red}failure}};
    \node[star, star points=8, minimum size = 10mm, fill=blue, label={[fill=white,above right,label distance=0.2cm]\huge $x_k$}] (center) at (\pollcenterX*\unit,\pollcenterY*\unit) {};
    \node[circle, minimum size = 1mm, fill=red, label={[fill=white,below,label distance=-0.5cm]\huge $p_4$}] (p1) at (\pAcoordX*\unit,\pAcoordY*\unit) {};
    \node[circle, minimum size = 1mm, fill=red, label={[fill=white,right,label distance=0.5cm]\huge $p_5$}] (p2) at (\pBcoordX*\unit,\pBcoordY*\unit) {};
    \node[circle, minimum size = 1mm, fill=red, label={[fill=white]\huge $p_6$}] (p3) at (\pCcoordX*\unit,\pCcoordY*\unit) {};
\fi
%
\ifnum\frameopt < 7
    \def\unit{0.3125}
    \def\meshsize{32}
    \def\pollX{0}
    \def\pollY{0}
    \def\pollsize{8}
    %
    \def\pAX{-4}
    \def\pAY{\pollsize/2}
    \def\pBX{\pollsize/2}
    \def\pBY{0}
    \def\pCX{0}
    \def\pCY{-\pollsize/2}
\fi
\ifnum\frameopt = 4
    % https://tex.stackexchange.com/a/183633
\newcommand{\Grid}[6]{%
    \def\startX{#1}
    \def\startY{#2}
    \def\maxX{#3}
    \def\maxY{#4}
    \def\size{#5}
    \def\borderthick{#6}
    %
    \draw[line width=\borderthick] (\size*0+\startX,\size*0+\startY) rectangle (\size*\maxX+\startX,\size*-\maxY+\startY);
    \foreach \x in {0,1,...,\maxX}{
        \draw (\size*\x+\startX,\size*0+\startY) -- (\size*\x+\startX,\size*-\maxY+\startY);
    }
    %
    \foreach \y in {0,1,...,\maxY}{
        \draw (\size*0+\startX,\size*-\y+\startY) -- (\size*\maxX+\startX,\size*-\y+\startY);
    }
}
%
\pgfmathsetmacro{\centerX}{\meshsize/2}
\pgfmathsetmacro{\centerY}{-\meshsize/2}
%
\pgfmathsetmacro{\pollcenterX}{\centerX+\pollX}
\pgfmathsetmacro{\pollcenterY}{\centerY+\pollY}
\pgfmathsetmacro{\pollcornerX}{\pollcenterX - \pollsize/2}
\pgfmathsetmacro{\pollcornerY}{\pollcenterY + \pollsize/2}
%
\pgfmathsetmacro{\pAcoordX}{\pollcenterX + \pAX}
\pgfmathsetmacro{\pAcoordY}{\pollcenterY - \pAY}
\pgfmathsetmacro{\pBcoordX}{\pollcenterX + \pBX}
\pgfmathsetmacro{\pBcoordY}{\pollcenterY - \pBY}
\pgfmathsetmacro{\pCcoordX}{\pollcenterX + \pCX}
\pgfmathsetmacro{\pCcoordY}{\pollcenterY - \pCY}
%
\pgfmathsetmacro{\deltameshn}{\pollcenterX-\pollcornerX}
\pgfmathsetmacro{\deltameshd}{\centerX}
\pgfmathsetmacro{\deltamesh}{(\pollcenterX-\pollcornerX)/\centerX}
\pgfmathsetmacro{\deltaframen}{\deltameshn * \deltameshn}
\pgfmathsetmacro{\deltaframed}{\deltameshd * \deltameshd}
\pgfmathsetmacro{\deltaframe}{\deltamesh * \deltamesh}
%
\Grid{0}{0}{\meshsize}{\meshsize}{\unit}{1pt}
\Grid{\pollcornerX*\unit}{\pollcornerY*\unit}{\pollsize}{\pollsize}{\unit}{3pt}
%
\ifdim \deltamesh pt = 1 pt
    \dimline[label style={above=0.5ex,}, line style = {line width=0.7},extension start length=+0.3,extension end length=+0.3] {(\pollcornerX*\unit,1)}{(\pollcenterX*\unit,1)}{\huge $\delta^{k}_\mathrm{poll}=1$};
    
    \dimline[label style={above=0.5ex,}, line style = {line width=0.7},extension start length=+0.3,extension end length=+0.3] {(\meshsize*\unit - 1*\unit,1)}{(\meshsize*\unit,1)}{\huge $\delta^{k}_\mathrm{mesh}=1$};
\else
    \dimline[label style={above=0.5ex,}, line style = {line width=0.7},extension start length=+0.3,extension end length=+0.3] {(\pollcornerX*\unit,0.5)}{(\pollcenterX*\unit,0.5)}{\huge $\delta^{k}_\mathrm{poll}=\addfraction{\deltameshn}{\deltameshd}$};
    
    \dimline[label style={above=0.5ex,}, line style = {line width=0.7},extension start length=+0.3,extension end length=+0.3] {(\meshsize*\unit - 1*\unit,0.5)}{(\meshsize*\unit,0.5)}{\huge $\delta^{k}_\mathrm{mesh}=\addfraction{\deltaframen}{\deltaframed}$};
\fi
    \node[circle, minimum size = 1mm, fill=red, label={[fill=white,above right,label distance=0.2cm]\huge $x_k$}] (center) at (\pollcenterX*\unit,\pollcenterY*\unit) {};
    \node[circle, minimum size = 1mm, fill=red, label={[fill=white,below,label distance=-0.5cm]\huge $p_7$}] (p1) at (\pAcoordX*\unit,\pAcoordY*\unit) {};
    \node[circle, minimum size = 1mm, fill=red, label={[fill=white,right,label distance=0.5cm]\huge $p_8$}] (p2) at (\pBcoordX*\unit,\pBcoordY*\unit) {};
    \node[circle, minimum size = 1mm, fill=red, label={[fill=white]\huge $p_9$}] (p3) at (\pCcoordX*\unit,\pCcoordY*\unit) {};
\fi
%
\ifnum\frameopt = 5
    % https://tex.stackexchange.com/a/183633
\newcommand{\Grid}[6]{%
    \def\startX{#1}
    \def\startY{#2}
    \def\maxX{#3}
    \def\maxY{#4}
    \def\size{#5}
    \def\borderthick{#6}
    %
    \draw[line width=\borderthick] (\size*0+\startX,\size*0+\startY) rectangle (\size*\maxX+\startX,\size*-\maxY+\startY);
    \foreach \x in {0,1,...,\maxX}{
        \draw (\size*\x+\startX,\size*0+\startY) -- (\size*\x+\startX,\size*-\maxY+\startY);
    }
    %
    \foreach \y in {0,1,...,\maxY}{
        \draw (\size*0+\startX,\size*-\y+\startY) -- (\size*\maxX+\startX,\size*-\y+\startY);
    }
}
%
\pgfmathsetmacro{\centerX}{\meshsize/2}
\pgfmathsetmacro{\centerY}{-\meshsize/2}
%
\pgfmathsetmacro{\pollcenterX}{\centerX+\pollX}
\pgfmathsetmacro{\pollcenterY}{\centerY+\pollY}
\pgfmathsetmacro{\pollcornerX}{\pollcenterX - \pollsize/2}
\pgfmathsetmacro{\pollcornerY}{\pollcenterY + \pollsize/2}
%
\pgfmathsetmacro{\pAcoordX}{\pollcenterX + \pAX}
\pgfmathsetmacro{\pAcoordY}{\pollcenterY - \pAY}
\pgfmathsetmacro{\pBcoordX}{\pollcenterX + \pBX}
\pgfmathsetmacro{\pBcoordY}{\pollcenterY - \pBY}
\pgfmathsetmacro{\pCcoordX}{\pollcenterX + \pCX}
\pgfmathsetmacro{\pCcoordY}{\pollcenterY - \pCY}
%
\pgfmathsetmacro{\deltameshn}{\pollcenterX-\pollcornerX}
\pgfmathsetmacro{\deltameshd}{\centerX}
\pgfmathsetmacro{\deltamesh}{(\pollcenterX-\pollcornerX)/\centerX}
\pgfmathsetmacro{\deltaframen}{\deltameshn * \deltameshn}
\pgfmathsetmacro{\deltaframed}{\deltameshd * \deltameshd}
\pgfmathsetmacro{\deltaframe}{\deltamesh * \deltamesh}
%
\Grid{0}{0}{\meshsize}{\meshsize}{\unit}{1pt}
\Grid{\pollcornerX*\unit}{\pollcornerY*\unit}{\pollsize}{\pollsize}{\unit}{3pt}
%
\ifdim \deltamesh pt = 1 pt
    \dimline[label style={above=0.5ex,}, line style = {line width=0.7},extension start length=+0.3,extension end length=+0.3] {(\pollcornerX*\unit,1)}{(\pollcenterX*\unit,1)}{\huge $\delta^{k}_\mathrm{poll}=1$};
    
    \dimline[label style={above=0.5ex,}, line style = {line width=0.7},extension start length=+0.3,extension end length=+0.3] {(\meshsize*\unit - 1*\unit,1)}{(\meshsize*\unit,1)}{\huge $\delta^{k}_\mathrm{mesh}=1$};
\else
    \dimline[label style={above=0.5ex,}, line style = {line width=0.7},extension start length=+0.3,extension end length=+0.3] {(\pollcornerX*\unit,0.5)}{(\pollcenterX*\unit,0.5)}{\huge $\delta^{k}_\mathrm{poll}=\addfraction{\deltameshn}{\deltameshd}$};
    
    \dimline[label style={above=0.5ex,}, line style = {line width=0.7},extension start length=+0.3,extension end length=+0.3] {(\meshsize*\unit - 1*\unit,0.5)}{(\meshsize*\unit,0.5)}{\huge $\delta^{k}_\mathrm{mesh}=\addfraction{\deltaframen}{\deltaframed}$};
\fi
    \node[align=center, minimum width = 100mm, above right] at (0,2) {\huge Poll \textbf{\color{red}failure}};
    \node[star, star points=8, minimum size = 6mm, fill=blue, label={[fill=white,above right,label distance=0.2cm]\huge $x_k$}] (center) at (\pollcenterX*\unit,\pollcenterY*\unit) {};
    \node[circle, minimum size = 1mm, fill=red, label={[fill=white,below,label distance=-0.5cm]\huge $p_7$}] (p1) at (\pAcoordX*\unit,\pAcoordY*\unit) {};
    \node[circle, minimum size = 1mm, fill=red, label={[fill=white,right,label distance=0.5cm]\huge $p_8$}] (p2) at (\pBcoordX*\unit,\pBcoordY*\unit) {};
    \node[circle, minimum size = 1mm, fill=red, label={[fill=white]\huge $p_9$}] (p3) at (\pCcoordX*\unit,\pCcoordY*\unit) {};
\fi
%
\ifnum\frameopt = 6
    % https://tex.stackexchange.com/a/183633
\newcommand{\Grid}[6]{%
    \def\startX{#1}
    \def\startY{#2}
    \def\maxX{#3}
    \def\maxY{#4}
    \def\size{#5}
    \def\borderthick{#6}
    %
    \draw[line width=\borderthick] (\size*0+\startX,\size*0+\startY) rectangle (\size*\maxX+\startX,\size*-\maxY+\startY);
    \foreach \x in {0,1,...,\maxX}{
        \draw (\size*\x+\startX,\size*0+\startY) -- (\size*\x+\startX,\size*-\maxY+\startY);
    }
    %
    \foreach \y in {0,1,...,\maxY}{
        \draw (\size*0+\startX,\size*-\y+\startY) -- (\size*\maxX+\startX,\size*-\y+\startY);
    }
}
%
\pgfmathsetmacro{\centerX}{\meshsize/2}
\pgfmathsetmacro{\centerY}{-\meshsize/2}
%
\pgfmathsetmacro{\pollcenterX}{\centerX+\pollX}
\pgfmathsetmacro{\pollcenterY}{\centerY+\pollY}
\pgfmathsetmacro{\pollcornerX}{\pollcenterX - \pollsize/2}
\pgfmathsetmacro{\pollcornerY}{\pollcenterY + \pollsize/2}
%
\pgfmathsetmacro{\pAcoordX}{\pollcenterX + \pAX}
\pgfmathsetmacro{\pAcoordY}{\pollcenterY - \pAY}
\pgfmathsetmacro{\pBcoordX}{\pollcenterX + \pBX}
\pgfmathsetmacro{\pBcoordY}{\pollcenterY - \pBY}
\pgfmathsetmacro{\pCcoordX}{\pollcenterX + \pCX}
\pgfmathsetmacro{\pCcoordY}{\pollcenterY - \pCY}
%
\pgfmathsetmacro{\deltameshn}{\pollcenterX-\pollcornerX}
\pgfmathsetmacro{\deltameshd}{\centerX}
\pgfmathsetmacro{\deltamesh}{(\pollcenterX-\pollcornerX)/\centerX}
\pgfmathsetmacro{\deltaframen}{\deltameshn * \deltameshn}
\pgfmathsetmacro{\deltaframed}{\deltameshd * \deltameshd}
\pgfmathsetmacro{\deltaframe}{\deltamesh * \deltamesh}
%
\Grid{0}{0}{\meshsize}{\meshsize}{\unit}{1pt}
\Grid{\pollcornerX*\unit}{\pollcornerY*\unit}{\pollsize}{\pollsize}{\unit}{3pt}
%
\ifdim \deltamesh pt = 1 pt
    \dimline[label style={above=0.5ex,}, line style = {line width=0.7},extension start length=+0.3,extension end length=+0.3] {(\pollcornerX*\unit,1)}{(\pollcenterX*\unit,1)}{\huge $\delta^{k}_\mathrm{poll}=1$};
    
    \dimline[label style={above=0.5ex,}, line style = {line width=0.7},extension start length=+0.3,extension end length=+0.3] {(\meshsize*\unit - 1*\unit,1)}{(\meshsize*\unit,1)}{\huge $\delta^{k}_\mathrm{mesh}=1$};
\else
    \dimline[label style={above=0.5ex,}, line style = {line width=0.7},extension start length=+0.3,extension end length=+0.3] {(\pollcornerX*\unit,0.5)}{(\pollcenterX*\unit,0.5)}{\huge $\delta^{k}_\mathrm{poll}=\addfraction{\deltameshn}{\deltameshd}$};
    
    \dimline[label style={above=0.5ex,}, line style = {line width=0.7},extension start length=+0.3,extension end length=+0.3] {(\meshsize*\unit - 1*\unit,0.5)}{(\meshsize*\unit,0.5)}{\huge $\delta^{k}_\mathrm{mesh}=\addfraction{\deltaframen}{\deltaframed}$};
\fi
    \node[align=center, minimum width = 100mm, above right] at (0,2) {\huge Poll \textbf{\color{darkgreen}success}};
    \node[circle, minimum size = 1mm, fill=red, label={[fill=white,above right,label distance=0.2cm]\huge $x_k$}] (center) at (\pollcenterX*\unit,\pollcenterY*\unit) {};
    \node[star, star points=8, minimum size = 6mm, fill=blue, label={[fill=white,below,label distance=-0.5cm]\huge $p_7$}] (p1) at (\pAcoordX*\unit,\pAcoordY*\unit) {};
    \node[circle, minimum size = 1mm, fill=red, label={[fill=white,right,label distance=0.5cm]\huge $p_8$}] (p2) at (\pBcoordX*\unit,\pBcoordY*\unit) {};
    \node[circle, minimum size = 1mm, fill=red, label={[fill=white]\huge $p_9$}] (p3) at (\pCcoordX*\unit,\pCcoordY*\unit) {};
\fi
%
\ifnum\frameopt < 8
    \def\unit{0.625}
    \def\meshsize{16}
    \def\pollX{-2}
    \def\pollY{-2}
    \def\pollsize{8}
    %
    \def\pAX{3}
    \def\pAY{-\pollsize/2}
    \def\pBX{-\pollsize/2}
    \def\pBY{0}
    \def\pCX{2}
    \def\pCY{\pollsize/2}
\fi
%
\ifnum\frameopt = 7
    % https://tex.stackexchange.com/a/183633
\newcommand{\Grid}[6]{%
    \def\startX{#1}
    \def\startY{#2}
    \def\maxX{#3}
    \def\maxY{#4}
    \def\size{#5}
    \def\borderthick{#6}
    %
    \draw[line width=\borderthick] (\size*0+\startX,\size*0+\startY) rectangle (\size*\maxX+\startX,\size*-\maxY+\startY);
    \foreach \x in {0,1,...,\maxX}{
        \draw (\size*\x+\startX,\size*0+\startY) -- (\size*\x+\startX,\size*-\maxY+\startY);
    }
    %
    \foreach \y in {0,1,...,\maxY}{
        \draw (\size*0+\startX,\size*-\y+\startY) -- (\size*\maxX+\startX,\size*-\y+\startY);
    }
}
%
\pgfmathsetmacro{\centerX}{\meshsize/2}
\pgfmathsetmacro{\centerY}{-\meshsize/2}
%
\pgfmathsetmacro{\pollcenterX}{\centerX+\pollX}
\pgfmathsetmacro{\pollcenterY}{\centerY+\pollY}
\pgfmathsetmacro{\pollcornerX}{\pollcenterX - \pollsize/2}
\pgfmathsetmacro{\pollcornerY}{\pollcenterY + \pollsize/2}
%
\pgfmathsetmacro{\pAcoordX}{\pollcenterX + \pAX}
\pgfmathsetmacro{\pAcoordY}{\pollcenterY - \pAY}
\pgfmathsetmacro{\pBcoordX}{\pollcenterX + \pBX}
\pgfmathsetmacro{\pBcoordY}{\pollcenterY - \pBY}
\pgfmathsetmacro{\pCcoordX}{\pollcenterX + \pCX}
\pgfmathsetmacro{\pCcoordY}{\pollcenterY - \pCY}
%
\pgfmathsetmacro{\deltameshn}{\pollcenterX-\pollcornerX}
\pgfmathsetmacro{\deltameshd}{\centerX}
\pgfmathsetmacro{\deltamesh}{(\pollcenterX-\pollcornerX)/\centerX}
\pgfmathsetmacro{\deltaframen}{\deltameshn * \deltameshn}
\pgfmathsetmacro{\deltaframed}{\deltameshd * \deltameshd}
\pgfmathsetmacro{\deltaframe}{\deltamesh * \deltamesh}
%
\Grid{0}{0}{\meshsize}{\meshsize}{\unit}{1pt}
\Grid{\pollcornerX*\unit}{\pollcornerY*\unit}{\pollsize}{\pollsize}{\unit}{3pt}
%
\ifdim \deltamesh pt = 1 pt
    \dimline[label style={above=0.5ex,}, line style = {line width=0.7},extension start length=+0.3,extension end length=+0.3] {(\pollcornerX*\unit,1)}{(\pollcenterX*\unit,1)}{\huge $\delta^{k}_\mathrm{poll}=1$};
    
    \dimline[label style={above=0.5ex,}, line style = {line width=0.7},extension start length=+0.3,extension end length=+0.3] {(\meshsize*\unit - 1*\unit,1)}{(\meshsize*\unit,1)}{\huge $\delta^{k}_\mathrm{mesh}=1$};
\else
    \dimline[label style={above=0.5ex,}, line style = {line width=0.7},extension start length=+0.3,extension end length=+0.3] {(\pollcornerX*\unit,0.5)}{(\pollcenterX*\unit,0.5)}{\huge $\delta^{k}_\mathrm{poll}=\addfraction{\deltameshn}{\deltameshd}$};
    
    \dimline[label style={above=0.5ex,}, line style = {line width=0.7},extension start length=+0.3,extension end length=+0.3] {(\meshsize*\unit - 1*\unit,0.5)}{(\meshsize*\unit,0.5)}{\huge $\delta^{k}_\mathrm{mesh}=\addfraction{\deltaframen}{\deltaframed}$};
\fi
    \node[circle, minimum size = 1mm, fill=red, label={[fill=white,above right,label distance=0.2cm]\huge $x_k$}] (center) at (\pollcenterX*\unit,\pollcenterY*\unit) {};
    \node[circle, minimum size = 1mm, fill=red, label={[fill=white,below,label distance=-0.5cm]\huge $p_{10}$}] (p1) at (\pAcoordX*\unit,\pAcoordY*\unit) {};
    \node[circle, minimum size = 1mm, fill=red, label={[fill=white,right,label distance=0.5cm]\huge $p_{11}$}] (p2) at (\pBcoordX*\unit,\pBcoordY*\unit) {};
    \node[circle, minimum size = 1mm, fill=red, label={[fill=white]\huge $p_{12}$}] (p3) at (\pCcoordX*\unit,\pCcoordY*\unit) {};
\fi
%
\ifnum\frameopt < 9
    \def\unit{0.3125}
    \def\meshsize{32}
    \def\pollX{-4}
    \def\pollY{-4}
    \def\pollsize{8}
    %
    \def\pAX{-2}
    \def\pAY{-\pollsize/2}
    \def\pBX{\pollsize/2}
    \def\pBY{-\pollsize/2}
    \def\pCX{\pollsize/2}
    \def\pCY{\pollsize/2}
\fi
%
\ifnum\frameopt = 8
    % https://tex.stackexchange.com/a/183633
\newcommand{\Grid}[6]{%
    \def\startX{#1}
    \def\startY{#2}
    \def\maxX{#3}
    \def\maxY{#4}
    \def\size{#5}
    \def\borderthick{#6}
    %
    \draw[line width=\borderthick] (\size*0+\startX,\size*0+\startY) rectangle (\size*\maxX+\startX,\size*-\maxY+\startY);
    \foreach \x in {0,1,...,\maxX}{
        \draw (\size*\x+\startX,\size*0+\startY) -- (\size*\x+\startX,\size*-\maxY+\startY);
    }
    %
    \foreach \y in {0,1,...,\maxY}{
        \draw (\size*0+\startX,\size*-\y+\startY) -- (\size*\maxX+\startX,\size*-\y+\startY);
    }
}
%
\pgfmathsetmacro{\centerX}{\meshsize/2}
\pgfmathsetmacro{\centerY}{-\meshsize/2}
%
\pgfmathsetmacro{\pollcenterX}{\centerX+\pollX}
\pgfmathsetmacro{\pollcenterY}{\centerY+\pollY}
\pgfmathsetmacro{\pollcornerX}{\pollcenterX - \pollsize/2}
\pgfmathsetmacro{\pollcornerY}{\pollcenterY + \pollsize/2}
%
\pgfmathsetmacro{\pAcoordX}{\pollcenterX + \pAX}
\pgfmathsetmacro{\pAcoordY}{\pollcenterY - \pAY}
\pgfmathsetmacro{\pBcoordX}{\pollcenterX + \pBX}
\pgfmathsetmacro{\pBcoordY}{\pollcenterY - \pBY}
\pgfmathsetmacro{\pCcoordX}{\pollcenterX + \pCX}
\pgfmathsetmacro{\pCcoordY}{\pollcenterY - \pCY}
%
\pgfmathsetmacro{\deltameshn}{\pollcenterX-\pollcornerX}
\pgfmathsetmacro{\deltameshd}{\centerX}
\pgfmathsetmacro{\deltamesh}{(\pollcenterX-\pollcornerX)/\centerX}
\pgfmathsetmacro{\deltaframen}{\deltameshn * \deltameshn}
\pgfmathsetmacro{\deltaframed}{\deltameshd * \deltameshd}
\pgfmathsetmacro{\deltaframe}{\deltamesh * \deltamesh}
%
\Grid{0}{0}{\meshsize}{\meshsize}{\unit}{1pt}
\Grid{\pollcornerX*\unit}{\pollcornerY*\unit}{\pollsize}{\pollsize}{\unit}{3pt}
%
\ifdim \deltamesh pt = 1 pt
    \dimline[label style={above=0.5ex,}, line style = {line width=0.7},extension start length=+0.3,extension end length=+0.3] {(\pollcornerX*\unit,1)}{(\pollcenterX*\unit,1)}{\huge $\delta^{k}_\mathrm{poll}=1$};
    
    \dimline[label style={above=0.5ex,}, line style = {line width=0.7},extension start length=+0.3,extension end length=+0.3] {(\meshsize*\unit - 1*\unit,1)}{(\meshsize*\unit,1)}{\huge $\delta^{k}_\mathrm{mesh}=1$};
\else
    \dimline[label style={above=0.5ex,}, line style = {line width=0.7},extension start length=+0.3,extension end length=+0.3] {(\pollcornerX*\unit,0.5)}{(\pollcenterX*\unit,0.5)}{\huge $\delta^{k}_\mathrm{poll}=\addfraction{\deltameshn}{\deltameshd}$};
    
    \dimline[label style={above=0.5ex,}, line style = {line width=0.7},extension start length=+0.3,extension end length=+0.3] {(\meshsize*\unit - 1*\unit,0.5)}{(\meshsize*\unit,0.5)}{\huge $\delta^{k}_\mathrm{mesh}=\addfraction{\deltaframen}{\deltaframed}$};
\fi
    \node[align=center, minimum width = 100mm, above right] at (0,2) {\huge Poll \textbf{\color{red}failure}};
    \node[circle, minimum size = 1mm, fill=red, label={[fill=white,above right,label distance=0.2cm]\huge $x_k$}] (center) at (\pollcenterX*\unit,\pollcenterY*\unit) {};
    \node[circle, minimum size = 1mm, fill=red, label={[fill=white,below,label distance=-0.5cm]\huge $p_{13}$}] (p1) at (\pAcoordX*\unit,\pAcoordY*\unit) {};
    \node[circle, minimum size = 1mm, fill=red, label={[fill=white,right,label distance=0.5cm]\huge $p_{14}$}] (p2) at (\pBcoordX*\unit,\pBcoordY*\unit) {};
    \node[circle, minimum size = 1mm, fill=red, label={[fill=white]\huge $p_{15}$}] (p3) at (\pCcoordX*\unit,\pCcoordY*\unit) {};
\fi
%
\ifnum\frameopt < 10
    \def\unit{0.15625}
    \def\meshsize{64}
    \def\pollX{-8}
    \def\pollY{-8}
    \def\pollsize{8}
    %
    \def\pAX{1}
    \def\pAY{\pollsize/2}
    \def\pBX{-\pollsize/2}
    \def\pBY{0}
    \def\pCX{\pollsize/2}
    \def\pCY{-\pollsize/2}
\fi
%
\ifnum\frameopt = 9
    % https://tex.stackexchange.com/a/183633
\newcommand{\Grid}[6]{%
    \def\startX{#1}
    \def\startY{#2}
    \def\maxX{#3}
    \def\maxY{#4}
    \def\size{#5}
    \def\borderthick{#6}
    %
    \draw[line width=\borderthick] (\size*0+\startX,\size*0+\startY) rectangle (\size*\maxX+\startX,\size*-\maxY+\startY);
    \foreach \x in {0,1,...,\maxX}{
        \draw (\size*\x+\startX,\size*0+\startY) -- (\size*\x+\startX,\size*-\maxY+\startY);
    }
    %
    \foreach \y in {0,1,...,\maxY}{
        \draw (\size*0+\startX,\size*-\y+\startY) -- (\size*\maxX+\startX,\size*-\y+\startY);
    }
}
%
\pgfmathsetmacro{\centerX}{\meshsize/2}
\pgfmathsetmacro{\centerY}{-\meshsize/2}
%
\pgfmathsetmacro{\pollcenterX}{\centerX+\pollX}
\pgfmathsetmacro{\pollcenterY}{\centerY+\pollY}
\pgfmathsetmacro{\pollcornerX}{\pollcenterX - \pollsize/2}
\pgfmathsetmacro{\pollcornerY}{\pollcenterY + \pollsize/2}
%
\pgfmathsetmacro{\pAcoordX}{\pollcenterX + \pAX}
\pgfmathsetmacro{\pAcoordY}{\pollcenterY - \pAY}
\pgfmathsetmacro{\pBcoordX}{\pollcenterX + \pBX}
\pgfmathsetmacro{\pBcoordY}{\pollcenterY - \pBY}
\pgfmathsetmacro{\pCcoordX}{\pollcenterX + \pCX}
\pgfmathsetmacro{\pCcoordY}{\pollcenterY - \pCY}
%
\pgfmathsetmacro{\deltameshn}{\pollcenterX-\pollcornerX}
\pgfmathsetmacro{\deltameshd}{\centerX}
\pgfmathsetmacro{\deltamesh}{(\pollcenterX-\pollcornerX)/\centerX}
\pgfmathsetmacro{\deltaframen}{\deltameshn * \deltameshn}
\pgfmathsetmacro{\deltaframed}{\deltameshd * \deltameshd}
\pgfmathsetmacro{\deltaframe}{\deltamesh * \deltamesh}
%
\Grid{0}{0}{\meshsize}{\meshsize}{\unit}{1pt}
\Grid{\pollcornerX*\unit}{\pollcornerY*\unit}{\pollsize}{\pollsize}{\unit}{3pt}
%
\ifdim \deltamesh pt = 1 pt
    \dimline[label style={above=0.5ex,}, line style = {line width=0.7},extension start length=+0.3,extension end length=+0.3] {(\pollcornerX*\unit,1)}{(\pollcenterX*\unit,1)}{\huge $\delta^{k}_\mathrm{poll}=1$};
    
    \dimline[label style={above=0.5ex,}, line style = {line width=0.7},extension start length=+0.3,extension end length=+0.3] {(\meshsize*\unit - 1*\unit,1)}{(\meshsize*\unit,1)}{\huge $\delta^{k}_\mathrm{mesh}=1$};
\else
    \dimline[label style={above=0.5ex,}, line style = {line width=0.7},extension start length=+0.3,extension end length=+0.3] {(\pollcornerX*\unit,0.5)}{(\pollcenterX*\unit,0.5)}{\huge $\delta^{k}_\mathrm{poll}=\addfraction{\deltameshn}{\deltameshd}$};
    
    \dimline[label style={above=0.5ex,}, line style = {line width=0.7},extension start length=+0.3,extension end length=+0.3] {(\meshsize*\unit - 1*\unit,0.5)}{(\meshsize*\unit,0.5)}{\huge $\delta^{k}_\mathrm{mesh}=\addfraction{\deltaframen}{\deltaframed}$};
\fi
    \node[align=center, minimum width = 100mm, above right] at (0,2) {\huge Poll \textbf{\color{red}failure}};
    \node[circle, minimum size = 1mm, fill=red] (center) at (\pollcenterX*\unit,\pollcenterY*\unit) {};
    \node[circle, minimum size = 1mm, fill=red] (p1) at (\pAcoordX*\unit,\pAcoordY*\unit) {};
    \node[circle, minimum size = 1mm, fill=red] (p2) at (\pBcoordX*\unit,\pBcoordY*\unit) {};
    \node[circle, minimum size = 1mm, fill=red] (p3) at (\pCcoordX*\unit,\pCcoordY*\unit) {};
\fi
%
\ifnum\frameopt > -1
    \draw[-, color=red, line width=2pt, opacity=1.0] (center) -- (p1);
    \draw[-, color=red, line width=2pt, opacity=1.0] (center) -- (p2);
    \draw[-, color=red, line width=2pt, opacity=1.0] (center) -- (p3);
\fi}%
				\end{tikzpicture}%
			}%
		\end{column}
	
	\end{columns}
	\vspace{-3em}
\end{frame}
\addtocounter{footnote}{-1}
%------------------------------------------------
\begin{frame}[t,label=abm_6]
	\frametitle{Overview of stochastic mesh adaptive direct search}
	\tikzstyle{background grid}=[draw, black!50,step=.5cm]
	%
	\uncover<1->{No gradient information available, blackbox is expensive and \emphasis{noisy}\ifshowcitations\footpartcite{Audet2020}\fi}%
	\only<9->{{\color{white}\ifshowcitations\footpartcite{Dzahini2022}\fi}}\\
	%
	\begin{columns}[t] % The "c" option specifies centered vertical alignment while the "t" option is used for top vertical alignment
		\begin{column}{.42\textwidth} % Left column and width
			\vspace{-1.2em}
			% Optimization problem
			\begin{exampleblock}{Objective and constraints}
				\vspace{-1.2em}
				\begin{equation*}
					\begin{aligned}
						& \underset{\mathbf{x}}{\text{min}}
						& & f(\mathbf{x}) = \mathbb{E}_{\Theta}\left[{f}_{\Theta}(\mathbf{x})\right]\\
						& \text{subject to}
						& & {c}(\mathbf{x}) = \mathbb{E}_{\Theta}\left[{c}_{\Theta}(\mathbf{x})\right] \le 0\\
						& \text{where}
						& & \mathbf{x}:\text{ variables}~\Theta\mathrm{:realizations}
					\end{aligned}
				\end{equation*}
			\end{exampleblock}
			\only<-8>{
				\small
				\begin{itemize}\itemsep0em
					\item Constructs estimates of objective:
					%
					\vspace{-1em}
					\begin{equation*} \label{eq:samplerate}
						f^{k} = \dfrac{1}{n^k}\sum_{i=1}^{n^k} f_{\Theta_{0,i}}({x}_k)
					\end{equation*}
					\vspace{-1em}
					%
					\item<2-> $n^k$ is the sampling rate
					\item<3-> Tracks uncertainty interval $\mathcal{I}(\delta^k_\text{poll})$ in the estimate
				\end{itemize}
			}%
			\only<9->{~\\~\\\centering Constraint handling using the \textit{progressive barrier} approach \footnotemark[2]}%
		\end{column}
		%
		\begin{column}{.5\textwidth} % Left column and width
			\tikzstyle{background grid}=[draw, black!50,step=.5cm]
			
			\only<-2>{\def\frameopt{2}}%
			\only<3>{\def\frameopts{1}}%
			\only<4>{\def\frameopts{2}}%
			\only<5>{\def\frameopts{3}}%
			\only<6>{\def\frameopts{4}}%
			\only<7>{\def\frameopts{5}}%
			\only<8->{\def\frameopts{6}}%

			\tikzstyle{background grid}=[draw, black!50,step=.1cm]
			\hspace*{5em}\raisebox{-1em}{%
				\begin{tikzpicture}[scale=0.45, every node/.style={scale=0.45}, remember picture, overlay] %show background grid, 
					\only<-2>{\ifnum\frameopt < 2
    \def\unit{5}
    \def\meshsize{2}
    \def\pollX{0}
    \def\pollY{0}
    \def\pollsize{2}
    %
    \def\pAX{-1}
    \def\pAY{\pollsize/2}
    \def\pBX{\pollsize/2}
    \def\pBY{0}
    \def\pCX{0}
    \def\pCY{-\pollsize/2}
\fi
%
\ifnum\frameopt=-2
    % https://tex.stackexchange.com/a/183633
\newcommand{\Grid}[6]{%
    \def\startX{#1}
    \def\startY{#2}
    \def\maxX{#3}
    \def\maxY{#4}
    \def\size{#5}
    \def\borderthick{#6}
    %
    \draw[line width=\borderthick] (\size*0+\startX,\size*0+\startY) rectangle (\size*\maxX+\startX,\size*-\maxY+\startY);
    \foreach \x in {0,1,...,\maxX}{
        \draw (\size*\x+\startX,\size*0+\startY) -- (\size*\x+\startX,\size*-\maxY+\startY);
    }
    %
    \foreach \y in {0,1,...,\maxY}{
        \draw (\size*0+\startX,\size*-\y+\startY) -- (\size*\maxX+\startX,\size*-\y+\startY);
    }
}
%
\pgfmathsetmacro{\centerX}{\meshsize/2}
\pgfmathsetmacro{\centerY}{-\meshsize/2}
%
\pgfmathsetmacro{\pollcenterX}{\centerX+\pollX}
\pgfmathsetmacro{\pollcenterY}{\centerY+\pollY}
\pgfmathsetmacro{\pollcornerX}{\pollcenterX - \pollsize/2}
\pgfmathsetmacro{\pollcornerY}{\pollcenterY + \pollsize/2}
%
\pgfmathsetmacro{\pAcoordX}{\pollcenterX + \pAX}
\pgfmathsetmacro{\pAcoordY}{\pollcenterY - \pAY}
\pgfmathsetmacro{\pBcoordX}{\pollcenterX + \pBX}
\pgfmathsetmacro{\pBcoordY}{\pollcenterY - \pBY}
\pgfmathsetmacro{\pCcoordX}{\pollcenterX + \pCX}
\pgfmathsetmacro{\pCcoordY}{\pollcenterY - \pCY}
%
\pgfmathsetmacro{\deltameshn}{\pollcenterX-\pollcornerX}
\pgfmathsetmacro{\deltameshd}{\centerX}
\pgfmathsetmacro{\deltamesh}{(\pollcenterX-\pollcornerX)/\centerX}
\pgfmathsetmacro{\deltaframen}{\deltameshn * \deltameshn}
\pgfmathsetmacro{\deltaframed}{\deltameshd * \deltameshd}
\pgfmathsetmacro{\deltaframe}{\deltamesh * \deltamesh}
%
\Grid{0}{0}{\meshsize}{\meshsize}{\unit}{1pt}
\Grid{\pollcornerX*\unit}{\pollcornerY*\unit}{\pollsize}{\pollsize}{\unit}{3pt}
%
\ifdim \deltamesh pt = 1 pt
    \dimline[label style={above=0.5ex,}, line style = {line width=0.7},extension start length=+0.3,extension end length=+0.3] {(\pollcornerX*\unit,1)}{(\pollcenterX*\unit,1)}{\huge $\delta^{k}_\mathrm{poll}=1$};
    
    \dimline[label style={above=0.5ex,}, line style = {line width=0.7},extension start length=+0.3,extension end length=+0.3] {(\meshsize*\unit - 1*\unit,1)}{(\meshsize*\unit,1)}{\huge $\delta^{k}_\mathrm{mesh}=1$};
\else
    \dimline[label style={above=0.5ex,}, line style = {line width=0.7},extension start length=+0.3,extension end length=+0.3] {(\pollcornerX*\unit,0.5)}{(\pollcenterX*\unit,0.5)}{\huge $\delta^{k}_\mathrm{poll}=\addfraction{\deltameshn}{\deltameshd}$};
    
    \dimline[label style={above=0.5ex,}, line style = {line width=0.7},extension start length=+0.3,extension end length=+0.3] {(\meshsize*\unit - 1*\unit,0.5)}{(\meshsize*\unit,0.5)}{\huge $\delta^{k}_\mathrm{mesh}=\addfraction{\deltaframen}{\deltaframed}$};
\fi
\fi
%
\ifnum\frameopt=-1
    % https://tex.stackexchange.com/a/183633
\newcommand{\Grid}[6]{%
    \def\startX{#1}
    \def\startY{#2}
    \def\maxX{#3}
    \def\maxY{#4}
    \def\size{#5}
    \def\borderthick{#6}
    %
    \draw[line width=\borderthick] (\size*0+\startX,\size*0+\startY) rectangle (\size*\maxX+\startX,\size*-\maxY+\startY);
    \foreach \x in {0,1,...,\maxX}{
        \draw (\size*\x+\startX,\size*0+\startY) -- (\size*\x+\startX,\size*-\maxY+\startY);
    }
    %
    \foreach \y in {0,1,...,\maxY}{
        \draw (\size*0+\startX,\size*-\y+\startY) -- (\size*\maxX+\startX,\size*-\y+\startY);
    }
}
%
\pgfmathsetmacro{\centerX}{\meshsize/2}
\pgfmathsetmacro{\centerY}{-\meshsize/2}
%
\pgfmathsetmacro{\pollcenterX}{\centerX+\pollX}
\pgfmathsetmacro{\pollcenterY}{\centerY+\pollY}
\pgfmathsetmacro{\pollcornerX}{\pollcenterX - \pollsize/2}
\pgfmathsetmacro{\pollcornerY}{\pollcenterY + \pollsize/2}
%
\pgfmathsetmacro{\pAcoordX}{\pollcenterX + \pAX}
\pgfmathsetmacro{\pAcoordY}{\pollcenterY - \pAY}
\pgfmathsetmacro{\pBcoordX}{\pollcenterX + \pBX}
\pgfmathsetmacro{\pBcoordY}{\pollcenterY - \pBY}
\pgfmathsetmacro{\pCcoordX}{\pollcenterX + \pCX}
\pgfmathsetmacro{\pCcoordY}{\pollcenterY - \pCY}
%
\pgfmathsetmacro{\deltameshn}{\pollcenterX-\pollcornerX}
\pgfmathsetmacro{\deltameshd}{\centerX}
\pgfmathsetmacro{\deltamesh}{(\pollcenterX-\pollcornerX)/\centerX}
\pgfmathsetmacro{\deltaframen}{\deltameshn * \deltameshn}
\pgfmathsetmacro{\deltaframed}{\deltameshd * \deltameshd}
\pgfmathsetmacro{\deltaframe}{\deltamesh * \deltamesh}
%
\Grid{0}{0}{\meshsize}{\meshsize}{\unit}{1pt}
\Grid{\pollcornerX*\unit}{\pollcornerY*\unit}{\pollsize}{\pollsize}{\unit}{3pt}
%
\ifdim \deltamesh pt = 1 pt
    \dimline[label style={above=0.5ex,}, line style = {line width=0.7},extension start length=+0.3,extension end length=+0.3] {(\pollcornerX*\unit,1)}{(\pollcenterX*\unit,1)}{\huge $\delta^{k}_\mathrm{poll}=1$};
    
    \dimline[label style={above=0.5ex,}, line style = {line width=0.7},extension start length=+0.3,extension end length=+0.3] {(\meshsize*\unit - 1*\unit,1)}{(\meshsize*\unit,1)}{\huge $\delta^{k}_\mathrm{mesh}=1$};
\else
    \dimline[label style={above=0.5ex,}, line style = {line width=0.7},extension start length=+0.3,extension end length=+0.3] {(\pollcornerX*\unit,0.5)}{(\pollcenterX*\unit,0.5)}{\huge $\delta^{k}_\mathrm{poll}=\addfraction{\deltameshn}{\deltameshd}$};
    
    \dimline[label style={above=0.5ex,}, line style = {line width=0.7},extension start length=+0.3,extension end length=+0.3] {(\meshsize*\unit - 1*\unit,0.5)}{(\meshsize*\unit,0.5)}{\huge $\delta^{k}_\mathrm{mesh}=\addfraction{\deltaframen}{\deltaframed}$};
\fi
    \node[circle, minimum size = 1mm, fill=red, label={[fill=white,above right,label distance=0.2cm]\huge $x_k$}] (center) at (\pollcenterX*\unit,\pollcenterY*\unit) {};
\fi
%
\ifnum\frameopt=0
    % https://tex.stackexchange.com/a/183633
\newcommand{\Grid}[6]{%
    \def\startX{#1}
    \def\startY{#2}
    \def\maxX{#3}
    \def\maxY{#4}
    \def\size{#5}
    \def\borderthick{#6}
    %
    \draw[line width=\borderthick] (\size*0+\startX,\size*0+\startY) rectangle (\size*\maxX+\startX,\size*-\maxY+\startY);
    \foreach \x in {0,1,...,\maxX}{
        \draw (\size*\x+\startX,\size*0+\startY) -- (\size*\x+\startX,\size*-\maxY+\startY);
    }
    %
    \foreach \y in {0,1,...,\maxY}{
        \draw (\size*0+\startX,\size*-\y+\startY) -- (\size*\maxX+\startX,\size*-\y+\startY);
    }
}
%
\pgfmathsetmacro{\centerX}{\meshsize/2}
\pgfmathsetmacro{\centerY}{-\meshsize/2}
%
\pgfmathsetmacro{\pollcenterX}{\centerX+\pollX}
\pgfmathsetmacro{\pollcenterY}{\centerY+\pollY}
\pgfmathsetmacro{\pollcornerX}{\pollcenterX - \pollsize/2}
\pgfmathsetmacro{\pollcornerY}{\pollcenterY + \pollsize/2}
%
\pgfmathsetmacro{\pAcoordX}{\pollcenterX + \pAX}
\pgfmathsetmacro{\pAcoordY}{\pollcenterY - \pAY}
\pgfmathsetmacro{\pBcoordX}{\pollcenterX + \pBX}
\pgfmathsetmacro{\pBcoordY}{\pollcenterY - \pBY}
\pgfmathsetmacro{\pCcoordX}{\pollcenterX + \pCX}
\pgfmathsetmacro{\pCcoordY}{\pollcenterY - \pCY}
%
\pgfmathsetmacro{\deltameshn}{\pollcenterX-\pollcornerX}
\pgfmathsetmacro{\deltameshd}{\centerX}
\pgfmathsetmacro{\deltamesh}{(\pollcenterX-\pollcornerX)/\centerX}
\pgfmathsetmacro{\deltaframen}{\deltameshn * \deltameshn}
\pgfmathsetmacro{\deltaframed}{\deltameshd * \deltameshd}
\pgfmathsetmacro{\deltaframe}{\deltamesh * \deltamesh}
%
\Grid{0}{0}{\meshsize}{\meshsize}{\unit}{1pt}
\Grid{\pollcornerX*\unit}{\pollcornerY*\unit}{\pollsize}{\pollsize}{\unit}{3pt}
%
\ifdim \deltamesh pt = 1 pt
    \dimline[label style={above=0.5ex,}, line style = {line width=0.7},extension start length=+0.3,extension end length=+0.3] {(\pollcornerX*\unit,1)}{(\pollcenterX*\unit,1)}{\huge $\delta^{k}_\mathrm{poll}=1$};
    
    \dimline[label style={above=0.5ex,}, line style = {line width=0.7},extension start length=+0.3,extension end length=+0.3] {(\meshsize*\unit - 1*\unit,1)}{(\meshsize*\unit,1)}{\huge $\delta^{k}_\mathrm{mesh}=1$};
\else
    \dimline[label style={above=0.5ex,}, line style = {line width=0.7},extension start length=+0.3,extension end length=+0.3] {(\pollcornerX*\unit,0.5)}{(\pollcenterX*\unit,0.5)}{\huge $\delta^{k}_\mathrm{poll}=\addfraction{\deltameshn}{\deltameshd}$};
    
    \dimline[label style={above=0.5ex,}, line style = {line width=0.7},extension start length=+0.3,extension end length=+0.3] {(\meshsize*\unit - 1*\unit,0.5)}{(\meshsize*\unit,0.5)}{\huge $\delta^{k}_\mathrm{mesh}=\addfraction{\deltaframen}{\deltaframed}$};
\fi
    \node[circle, minimum size = 1mm, fill=red, label={[fill=white,above right,label distance=0.2cm]\huge $x_k$}] (center) at (\pollcenterX*\unit,\pollcenterY*\unit) {};
    \node[circle, minimum size = 1mm, fill=red, label={[fill=white,below,label distance=-0.5cm]\huge $p_1$}] (p1) at (\pAcoordX*\unit,\pAcoordY*\unit) {};
    \node[circle, minimum size = 1mm, fill=red, label={[fill=white,right,label distance=0.5cm]\huge $p_2$}] (p2) at (\pBcoordX*\unit,\pBcoordY*\unit) {};
    \node[circle, minimum size = 1mm, fill=red, label={[fill=white]\huge $p_3$}] (p3) at (\pCcoordX*\unit,\pCcoordY*\unit) {};
\fi
%
\ifnum\frameopt=1
    % https://tex.stackexchange.com/a/183633
\newcommand{\Grid}[6]{%
    \def\startX{#1}
    \def\startY{#2}
    \def\maxX{#3}
    \def\maxY{#4}
    \def\size{#5}
    \def\borderthick{#6}
    %
    \draw[line width=\borderthick] (\size*0+\startX,\size*0+\startY) rectangle (\size*\maxX+\startX,\size*-\maxY+\startY);
    \foreach \x in {0,1,...,\maxX}{
        \draw (\size*\x+\startX,\size*0+\startY) -- (\size*\x+\startX,\size*-\maxY+\startY);
    }
    %
    \foreach \y in {0,1,...,\maxY}{
        \draw (\size*0+\startX,\size*-\y+\startY) -- (\size*\maxX+\startX,\size*-\y+\startY);
    }
}
%
\pgfmathsetmacro{\centerX}{\meshsize/2}
\pgfmathsetmacro{\centerY}{-\meshsize/2}
%
\pgfmathsetmacro{\pollcenterX}{\centerX+\pollX}
\pgfmathsetmacro{\pollcenterY}{\centerY+\pollY}
\pgfmathsetmacro{\pollcornerX}{\pollcenterX - \pollsize/2}
\pgfmathsetmacro{\pollcornerY}{\pollcenterY + \pollsize/2}
%
\pgfmathsetmacro{\pAcoordX}{\pollcenterX + \pAX}
\pgfmathsetmacro{\pAcoordY}{\pollcenterY - \pAY}
\pgfmathsetmacro{\pBcoordX}{\pollcenterX + \pBX}
\pgfmathsetmacro{\pBcoordY}{\pollcenterY - \pBY}
\pgfmathsetmacro{\pCcoordX}{\pollcenterX + \pCX}
\pgfmathsetmacro{\pCcoordY}{\pollcenterY - \pCY}
%
\pgfmathsetmacro{\deltameshn}{\pollcenterX-\pollcornerX}
\pgfmathsetmacro{\deltameshd}{\centerX}
\pgfmathsetmacro{\deltamesh}{(\pollcenterX-\pollcornerX)/\centerX}
\pgfmathsetmacro{\deltaframen}{\deltameshn * \deltameshn}
\pgfmathsetmacro{\deltaframed}{\deltameshd * \deltameshd}
\pgfmathsetmacro{\deltaframe}{\deltamesh * \deltamesh}
%
\Grid{0}{0}{\meshsize}{\meshsize}{\unit}{1pt}
\Grid{\pollcornerX*\unit}{\pollcornerY*\unit}{\pollsize}{\pollsize}{\unit}{3pt}
%
\ifdim \deltamesh pt = 1 pt
    \dimline[label style={above=0.5ex,}, line style = {line width=0.7},extension start length=+0.3,extension end length=+0.3] {(\pollcornerX*\unit,1)}{(\pollcenterX*\unit,1)}{\huge $\delta^{k}_\mathrm{poll}=1$};
    
    \dimline[label style={above=0.5ex,}, line style = {line width=0.7},extension start length=+0.3,extension end length=+0.3] {(\meshsize*\unit - 1*\unit,1)}{(\meshsize*\unit,1)}{\huge $\delta^{k}_\mathrm{mesh}=1$};
\else
    \dimline[label style={above=0.5ex,}, line style = {line width=0.7},extension start length=+0.3,extension end length=+0.3] {(\pollcornerX*\unit,0.5)}{(\pollcenterX*\unit,0.5)}{\huge $\delta^{k}_\mathrm{poll}=\addfraction{\deltameshn}{\deltameshd}$};
    
    \dimline[label style={above=0.5ex,}, line style = {line width=0.7},extension start length=+0.3,extension end length=+0.3] {(\meshsize*\unit - 1*\unit,0.5)}{(\meshsize*\unit,0.5)}{\huge $\delta^{k}_\mathrm{mesh}=\addfraction{\deltaframen}{\deltaframed}$};
\fi
    \node[align=center, minimum width = 100mm, above right] at (0,3) {\huge Poll \textbf{\color{red}failure}};
    \node[star, star points=8, minimum size = 10mm, fill=blue, label={[fill=white,above right,label distance=0.2cm]\huge $x_k$}] (center) at (\pollcenterX*\unit,\pollcenterY*\unit) {};
    \node[circle, minimum size = 1mm, fill=red, label={[fill=white,below,label distance=-0.5cm]\huge $p_1$}] (p1) at (\pAcoordX*\unit,\pAcoordY*\unit) {};
    \node[circle, minimum size = 1mm, fill=red, label={[fill=white,right,label distance=0.5cm]\huge $p_2$}] (p2) at (\pBcoordX*\unit,\pBcoordY*\unit) {};
    \node[circle, minimum size = 1mm, fill=red, label={[fill=white]\huge $p_3$}] (p3) at (\pCcoordX*\unit,\pCcoordY*\unit) {};
\fi
%
\ifnum\frameopt < 4
    \def\unit{1.25}
    \def\meshsize{8}
    \def\pollX{0}
    \def\pollY{0}
    \def\pollsize{4}
    %
    \def\pAX{-\pollsize/2}
    \def\pAY{-1}
    \def\pBX{\pollsize/2}
    \def\pBY{\pollsize/2}
    \def\pCX{\pollsize/2}
    \def\pCY{-1}
\fi
%
\ifnum\frameopt=2
    % https://tex.stackexchange.com/a/183633
\newcommand{\Grid}[6]{%
    \def\startX{#1}
    \def\startY{#2}
    \def\maxX{#3}
    \def\maxY{#4}
    \def\size{#5}
    \def\borderthick{#6}
    %
    \draw[line width=\borderthick] (\size*0+\startX,\size*0+\startY) rectangle (\size*\maxX+\startX,\size*-\maxY+\startY);
    \foreach \x in {0,1,...,\maxX}{
        \draw (\size*\x+\startX,\size*0+\startY) -- (\size*\x+\startX,\size*-\maxY+\startY);
    }
    %
    \foreach \y in {0,1,...,\maxY}{
        \draw (\size*0+\startX,\size*-\y+\startY) -- (\size*\maxX+\startX,\size*-\y+\startY);
    }
}
%
\pgfmathsetmacro{\centerX}{\meshsize/2}
\pgfmathsetmacro{\centerY}{-\meshsize/2}
%
\pgfmathsetmacro{\pollcenterX}{\centerX+\pollX}
\pgfmathsetmacro{\pollcenterY}{\centerY+\pollY}
\pgfmathsetmacro{\pollcornerX}{\pollcenterX - \pollsize/2}
\pgfmathsetmacro{\pollcornerY}{\pollcenterY + \pollsize/2}
%
\pgfmathsetmacro{\pAcoordX}{\pollcenterX + \pAX}
\pgfmathsetmacro{\pAcoordY}{\pollcenterY - \pAY}
\pgfmathsetmacro{\pBcoordX}{\pollcenterX + \pBX}
\pgfmathsetmacro{\pBcoordY}{\pollcenterY - \pBY}
\pgfmathsetmacro{\pCcoordX}{\pollcenterX + \pCX}
\pgfmathsetmacro{\pCcoordY}{\pollcenterY - \pCY}
%
\pgfmathsetmacro{\deltameshn}{\pollcenterX-\pollcornerX}
\pgfmathsetmacro{\deltameshd}{\centerX}
\pgfmathsetmacro{\deltamesh}{(\pollcenterX-\pollcornerX)/\centerX}
\pgfmathsetmacro{\deltaframen}{\deltameshn * \deltameshn}
\pgfmathsetmacro{\deltaframed}{\deltameshd * \deltameshd}
\pgfmathsetmacro{\deltaframe}{\deltamesh * \deltamesh}
%
\Grid{0}{0}{\meshsize}{\meshsize}{\unit}{1pt}
\Grid{\pollcornerX*\unit}{\pollcornerY*\unit}{\pollsize}{\pollsize}{\unit}{3pt}
%
\ifdim \deltamesh pt = 1 pt
    \dimline[label style={above=0.5ex,}, line style = {line width=0.7},extension start length=+0.3,extension end length=+0.3] {(\pollcornerX*\unit,1)}{(\pollcenterX*\unit,1)}{\huge $\delta^{k}_\mathrm{poll}=1$};
    
    \dimline[label style={above=0.5ex,}, line style = {line width=0.7},extension start length=+0.3,extension end length=+0.3] {(\meshsize*\unit - 1*\unit,1)}{(\meshsize*\unit,1)}{\huge $\delta^{k}_\mathrm{mesh}=1$};
\else
    \dimline[label style={above=0.5ex,}, line style = {line width=0.7},extension start length=+0.3,extension end length=+0.3] {(\pollcornerX*\unit,0.5)}{(\pollcenterX*\unit,0.5)}{\huge $\delta^{k}_\mathrm{poll}=\addfraction{\deltameshn}{\deltameshd}$};
    
    \dimline[label style={above=0.5ex,}, line style = {line width=0.7},extension start length=+0.3,extension end length=+0.3] {(\meshsize*\unit - 1*\unit,0.5)}{(\meshsize*\unit,0.5)}{\huge $\delta^{k}_\mathrm{mesh}=\addfraction{\deltaframen}{\deltaframed}$};
\fi
    \node[circle, minimum size = 1mm, fill=red, label={[fill=white,above right,label distance=0.2cm]\huge $x_k$}] (center) at (\pollcenterX*\unit,\pollcenterY*\unit) {};
    \node[circle, minimum size = 1mm, fill=red, label={[fill=white,below,label distance=-0.5cm]\huge $p_4$}] (p1) at (\pAcoordX*\unit,\pAcoordY*\unit) {};
    \node[circle, minimum size = 1mm, fill=red, label={[fill=white,right,label distance=0.5cm]\huge $p_5$}] (p2) at (\pBcoordX*\unit,\pBcoordY*\unit) {};
    \node[circle, minimum size = 1mm, fill=red, label={[fill=white]\huge $p_6$}] (p3) at (\pCcoordX*\unit,\pCcoordY*\unit) {};
\fi
%
\ifnum\frameopt=3
    % https://tex.stackexchange.com/a/183633
\newcommand{\Grid}[6]{%
    \def\startX{#1}
    \def\startY{#2}
    \def\maxX{#3}
    \def\maxY{#4}
    \def\size{#5}
    \def\borderthick{#6}
    %
    \draw[line width=\borderthick] (\size*0+\startX,\size*0+\startY) rectangle (\size*\maxX+\startX,\size*-\maxY+\startY);
    \foreach \x in {0,1,...,\maxX}{
        \draw (\size*\x+\startX,\size*0+\startY) -- (\size*\x+\startX,\size*-\maxY+\startY);
    }
    %
    \foreach \y in {0,1,...,\maxY}{
        \draw (\size*0+\startX,\size*-\y+\startY) -- (\size*\maxX+\startX,\size*-\y+\startY);
    }
}
%
\pgfmathsetmacro{\centerX}{\meshsize/2}
\pgfmathsetmacro{\centerY}{-\meshsize/2}
%
\pgfmathsetmacro{\pollcenterX}{\centerX+\pollX}
\pgfmathsetmacro{\pollcenterY}{\centerY+\pollY}
\pgfmathsetmacro{\pollcornerX}{\pollcenterX - \pollsize/2}
\pgfmathsetmacro{\pollcornerY}{\pollcenterY + \pollsize/2}
%
\pgfmathsetmacro{\pAcoordX}{\pollcenterX + \pAX}
\pgfmathsetmacro{\pAcoordY}{\pollcenterY - \pAY}
\pgfmathsetmacro{\pBcoordX}{\pollcenterX + \pBX}
\pgfmathsetmacro{\pBcoordY}{\pollcenterY - \pBY}
\pgfmathsetmacro{\pCcoordX}{\pollcenterX + \pCX}
\pgfmathsetmacro{\pCcoordY}{\pollcenterY - \pCY}
%
\pgfmathsetmacro{\deltameshn}{\pollcenterX-\pollcornerX}
\pgfmathsetmacro{\deltameshd}{\centerX}
\pgfmathsetmacro{\deltamesh}{(\pollcenterX-\pollcornerX)/\centerX}
\pgfmathsetmacro{\deltaframen}{\deltameshn * \deltameshn}
\pgfmathsetmacro{\deltaframed}{\deltameshd * \deltameshd}
\pgfmathsetmacro{\deltaframe}{\deltamesh * \deltamesh}
%
\Grid{0}{0}{\meshsize}{\meshsize}{\unit}{1pt}
\Grid{\pollcornerX*\unit}{\pollcornerY*\unit}{\pollsize}{\pollsize}{\unit}{3pt}
%
\ifdim \deltamesh pt = 1 pt
    \dimline[label style={above=0.5ex,}, line style = {line width=0.7},extension start length=+0.3,extension end length=+0.3] {(\pollcornerX*\unit,1)}{(\pollcenterX*\unit,1)}{\huge $\delta^{k}_\mathrm{poll}=1$};
    
    \dimline[label style={above=0.5ex,}, line style = {line width=0.7},extension start length=+0.3,extension end length=+0.3] {(\meshsize*\unit - 1*\unit,1)}{(\meshsize*\unit,1)}{\huge $\delta^{k}_\mathrm{mesh}=1$};
\else
    \dimline[label style={above=0.5ex,}, line style = {line width=0.7},extension start length=+0.3,extension end length=+0.3] {(\pollcornerX*\unit,0.5)}{(\pollcenterX*\unit,0.5)}{\huge $\delta^{k}_\mathrm{poll}=\addfraction{\deltameshn}{\deltameshd}$};
    
    \dimline[label style={above=0.5ex,}, line style = {line width=0.7},extension start length=+0.3,extension end length=+0.3] {(\meshsize*\unit - 1*\unit,0.5)}{(\meshsize*\unit,0.5)}{\huge $\delta^{k}_\mathrm{mesh}=\addfraction{\deltaframen}{\deltaframed}$};
\fi
    \node[align=center, minimum width = 100mm, above right] at (0,2) {\huge Poll \textbf{\color{red}failure}};
    \node[star, star points=8, minimum size = 10mm, fill=blue, label={[fill=white,above right,label distance=0.2cm]\huge $x_k$}] (center) at (\pollcenterX*\unit,\pollcenterY*\unit) {};
    \node[circle, minimum size = 1mm, fill=red, label={[fill=white,below,label distance=-0.5cm]\huge $p_4$}] (p1) at (\pAcoordX*\unit,\pAcoordY*\unit) {};
    \node[circle, minimum size = 1mm, fill=red, label={[fill=white,right,label distance=0.5cm]\huge $p_5$}] (p2) at (\pBcoordX*\unit,\pBcoordY*\unit) {};
    \node[circle, minimum size = 1mm, fill=red, label={[fill=white]\huge $p_6$}] (p3) at (\pCcoordX*\unit,\pCcoordY*\unit) {};
\fi
%
\ifnum\frameopt < 7
    \def\unit{0.3125}
    \def\meshsize{32}
    \def\pollX{0}
    \def\pollY{0}
    \def\pollsize{8}
    %
    \def\pAX{-4}
    \def\pAY{\pollsize/2}
    \def\pBX{\pollsize/2}
    \def\pBY{0}
    \def\pCX{0}
    \def\pCY{-\pollsize/2}
\fi
\ifnum\frameopt = 4
    % https://tex.stackexchange.com/a/183633
\newcommand{\Grid}[6]{%
    \def\startX{#1}
    \def\startY{#2}
    \def\maxX{#3}
    \def\maxY{#4}
    \def\size{#5}
    \def\borderthick{#6}
    %
    \draw[line width=\borderthick] (\size*0+\startX,\size*0+\startY) rectangle (\size*\maxX+\startX,\size*-\maxY+\startY);
    \foreach \x in {0,1,...,\maxX}{
        \draw (\size*\x+\startX,\size*0+\startY) -- (\size*\x+\startX,\size*-\maxY+\startY);
    }
    %
    \foreach \y in {0,1,...,\maxY}{
        \draw (\size*0+\startX,\size*-\y+\startY) -- (\size*\maxX+\startX,\size*-\y+\startY);
    }
}
%
\pgfmathsetmacro{\centerX}{\meshsize/2}
\pgfmathsetmacro{\centerY}{-\meshsize/2}
%
\pgfmathsetmacro{\pollcenterX}{\centerX+\pollX}
\pgfmathsetmacro{\pollcenterY}{\centerY+\pollY}
\pgfmathsetmacro{\pollcornerX}{\pollcenterX - \pollsize/2}
\pgfmathsetmacro{\pollcornerY}{\pollcenterY + \pollsize/2}
%
\pgfmathsetmacro{\pAcoordX}{\pollcenterX + \pAX}
\pgfmathsetmacro{\pAcoordY}{\pollcenterY - \pAY}
\pgfmathsetmacro{\pBcoordX}{\pollcenterX + \pBX}
\pgfmathsetmacro{\pBcoordY}{\pollcenterY - \pBY}
\pgfmathsetmacro{\pCcoordX}{\pollcenterX + \pCX}
\pgfmathsetmacro{\pCcoordY}{\pollcenterY - \pCY}
%
\pgfmathsetmacro{\deltameshn}{\pollcenterX-\pollcornerX}
\pgfmathsetmacro{\deltameshd}{\centerX}
\pgfmathsetmacro{\deltamesh}{(\pollcenterX-\pollcornerX)/\centerX}
\pgfmathsetmacro{\deltaframen}{\deltameshn * \deltameshn}
\pgfmathsetmacro{\deltaframed}{\deltameshd * \deltameshd}
\pgfmathsetmacro{\deltaframe}{\deltamesh * \deltamesh}
%
\Grid{0}{0}{\meshsize}{\meshsize}{\unit}{1pt}
\Grid{\pollcornerX*\unit}{\pollcornerY*\unit}{\pollsize}{\pollsize}{\unit}{3pt}
%
\ifdim \deltamesh pt = 1 pt
    \dimline[label style={above=0.5ex,}, line style = {line width=0.7},extension start length=+0.3,extension end length=+0.3] {(\pollcornerX*\unit,1)}{(\pollcenterX*\unit,1)}{\huge $\delta^{k}_\mathrm{poll}=1$};
    
    \dimline[label style={above=0.5ex,}, line style = {line width=0.7},extension start length=+0.3,extension end length=+0.3] {(\meshsize*\unit - 1*\unit,1)}{(\meshsize*\unit,1)}{\huge $\delta^{k}_\mathrm{mesh}=1$};
\else
    \dimline[label style={above=0.5ex,}, line style = {line width=0.7},extension start length=+0.3,extension end length=+0.3] {(\pollcornerX*\unit,0.5)}{(\pollcenterX*\unit,0.5)}{\huge $\delta^{k}_\mathrm{poll}=\addfraction{\deltameshn}{\deltameshd}$};
    
    \dimline[label style={above=0.5ex,}, line style = {line width=0.7},extension start length=+0.3,extension end length=+0.3] {(\meshsize*\unit - 1*\unit,0.5)}{(\meshsize*\unit,0.5)}{\huge $\delta^{k}_\mathrm{mesh}=\addfraction{\deltaframen}{\deltaframed}$};
\fi
    \node[circle, minimum size = 1mm, fill=red, label={[fill=white,above right,label distance=0.2cm]\huge $x_k$}] (center) at (\pollcenterX*\unit,\pollcenterY*\unit) {};
    \node[circle, minimum size = 1mm, fill=red, label={[fill=white,below,label distance=-0.5cm]\huge $p_7$}] (p1) at (\pAcoordX*\unit,\pAcoordY*\unit) {};
    \node[circle, minimum size = 1mm, fill=red, label={[fill=white,right,label distance=0.5cm]\huge $p_8$}] (p2) at (\pBcoordX*\unit,\pBcoordY*\unit) {};
    \node[circle, minimum size = 1mm, fill=red, label={[fill=white]\huge $p_9$}] (p3) at (\pCcoordX*\unit,\pCcoordY*\unit) {};
\fi
%
\ifnum\frameopt = 5
    % https://tex.stackexchange.com/a/183633
\newcommand{\Grid}[6]{%
    \def\startX{#1}
    \def\startY{#2}
    \def\maxX{#3}
    \def\maxY{#4}
    \def\size{#5}
    \def\borderthick{#6}
    %
    \draw[line width=\borderthick] (\size*0+\startX,\size*0+\startY) rectangle (\size*\maxX+\startX,\size*-\maxY+\startY);
    \foreach \x in {0,1,...,\maxX}{
        \draw (\size*\x+\startX,\size*0+\startY) -- (\size*\x+\startX,\size*-\maxY+\startY);
    }
    %
    \foreach \y in {0,1,...,\maxY}{
        \draw (\size*0+\startX,\size*-\y+\startY) -- (\size*\maxX+\startX,\size*-\y+\startY);
    }
}
%
\pgfmathsetmacro{\centerX}{\meshsize/2}
\pgfmathsetmacro{\centerY}{-\meshsize/2}
%
\pgfmathsetmacro{\pollcenterX}{\centerX+\pollX}
\pgfmathsetmacro{\pollcenterY}{\centerY+\pollY}
\pgfmathsetmacro{\pollcornerX}{\pollcenterX - \pollsize/2}
\pgfmathsetmacro{\pollcornerY}{\pollcenterY + \pollsize/2}
%
\pgfmathsetmacro{\pAcoordX}{\pollcenterX + \pAX}
\pgfmathsetmacro{\pAcoordY}{\pollcenterY - \pAY}
\pgfmathsetmacro{\pBcoordX}{\pollcenterX + \pBX}
\pgfmathsetmacro{\pBcoordY}{\pollcenterY - \pBY}
\pgfmathsetmacro{\pCcoordX}{\pollcenterX + \pCX}
\pgfmathsetmacro{\pCcoordY}{\pollcenterY - \pCY}
%
\pgfmathsetmacro{\deltameshn}{\pollcenterX-\pollcornerX}
\pgfmathsetmacro{\deltameshd}{\centerX}
\pgfmathsetmacro{\deltamesh}{(\pollcenterX-\pollcornerX)/\centerX}
\pgfmathsetmacro{\deltaframen}{\deltameshn * \deltameshn}
\pgfmathsetmacro{\deltaframed}{\deltameshd * \deltameshd}
\pgfmathsetmacro{\deltaframe}{\deltamesh * \deltamesh}
%
\Grid{0}{0}{\meshsize}{\meshsize}{\unit}{1pt}
\Grid{\pollcornerX*\unit}{\pollcornerY*\unit}{\pollsize}{\pollsize}{\unit}{3pt}
%
\ifdim \deltamesh pt = 1 pt
    \dimline[label style={above=0.5ex,}, line style = {line width=0.7},extension start length=+0.3,extension end length=+0.3] {(\pollcornerX*\unit,1)}{(\pollcenterX*\unit,1)}{\huge $\delta^{k}_\mathrm{poll}=1$};
    
    \dimline[label style={above=0.5ex,}, line style = {line width=0.7},extension start length=+0.3,extension end length=+0.3] {(\meshsize*\unit - 1*\unit,1)}{(\meshsize*\unit,1)}{\huge $\delta^{k}_\mathrm{mesh}=1$};
\else
    \dimline[label style={above=0.5ex,}, line style = {line width=0.7},extension start length=+0.3,extension end length=+0.3] {(\pollcornerX*\unit,0.5)}{(\pollcenterX*\unit,0.5)}{\huge $\delta^{k}_\mathrm{poll}=\addfraction{\deltameshn}{\deltameshd}$};
    
    \dimline[label style={above=0.5ex,}, line style = {line width=0.7},extension start length=+0.3,extension end length=+0.3] {(\meshsize*\unit - 1*\unit,0.5)}{(\meshsize*\unit,0.5)}{\huge $\delta^{k}_\mathrm{mesh}=\addfraction{\deltaframen}{\deltaframed}$};
\fi
    \node[align=center, minimum width = 100mm, above right] at (0,2) {\huge Poll \textbf{\color{red}failure}};
    \node[star, star points=8, minimum size = 6mm, fill=blue, label={[fill=white,above right,label distance=0.2cm]\huge $x_k$}] (center) at (\pollcenterX*\unit,\pollcenterY*\unit) {};
    \node[circle, minimum size = 1mm, fill=red, label={[fill=white,below,label distance=-0.5cm]\huge $p_7$}] (p1) at (\pAcoordX*\unit,\pAcoordY*\unit) {};
    \node[circle, minimum size = 1mm, fill=red, label={[fill=white,right,label distance=0.5cm]\huge $p_8$}] (p2) at (\pBcoordX*\unit,\pBcoordY*\unit) {};
    \node[circle, minimum size = 1mm, fill=red, label={[fill=white]\huge $p_9$}] (p3) at (\pCcoordX*\unit,\pCcoordY*\unit) {};
\fi
%
\ifnum\frameopt = 6
    % https://tex.stackexchange.com/a/183633
\newcommand{\Grid}[6]{%
    \def\startX{#1}
    \def\startY{#2}
    \def\maxX{#3}
    \def\maxY{#4}
    \def\size{#5}
    \def\borderthick{#6}
    %
    \draw[line width=\borderthick] (\size*0+\startX,\size*0+\startY) rectangle (\size*\maxX+\startX,\size*-\maxY+\startY);
    \foreach \x in {0,1,...,\maxX}{
        \draw (\size*\x+\startX,\size*0+\startY) -- (\size*\x+\startX,\size*-\maxY+\startY);
    }
    %
    \foreach \y in {0,1,...,\maxY}{
        \draw (\size*0+\startX,\size*-\y+\startY) -- (\size*\maxX+\startX,\size*-\y+\startY);
    }
}
%
\pgfmathsetmacro{\centerX}{\meshsize/2}
\pgfmathsetmacro{\centerY}{-\meshsize/2}
%
\pgfmathsetmacro{\pollcenterX}{\centerX+\pollX}
\pgfmathsetmacro{\pollcenterY}{\centerY+\pollY}
\pgfmathsetmacro{\pollcornerX}{\pollcenterX - \pollsize/2}
\pgfmathsetmacro{\pollcornerY}{\pollcenterY + \pollsize/2}
%
\pgfmathsetmacro{\pAcoordX}{\pollcenterX + \pAX}
\pgfmathsetmacro{\pAcoordY}{\pollcenterY - \pAY}
\pgfmathsetmacro{\pBcoordX}{\pollcenterX + \pBX}
\pgfmathsetmacro{\pBcoordY}{\pollcenterY - \pBY}
\pgfmathsetmacro{\pCcoordX}{\pollcenterX + \pCX}
\pgfmathsetmacro{\pCcoordY}{\pollcenterY - \pCY}
%
\pgfmathsetmacro{\deltameshn}{\pollcenterX-\pollcornerX}
\pgfmathsetmacro{\deltameshd}{\centerX}
\pgfmathsetmacro{\deltamesh}{(\pollcenterX-\pollcornerX)/\centerX}
\pgfmathsetmacro{\deltaframen}{\deltameshn * \deltameshn}
\pgfmathsetmacro{\deltaframed}{\deltameshd * \deltameshd}
\pgfmathsetmacro{\deltaframe}{\deltamesh * \deltamesh}
%
\Grid{0}{0}{\meshsize}{\meshsize}{\unit}{1pt}
\Grid{\pollcornerX*\unit}{\pollcornerY*\unit}{\pollsize}{\pollsize}{\unit}{3pt}
%
\ifdim \deltamesh pt = 1 pt
    \dimline[label style={above=0.5ex,}, line style = {line width=0.7},extension start length=+0.3,extension end length=+0.3] {(\pollcornerX*\unit,1)}{(\pollcenterX*\unit,1)}{\huge $\delta^{k}_\mathrm{poll}=1$};
    
    \dimline[label style={above=0.5ex,}, line style = {line width=0.7},extension start length=+0.3,extension end length=+0.3] {(\meshsize*\unit - 1*\unit,1)}{(\meshsize*\unit,1)}{\huge $\delta^{k}_\mathrm{mesh}=1$};
\else
    \dimline[label style={above=0.5ex,}, line style = {line width=0.7},extension start length=+0.3,extension end length=+0.3] {(\pollcornerX*\unit,0.5)}{(\pollcenterX*\unit,0.5)}{\huge $\delta^{k}_\mathrm{poll}=\addfraction{\deltameshn}{\deltameshd}$};
    
    \dimline[label style={above=0.5ex,}, line style = {line width=0.7},extension start length=+0.3,extension end length=+0.3] {(\meshsize*\unit - 1*\unit,0.5)}{(\meshsize*\unit,0.5)}{\huge $\delta^{k}_\mathrm{mesh}=\addfraction{\deltaframen}{\deltaframed}$};
\fi
    \node[align=center, minimum width = 100mm, above right] at (0,2) {\huge Poll \textbf{\color{darkgreen}success}};
    \node[circle, minimum size = 1mm, fill=red, label={[fill=white,above right,label distance=0.2cm]\huge $x_k$}] (center) at (\pollcenterX*\unit,\pollcenterY*\unit) {};
    \node[star, star points=8, minimum size = 6mm, fill=blue, label={[fill=white,below,label distance=-0.5cm]\huge $p_7$}] (p1) at (\pAcoordX*\unit,\pAcoordY*\unit) {};
    \node[circle, minimum size = 1mm, fill=red, label={[fill=white,right,label distance=0.5cm]\huge $p_8$}] (p2) at (\pBcoordX*\unit,\pBcoordY*\unit) {};
    \node[circle, minimum size = 1mm, fill=red, label={[fill=white]\huge $p_9$}] (p3) at (\pCcoordX*\unit,\pCcoordY*\unit) {};
\fi
%
\ifnum\frameopt < 8
    \def\unit{0.625}
    \def\meshsize{16}
    \def\pollX{-2}
    \def\pollY{-2}
    \def\pollsize{8}
    %
    \def\pAX{3}
    \def\pAY{-\pollsize/2}
    \def\pBX{-\pollsize/2}
    \def\pBY{0}
    \def\pCX{2}
    \def\pCY{\pollsize/2}
\fi
%
\ifnum\frameopt = 7
    % https://tex.stackexchange.com/a/183633
\newcommand{\Grid}[6]{%
    \def\startX{#1}
    \def\startY{#2}
    \def\maxX{#3}
    \def\maxY{#4}
    \def\size{#5}
    \def\borderthick{#6}
    %
    \draw[line width=\borderthick] (\size*0+\startX,\size*0+\startY) rectangle (\size*\maxX+\startX,\size*-\maxY+\startY);
    \foreach \x in {0,1,...,\maxX}{
        \draw (\size*\x+\startX,\size*0+\startY) -- (\size*\x+\startX,\size*-\maxY+\startY);
    }
    %
    \foreach \y in {0,1,...,\maxY}{
        \draw (\size*0+\startX,\size*-\y+\startY) -- (\size*\maxX+\startX,\size*-\y+\startY);
    }
}
%
\pgfmathsetmacro{\centerX}{\meshsize/2}
\pgfmathsetmacro{\centerY}{-\meshsize/2}
%
\pgfmathsetmacro{\pollcenterX}{\centerX+\pollX}
\pgfmathsetmacro{\pollcenterY}{\centerY+\pollY}
\pgfmathsetmacro{\pollcornerX}{\pollcenterX - \pollsize/2}
\pgfmathsetmacro{\pollcornerY}{\pollcenterY + \pollsize/2}
%
\pgfmathsetmacro{\pAcoordX}{\pollcenterX + \pAX}
\pgfmathsetmacro{\pAcoordY}{\pollcenterY - \pAY}
\pgfmathsetmacro{\pBcoordX}{\pollcenterX + \pBX}
\pgfmathsetmacro{\pBcoordY}{\pollcenterY - \pBY}
\pgfmathsetmacro{\pCcoordX}{\pollcenterX + \pCX}
\pgfmathsetmacro{\pCcoordY}{\pollcenterY - \pCY}
%
\pgfmathsetmacro{\deltameshn}{\pollcenterX-\pollcornerX}
\pgfmathsetmacro{\deltameshd}{\centerX}
\pgfmathsetmacro{\deltamesh}{(\pollcenterX-\pollcornerX)/\centerX}
\pgfmathsetmacro{\deltaframen}{\deltameshn * \deltameshn}
\pgfmathsetmacro{\deltaframed}{\deltameshd * \deltameshd}
\pgfmathsetmacro{\deltaframe}{\deltamesh * \deltamesh}
%
\Grid{0}{0}{\meshsize}{\meshsize}{\unit}{1pt}
\Grid{\pollcornerX*\unit}{\pollcornerY*\unit}{\pollsize}{\pollsize}{\unit}{3pt}
%
\ifdim \deltamesh pt = 1 pt
    \dimline[label style={above=0.5ex,}, line style = {line width=0.7},extension start length=+0.3,extension end length=+0.3] {(\pollcornerX*\unit,1)}{(\pollcenterX*\unit,1)}{\huge $\delta^{k}_\mathrm{poll}=1$};
    
    \dimline[label style={above=0.5ex,}, line style = {line width=0.7},extension start length=+0.3,extension end length=+0.3] {(\meshsize*\unit - 1*\unit,1)}{(\meshsize*\unit,1)}{\huge $\delta^{k}_\mathrm{mesh}=1$};
\else
    \dimline[label style={above=0.5ex,}, line style = {line width=0.7},extension start length=+0.3,extension end length=+0.3] {(\pollcornerX*\unit,0.5)}{(\pollcenterX*\unit,0.5)}{\huge $\delta^{k}_\mathrm{poll}=\addfraction{\deltameshn}{\deltameshd}$};
    
    \dimline[label style={above=0.5ex,}, line style = {line width=0.7},extension start length=+0.3,extension end length=+0.3] {(\meshsize*\unit - 1*\unit,0.5)}{(\meshsize*\unit,0.5)}{\huge $\delta^{k}_\mathrm{mesh}=\addfraction{\deltaframen}{\deltaframed}$};
\fi
    \node[circle, minimum size = 1mm, fill=red, label={[fill=white,above right,label distance=0.2cm]\huge $x_k$}] (center) at (\pollcenterX*\unit,\pollcenterY*\unit) {};
    \node[circle, minimum size = 1mm, fill=red, label={[fill=white,below,label distance=-0.5cm]\huge $p_{10}$}] (p1) at (\pAcoordX*\unit,\pAcoordY*\unit) {};
    \node[circle, minimum size = 1mm, fill=red, label={[fill=white,right,label distance=0.5cm]\huge $p_{11}$}] (p2) at (\pBcoordX*\unit,\pBcoordY*\unit) {};
    \node[circle, minimum size = 1mm, fill=red, label={[fill=white]\huge $p_{12}$}] (p3) at (\pCcoordX*\unit,\pCcoordY*\unit) {};
\fi
%
\ifnum\frameopt < 9
    \def\unit{0.3125}
    \def\meshsize{32}
    \def\pollX{-4}
    \def\pollY{-4}
    \def\pollsize{8}
    %
    \def\pAX{-2}
    \def\pAY{-\pollsize/2}
    \def\pBX{\pollsize/2}
    \def\pBY{-\pollsize/2}
    \def\pCX{\pollsize/2}
    \def\pCY{\pollsize/2}
\fi
%
\ifnum\frameopt = 8
    % https://tex.stackexchange.com/a/183633
\newcommand{\Grid}[6]{%
    \def\startX{#1}
    \def\startY{#2}
    \def\maxX{#3}
    \def\maxY{#4}
    \def\size{#5}
    \def\borderthick{#6}
    %
    \draw[line width=\borderthick] (\size*0+\startX,\size*0+\startY) rectangle (\size*\maxX+\startX,\size*-\maxY+\startY);
    \foreach \x in {0,1,...,\maxX}{
        \draw (\size*\x+\startX,\size*0+\startY) -- (\size*\x+\startX,\size*-\maxY+\startY);
    }
    %
    \foreach \y in {0,1,...,\maxY}{
        \draw (\size*0+\startX,\size*-\y+\startY) -- (\size*\maxX+\startX,\size*-\y+\startY);
    }
}
%
\pgfmathsetmacro{\centerX}{\meshsize/2}
\pgfmathsetmacro{\centerY}{-\meshsize/2}
%
\pgfmathsetmacro{\pollcenterX}{\centerX+\pollX}
\pgfmathsetmacro{\pollcenterY}{\centerY+\pollY}
\pgfmathsetmacro{\pollcornerX}{\pollcenterX - \pollsize/2}
\pgfmathsetmacro{\pollcornerY}{\pollcenterY + \pollsize/2}
%
\pgfmathsetmacro{\pAcoordX}{\pollcenterX + \pAX}
\pgfmathsetmacro{\pAcoordY}{\pollcenterY - \pAY}
\pgfmathsetmacro{\pBcoordX}{\pollcenterX + \pBX}
\pgfmathsetmacro{\pBcoordY}{\pollcenterY - \pBY}
\pgfmathsetmacro{\pCcoordX}{\pollcenterX + \pCX}
\pgfmathsetmacro{\pCcoordY}{\pollcenterY - \pCY}
%
\pgfmathsetmacro{\deltameshn}{\pollcenterX-\pollcornerX}
\pgfmathsetmacro{\deltameshd}{\centerX}
\pgfmathsetmacro{\deltamesh}{(\pollcenterX-\pollcornerX)/\centerX}
\pgfmathsetmacro{\deltaframen}{\deltameshn * \deltameshn}
\pgfmathsetmacro{\deltaframed}{\deltameshd * \deltameshd}
\pgfmathsetmacro{\deltaframe}{\deltamesh * \deltamesh}
%
\Grid{0}{0}{\meshsize}{\meshsize}{\unit}{1pt}
\Grid{\pollcornerX*\unit}{\pollcornerY*\unit}{\pollsize}{\pollsize}{\unit}{3pt}
%
\ifdim \deltamesh pt = 1 pt
    \dimline[label style={above=0.5ex,}, line style = {line width=0.7},extension start length=+0.3,extension end length=+0.3] {(\pollcornerX*\unit,1)}{(\pollcenterX*\unit,1)}{\huge $\delta^{k}_\mathrm{poll}=1$};
    
    \dimline[label style={above=0.5ex,}, line style = {line width=0.7},extension start length=+0.3,extension end length=+0.3] {(\meshsize*\unit - 1*\unit,1)}{(\meshsize*\unit,1)}{\huge $\delta^{k}_\mathrm{mesh}=1$};
\else
    \dimline[label style={above=0.5ex,}, line style = {line width=0.7},extension start length=+0.3,extension end length=+0.3] {(\pollcornerX*\unit,0.5)}{(\pollcenterX*\unit,0.5)}{\huge $\delta^{k}_\mathrm{poll}=\addfraction{\deltameshn}{\deltameshd}$};
    
    \dimline[label style={above=0.5ex,}, line style = {line width=0.7},extension start length=+0.3,extension end length=+0.3] {(\meshsize*\unit - 1*\unit,0.5)}{(\meshsize*\unit,0.5)}{\huge $\delta^{k}_\mathrm{mesh}=\addfraction{\deltaframen}{\deltaframed}$};
\fi
    \node[align=center, minimum width = 100mm, above right] at (0,2) {\huge Poll \textbf{\color{red}failure}};
    \node[circle, minimum size = 1mm, fill=red, label={[fill=white,above right,label distance=0.2cm]\huge $x_k$}] (center) at (\pollcenterX*\unit,\pollcenterY*\unit) {};
    \node[circle, minimum size = 1mm, fill=red, label={[fill=white,below,label distance=-0.5cm]\huge $p_{13}$}] (p1) at (\pAcoordX*\unit,\pAcoordY*\unit) {};
    \node[circle, minimum size = 1mm, fill=red, label={[fill=white,right,label distance=0.5cm]\huge $p_{14}$}] (p2) at (\pBcoordX*\unit,\pBcoordY*\unit) {};
    \node[circle, minimum size = 1mm, fill=red, label={[fill=white]\huge $p_{15}$}] (p3) at (\pCcoordX*\unit,\pCcoordY*\unit) {};
\fi
%
\ifnum\frameopt < 10
    \def\unit{0.15625}
    \def\meshsize{64}
    \def\pollX{-8}
    \def\pollY{-8}
    \def\pollsize{8}
    %
    \def\pAX{1}
    \def\pAY{\pollsize/2}
    \def\pBX{-\pollsize/2}
    \def\pBY{0}
    \def\pCX{\pollsize/2}
    \def\pCY{-\pollsize/2}
\fi
%
\ifnum\frameopt = 9
    % https://tex.stackexchange.com/a/183633
\newcommand{\Grid}[6]{%
    \def\startX{#1}
    \def\startY{#2}
    \def\maxX{#3}
    \def\maxY{#4}
    \def\size{#5}
    \def\borderthick{#6}
    %
    \draw[line width=\borderthick] (\size*0+\startX,\size*0+\startY) rectangle (\size*\maxX+\startX,\size*-\maxY+\startY);
    \foreach \x in {0,1,...,\maxX}{
        \draw (\size*\x+\startX,\size*0+\startY) -- (\size*\x+\startX,\size*-\maxY+\startY);
    }
    %
    \foreach \y in {0,1,...,\maxY}{
        \draw (\size*0+\startX,\size*-\y+\startY) -- (\size*\maxX+\startX,\size*-\y+\startY);
    }
}
%
\pgfmathsetmacro{\centerX}{\meshsize/2}
\pgfmathsetmacro{\centerY}{-\meshsize/2}
%
\pgfmathsetmacro{\pollcenterX}{\centerX+\pollX}
\pgfmathsetmacro{\pollcenterY}{\centerY+\pollY}
\pgfmathsetmacro{\pollcornerX}{\pollcenterX - \pollsize/2}
\pgfmathsetmacro{\pollcornerY}{\pollcenterY + \pollsize/2}
%
\pgfmathsetmacro{\pAcoordX}{\pollcenterX + \pAX}
\pgfmathsetmacro{\pAcoordY}{\pollcenterY - \pAY}
\pgfmathsetmacro{\pBcoordX}{\pollcenterX + \pBX}
\pgfmathsetmacro{\pBcoordY}{\pollcenterY - \pBY}
\pgfmathsetmacro{\pCcoordX}{\pollcenterX + \pCX}
\pgfmathsetmacro{\pCcoordY}{\pollcenterY - \pCY}
%
\pgfmathsetmacro{\deltameshn}{\pollcenterX-\pollcornerX}
\pgfmathsetmacro{\deltameshd}{\centerX}
\pgfmathsetmacro{\deltamesh}{(\pollcenterX-\pollcornerX)/\centerX}
\pgfmathsetmacro{\deltaframen}{\deltameshn * \deltameshn}
\pgfmathsetmacro{\deltaframed}{\deltameshd * \deltameshd}
\pgfmathsetmacro{\deltaframe}{\deltamesh * \deltamesh}
%
\Grid{0}{0}{\meshsize}{\meshsize}{\unit}{1pt}
\Grid{\pollcornerX*\unit}{\pollcornerY*\unit}{\pollsize}{\pollsize}{\unit}{3pt}
%
\ifdim \deltamesh pt = 1 pt
    \dimline[label style={above=0.5ex,}, line style = {line width=0.7},extension start length=+0.3,extension end length=+0.3] {(\pollcornerX*\unit,1)}{(\pollcenterX*\unit,1)}{\huge $\delta^{k}_\mathrm{poll}=1$};
    
    \dimline[label style={above=0.5ex,}, line style = {line width=0.7},extension start length=+0.3,extension end length=+0.3] {(\meshsize*\unit - 1*\unit,1)}{(\meshsize*\unit,1)}{\huge $\delta^{k}_\mathrm{mesh}=1$};
\else
    \dimline[label style={above=0.5ex,}, line style = {line width=0.7},extension start length=+0.3,extension end length=+0.3] {(\pollcornerX*\unit,0.5)}{(\pollcenterX*\unit,0.5)}{\huge $\delta^{k}_\mathrm{poll}=\addfraction{\deltameshn}{\deltameshd}$};
    
    \dimline[label style={above=0.5ex,}, line style = {line width=0.7},extension start length=+0.3,extension end length=+0.3] {(\meshsize*\unit - 1*\unit,0.5)}{(\meshsize*\unit,0.5)}{\huge $\delta^{k}_\mathrm{mesh}=\addfraction{\deltaframen}{\deltaframed}$};
\fi
    \node[align=center, minimum width = 100mm, above right] at (0,2) {\huge Poll \textbf{\color{red}failure}};
    \node[circle, minimum size = 1mm, fill=red] (center) at (\pollcenterX*\unit,\pollcenterY*\unit) {};
    \node[circle, minimum size = 1mm, fill=red] (p1) at (\pAcoordX*\unit,\pAcoordY*\unit) {};
    \node[circle, minimum size = 1mm, fill=red] (p2) at (\pBcoordX*\unit,\pBcoordY*\unit) {};
    \node[circle, minimum size = 1mm, fill=red] (p3) at (\pCcoordX*\unit,\pCcoordY*\unit) {};
\fi
%
\ifnum\frameopt > -1
    \draw[-, color=red, line width=2pt, opacity=1.0] (center) -- (p1);
    \draw[-, color=red, line width=2pt, opacity=1.0] (center) -- (p2);
    \draw[-, color=red, line width=2pt, opacity=1.0] (center) -- (p3);
\fi}%
					\only<3->{\ifnum\frameopts < 2
    \def\unit{1.25}
    \def\meshsize{8}
    \def\pollX{0}
    \def\pollY{0}
    \def\pollsize{4}
    %
    \def\pAX{-\pollsize/2}
    \def\pAY{-1}
    \def\pBX{\pollsize/2}
    \def\pBY{\pollsize/2}
    \def\pCX{\pollsize/2}
    \def\pCY{-1}
\fi
%
\ifnum\frameopts=0
    % https://tex.stackexchange.com/a/183633
\newcommand{\Grid}[6]{%
    \def\startX{#1}
    \def\startY{#2}
    \def\maxX{#3}
    \def\maxY{#4}
    \def\size{#5}
    \def\borderthick{#6}
    %
    \draw[line width=\borderthick] (\size*0+\startX,\size*0+\startY) rectangle (\size*\maxX+\startX,\size*-\maxY+\startY);
    \foreach \x in {0,1,...,\maxX}{
        \draw (\size*\x+\startX,\size*0+\startY) -- (\size*\x+\startX,\size*-\maxY+\startY);
    }
    %
    \foreach \y in {0,1,...,\maxY}{
        \draw (\size*0+\startX,\size*-\y+\startY) -- (\size*\maxX+\startX,\size*-\y+\startY);
    }
}
%
\pgfmathsetmacro{\centerX}{\meshsize/2}
\pgfmathsetmacro{\centerY}{-\meshsize/2}
%
\pgfmathsetmacro{\pollcenterX}{\centerX+\pollX}
\pgfmathsetmacro{\pollcenterY}{\centerY+\pollY}
\pgfmathsetmacro{\pollcornerX}{\pollcenterX - \pollsize/2}
\pgfmathsetmacro{\pollcornerY}{\pollcenterY + \pollsize/2}
%
\pgfmathsetmacro{\pAcoordX}{\pollcenterX + \pAX}
\pgfmathsetmacro{\pAcoordY}{\pollcenterY - \pAY}
\pgfmathsetmacro{\pBcoordX}{\pollcenterX + \pBX}
\pgfmathsetmacro{\pBcoordY}{\pollcenterY - \pBY}
\pgfmathsetmacro{\pCcoordX}{\pollcenterX + \pCX}
\pgfmathsetmacro{\pCcoordY}{\pollcenterY - \pCY}
%
\pgfmathsetmacro{\deltameshn}{\pollcenterX-\pollcornerX}
\pgfmathsetmacro{\deltameshd}{\centerX}
\pgfmathsetmacro{\deltamesh}{(\pollcenterX-\pollcornerX)/\centerX}
\pgfmathsetmacro{\deltaframen}{\deltameshn * \deltameshn}
\pgfmathsetmacro{\deltaframed}{\deltameshd * \deltameshd}
\pgfmathsetmacro{\deltaframe}{\deltamesh * \deltamesh}
%
\Grid{0}{0}{\meshsize}{\meshsize}{\unit}{1pt}
\Grid{\pollcornerX*\unit}{\pollcornerY*\unit}{\pollsize}{\pollsize}{\unit}{3pt}
%
\ifdim \deltamesh pt = 1 pt
    \dimline[label style={above=0.5ex,}, line style = {line width=0.7},extension start length=+0.3,extension end length=+0.3] {(\pollcornerX*\unit,1)}{(\pollcenterX*\unit,1)}{\huge $\delta^{k}_\mathrm{poll}=1$};
    
    \dimline[label style={above=0.5ex,}, line style = {line width=0.7},extension start length=+0.3,extension end length=+0.3] {(\meshsize*\unit - 1*\unit,1)}{(\meshsize*\unit,1)}{\huge $\delta^{k}_\mathrm{mesh}=1$};
\else
    \dimline[label style={above=0.5ex,}, line style = {line width=0.7},extension start length=+0.3,extension end length=+0.3] {(\pollcornerX*\unit,0.5)}{(\pollcenterX*\unit,0.5)}{\huge $\delta^{k}_\mathrm{poll}=\addfraction{\deltameshn}{\deltameshd}$};
    
    \dimline[label style={above=0.5ex,}, line style = {line width=0.7},extension start length=+0.3,extension end length=+0.3] {(\meshsize*\unit - 1*\unit,0.5)}{(\meshsize*\unit,0.5)}{\huge $\delta^{k}_\mathrm{mesh}=\addfraction{\deltaframen}{\deltaframed}$};
\fi
    \node[circle, minimum size = 1mm, fill=red, label={[fill=white,above right,label distance=0.2cm]\huge $x_k$}] (center) at (\pollcenterX*\unit,\pollcenterY*\unit) {};
    \node[circle, minimum size = 1mm, fill=red, label={[fill=white,below,label distance=-0.5cm]\huge $p_4$}] (p1) at (\pAcoordX*\unit,\pAcoordY*\unit) {};
    \node[circle, minimum size = 1mm, fill=red, label={[fill=white,right,label distance=0.5cm]\huge $p_5$}] (p2) at (\pBcoordX*\unit,\pBcoordY*\unit) {};
    \node[circle, minimum size = 1mm, fill=red, label={[fill=white]\huge $p_6$}] (p3) at (\pCcoordX*\unit,\pCcoordY*\unit) {};
\fi
%
\ifnum\frameopts=1
    % https://tex.stackexchange.com/a/183633
\newcommand{\Grid}[6]{%
    \def\startX{#1}
    \def\startY{#2}
    \def\maxX{#3}
    \def\maxY{#4}
    \def\size{#5}
    \def\borderthick{#6}
    %
    \draw[line width=\borderthick] (\size*0+\startX,\size*0+\startY) rectangle (\size*\maxX+\startX,\size*-\maxY+\startY);
    \foreach \x in {0,1,...,\maxX}{
        \draw (\size*\x+\startX,\size*0+\startY) -- (\size*\x+\startX,\size*-\maxY+\startY);
    }
    %
    \foreach \y in {0,1,...,\maxY}{
        \draw (\size*0+\startX,\size*-\y+\startY) -- (\size*\maxX+\startX,\size*-\y+\startY);
    }
}
%
\pgfmathsetmacro{\centerX}{\meshsize/2}
\pgfmathsetmacro{\centerY}{-\meshsize/2}
%
\pgfmathsetmacro{\pollcenterX}{\centerX+\pollX}
\pgfmathsetmacro{\pollcenterY}{\centerY+\pollY}
\pgfmathsetmacro{\pollcornerX}{\pollcenterX - \pollsize/2}
\pgfmathsetmacro{\pollcornerY}{\pollcenterY + \pollsize/2}
%
\pgfmathsetmacro{\pAcoordX}{\pollcenterX + \pAX}
\pgfmathsetmacro{\pAcoordY}{\pollcenterY - \pAY}
\pgfmathsetmacro{\pBcoordX}{\pollcenterX + \pBX}
\pgfmathsetmacro{\pBcoordY}{\pollcenterY - \pBY}
\pgfmathsetmacro{\pCcoordX}{\pollcenterX + \pCX}
\pgfmathsetmacro{\pCcoordY}{\pollcenterY - \pCY}
%
\pgfmathsetmacro{\deltameshn}{\pollcenterX-\pollcornerX}
\pgfmathsetmacro{\deltameshd}{\centerX}
\pgfmathsetmacro{\deltamesh}{(\pollcenterX-\pollcornerX)/\centerX}
\pgfmathsetmacro{\deltaframen}{\deltameshn * \deltameshn}
\pgfmathsetmacro{\deltaframed}{\deltameshd * \deltameshd}
\pgfmathsetmacro{\deltaframe}{\deltamesh * \deltamesh}
%
\Grid{0}{0}{\meshsize}{\meshsize}{\unit}{1pt}
\Grid{\pollcornerX*\unit}{\pollcornerY*\unit}{\pollsize}{\pollsize}{\unit}{3pt}
%
\ifdim \deltamesh pt = 1 pt
    \dimline[label style={above=0.5ex,}, line style = {line width=0.7},extension start length=+0.3,extension end length=+0.3] {(\pollcornerX*\unit,1)}{(\pollcenterX*\unit,1)}{\huge $\delta^{k}_\mathrm{poll}=1$};
    
    \dimline[label style={above=0.5ex,}, line style = {line width=0.7},extension start length=+0.3,extension end length=+0.3] {(\meshsize*\unit - 1*\unit,1)}{(\meshsize*\unit,1)}{\huge $\delta^{k}_\mathrm{mesh}=1$};
\else
    \dimline[label style={above=0.5ex,}, line style = {line width=0.7},extension start length=+0.3,extension end length=+0.3] {(\pollcornerX*\unit,0.5)}{(\pollcenterX*\unit,0.5)}{\huge $\delta^{k}_\mathrm{poll}=\addfraction{\deltameshn}{\deltameshd}$};
    
    \dimline[label style={above=0.5ex,}, line style = {line width=0.7},extension start length=+0.3,extension end length=+0.3] {(\meshsize*\unit - 1*\unit,0.5)}{(\meshsize*\unit,0.5)}{\huge $\delta^{k}_\mathrm{mesh}=\addfraction{\deltaframen}{\deltaframed}$};
\fi
    \node[align=center, minimum width = 100mm, above right] at (0,2) {\huge Poll \textbf{certain} \textbf{\color{red}failure}};
    \node[star, star points=8, minimum size = 6mm, fill=blue, label={[fill=white,above right,label distance=0.2cm]\huge $x_k$}] (center) at (\pollcenterX*\unit,\pollcenterY*\unit) {};
    \node[circle, minimum size = 1mm, fill=red, label={[fill=white,below,label distance=-0.5cm]\huge $p_4$}] (p1) at (\pAcoordX*\unit,\pAcoordY*\unit) {};
    \node[circle, minimum size = 1mm, fill=red, label={[fill=white,right,label distance=0.5cm]\huge $p_5$}] (p2) at (\pBcoordX*\unit,\pBcoordY*\unit) {};
    \node[circle, minimum size = 1mm, fill=red, label={[fill=white]\huge $p_6$}] (p3) at (\pCcoordX*\unit,\pCcoordY*\unit) {};
    \node[circle, minimum size = 1.41*\unit*20mm, fill=blue, opacity=0.5] (uncertainty) at (\pollcenterX*\unit,\pollcenterY*\unit) {};
\fi
%
\ifnum\frameopts < 3
    \def\unit{0.3125}
    \def\meshsize{32}
    \def\pollX{0}
    \def\pollY{0}
    \def\pollsize{8}
    %
    \def\pAX{-\pollsize/2}
    \def\pAY{-3}
    \def\pBX{3}
    \def\pBY{\pollsize/2}
    \def\pCX{\pollsize/2}
    \def\pCY{-\pollsize/2}
\fi
%
\ifnum\frameopts=2
    % https://tex.stackexchange.com/a/183633
\newcommand{\Grid}[6]{%
    \def\startX{#1}
    \def\startY{#2}
    \def\maxX{#3}
    \def\maxY{#4}
    \def\size{#5}
    \def\borderthick{#6}
    %
    \draw[line width=\borderthick] (\size*0+\startX,\size*0+\startY) rectangle (\size*\maxX+\startX,\size*-\maxY+\startY);
    \foreach \x in {0,1,...,\maxX}{
        \draw (\size*\x+\startX,\size*0+\startY) -- (\size*\x+\startX,\size*-\maxY+\startY);
    }
    %
    \foreach \y in {0,1,...,\maxY}{
        \draw (\size*0+\startX,\size*-\y+\startY) -- (\size*\maxX+\startX,\size*-\y+\startY);
    }
}
%
\pgfmathsetmacro{\centerX}{\meshsize/2}
\pgfmathsetmacro{\centerY}{-\meshsize/2}
%
\pgfmathsetmacro{\pollcenterX}{\centerX+\pollX}
\pgfmathsetmacro{\pollcenterY}{\centerY+\pollY}
\pgfmathsetmacro{\pollcornerX}{\pollcenterX - \pollsize/2}
\pgfmathsetmacro{\pollcornerY}{\pollcenterY + \pollsize/2}
%
\pgfmathsetmacro{\pAcoordX}{\pollcenterX + \pAX}
\pgfmathsetmacro{\pAcoordY}{\pollcenterY - \pAY}
\pgfmathsetmacro{\pBcoordX}{\pollcenterX + \pBX}
\pgfmathsetmacro{\pBcoordY}{\pollcenterY - \pBY}
\pgfmathsetmacro{\pCcoordX}{\pollcenterX + \pCX}
\pgfmathsetmacro{\pCcoordY}{\pollcenterY - \pCY}
%
\pgfmathsetmacro{\deltameshn}{\pollcenterX-\pollcornerX}
\pgfmathsetmacro{\deltameshd}{\centerX}
\pgfmathsetmacro{\deltamesh}{(\pollcenterX-\pollcornerX)/\centerX}
\pgfmathsetmacro{\deltaframen}{\deltameshn * \deltameshn}
\pgfmathsetmacro{\deltaframed}{\deltameshd * \deltameshd}
\pgfmathsetmacro{\deltaframe}{\deltamesh * \deltamesh}
%
\Grid{0}{0}{\meshsize}{\meshsize}{\unit}{1pt}
\Grid{\pollcornerX*\unit}{\pollcornerY*\unit}{\pollsize}{\pollsize}{\unit}{3pt}
%
\ifdim \deltamesh pt = 1 pt
    \dimline[label style={above=0.5ex,}, line style = {line width=0.7},extension start length=+0.3,extension end length=+0.3] {(\pollcornerX*\unit,1)}{(\pollcenterX*\unit,1)}{\huge $\delta^{k}_\mathrm{poll}=1$};
    
    \dimline[label style={above=0.5ex,}, line style = {line width=0.7},extension start length=+0.3,extension end length=+0.3] {(\meshsize*\unit - 1*\unit,1)}{(\meshsize*\unit,1)}{\huge $\delta^{k}_\mathrm{mesh}=1$};
\else
    \dimline[label style={above=0.5ex,}, line style = {line width=0.7},extension start length=+0.3,extension end length=+0.3] {(\pollcornerX*\unit,0.5)}{(\pollcenterX*\unit,0.5)}{\huge $\delta^{k}_\mathrm{poll}=\addfraction{\deltameshn}{\deltameshd}$};
    
    \dimline[label style={above=0.5ex,}, line style = {line width=0.7},extension start length=+0.3,extension end length=+0.3] {(\meshsize*\unit - 1*\unit,0.5)}{(\meshsize*\unit,0.5)}{\huge $\delta^{k}_\mathrm{mesh}=\addfraction{\deltaframen}{\deltaframed}$};
\fi
    \node[align=center, minimum width = 100mm, above right] at (0,2) {\huge Poll \textbf{certain} \textbf{\color{red}failure}};
    \node[star, star points=8, minimum size = 6mm, fill=blue, label={[fill=white,above right,label distance=0.2cm]\huge $x_k$}] (center) at (\pollcenterX*\unit,\pollcenterY*\unit) {};
    \node[circle, minimum size = 1mm, fill=red, label={[fill=white,below,label distance=-0.5cm]\huge $p_7$}] (p1) at (\pAcoordX*\unit,\pAcoordY*\unit) {};
    \node[circle, minimum size = 1mm, fill=red, label={[fill=white,right,label distance=0.5cm]\huge $p_8$}] (p2) at (\pBcoordX*\unit,\pBcoordY*\unit) {};
    \node[circle, minimum size = 1mm, fill=red, label={[fill=white]\huge $p_9$}] (p3) at (\pCcoordX*\unit,\pCcoordY*\unit) {};
    \node[circle, minimum size = 1.41*\unit*20mm, fill=blue, opacity=0.5] (uncertainty) at (\pollcenterX*\unit,\pollcenterY*\unit) {};
\fi
\ifnum\frameopts < 5
    \def\unit{1.25}
    \def\meshsize{8}
    \def\pollX{0}
    \def\pollY{0}
    \def\pollsize{4}
    %
    \def\pAX{-\pollsize/2}
    \def\pAY{-1}
    \def\pBX{\pollsize/2}
    \def\pBY{\pollsize/2}
    \def\pCX{\pollsize/2}
    \def\pCY{-1}
\fi
%
\ifnum\frameopts=3
    % https://tex.stackexchange.com/a/183633
\newcommand{\Grid}[6]{%
    \def\startX{#1}
    \def\startY{#2}
    \def\maxX{#3}
    \def\maxY{#4}
    \def\size{#5}
    \def\borderthick{#6}
    %
    \draw[line width=\borderthick] (\size*0+\startX,\size*0+\startY) rectangle (\size*\maxX+\startX,\size*-\maxY+\startY);
    \foreach \x in {0,1,...,\maxX}{
        \draw (\size*\x+\startX,\size*0+\startY) -- (\size*\x+\startX,\size*-\maxY+\startY);
    }
    %
    \foreach \y in {0,1,...,\maxY}{
        \draw (\size*0+\startX,\size*-\y+\startY) -- (\size*\maxX+\startX,\size*-\y+\startY);
    }
}
%
\pgfmathsetmacro{\centerX}{\meshsize/2}
\pgfmathsetmacro{\centerY}{-\meshsize/2}
%
\pgfmathsetmacro{\pollcenterX}{\centerX+\pollX}
\pgfmathsetmacro{\pollcenterY}{\centerY+\pollY}
\pgfmathsetmacro{\pollcornerX}{\pollcenterX - \pollsize/2}
\pgfmathsetmacro{\pollcornerY}{\pollcenterY + \pollsize/2}
%
\pgfmathsetmacro{\pAcoordX}{\pollcenterX + \pAX}
\pgfmathsetmacro{\pAcoordY}{\pollcenterY - \pAY}
\pgfmathsetmacro{\pBcoordX}{\pollcenterX + \pBX}
\pgfmathsetmacro{\pBcoordY}{\pollcenterY - \pBY}
\pgfmathsetmacro{\pCcoordX}{\pollcenterX + \pCX}
\pgfmathsetmacro{\pCcoordY}{\pollcenterY - \pCY}
%
\pgfmathsetmacro{\deltameshn}{\pollcenterX-\pollcornerX}
\pgfmathsetmacro{\deltameshd}{\centerX}
\pgfmathsetmacro{\deltamesh}{(\pollcenterX-\pollcornerX)/\centerX}
\pgfmathsetmacro{\deltaframen}{\deltameshn * \deltameshn}
\pgfmathsetmacro{\deltaframed}{\deltameshd * \deltameshd}
\pgfmathsetmacro{\deltaframe}{\deltamesh * \deltamesh}
%
\Grid{0}{0}{\meshsize}{\meshsize}{\unit}{1pt}
\Grid{\pollcornerX*\unit}{\pollcornerY*\unit}{\pollsize}{\pollsize}{\unit}{3pt}
%
\ifdim \deltamesh pt = 1 pt
    \dimline[label style={above=0.5ex,}, line style = {line width=0.7},extension start length=+0.3,extension end length=+0.3] {(\pollcornerX*\unit,1)}{(\pollcenterX*\unit,1)}{\huge $\delta^{k}_\mathrm{poll}=1$};
    
    \dimline[label style={above=0.5ex,}, line style = {line width=0.7},extension start length=+0.3,extension end length=+0.3] {(\meshsize*\unit - 1*\unit,1)}{(\meshsize*\unit,1)}{\huge $\delta^{k}_\mathrm{mesh}=1$};
\else
    \dimline[label style={above=0.5ex,}, line style = {line width=0.7},extension start length=+0.3,extension end length=+0.3] {(\pollcornerX*\unit,0.5)}{(\pollcenterX*\unit,0.5)}{\huge $\delta^{k}_\mathrm{poll}=\addfraction{\deltameshn}{\deltameshd}$};
    
    \dimline[label style={above=0.5ex,}, line style = {line width=0.7},extension start length=+0.3,extension end length=+0.3] {(\meshsize*\unit - 1*\unit,0.5)}{(\meshsize*\unit,0.5)}{\huge $\delta^{k}_\mathrm{mesh}=\addfraction{\deltaframen}{\deltaframed}$};
\fi
    \node[align=center, minimum width = 100mm, above right] at (0,2) {\huge Poll \textbf{certain} \textbf{\color{red}failure}};
    \node[star, star points=8, minimum size = 6mm, fill=blue, label={[fill=white,above right,label distance=0.2cm]\huge $x_k$}] (center) at (\pollcenterX*\unit,\pollcenterY*\unit) {};
    \node[circle, minimum size = 1mm, fill=red, label={[fill=white,below,label distance=-0.5cm]\huge $p_4$}] (p1) at (\pAcoordX*\unit,\pAcoordY*\unit) {};
    \node[circle, minimum size = 1mm, fill=red, label={[fill=white,right,label distance=0.5cm]\huge $p_5$}] (p2) at (\pBcoordX*\unit,\pBcoordY*\unit) {};
    \node[circle, minimum size = 1mm, fill=red, label={[fill=white]\huge $p_6$}] (p3) at (\pCcoordX*\unit,\pCcoordY*\unit) {};
    \node[circle, minimum size = 1.41*\unit*20mm, fill=blue, opacity=0.5] (uncertainty) at (\pollcenterX*\unit,\pollcenterY*\unit) {};
\fi
%
\ifnum\frameopts=4
    % https://tex.stackexchange.com/a/183633
\newcommand{\Grid}[6]{%
    \def\startX{#1}
    \def\startY{#2}
    \def\maxX{#3}
    \def\maxY{#4}
    \def\size{#5}
    \def\borderthick{#6}
    %
    \draw[line width=\borderthick] (\size*0+\startX,\size*0+\startY) rectangle (\size*\maxX+\startX,\size*-\maxY+\startY);
    \foreach \x in {0,1,...,\maxX}{
        \draw (\size*\x+\startX,\size*0+\startY) -- (\size*\x+\startX,\size*-\maxY+\startY);
    }
    %
    \foreach \y in {0,1,...,\maxY}{
        \draw (\size*0+\startX,\size*-\y+\startY) -- (\size*\maxX+\startX,\size*-\y+\startY);
    }
}
%
\pgfmathsetmacro{\centerX}{\meshsize/2}
\pgfmathsetmacro{\centerY}{-\meshsize/2}
%
\pgfmathsetmacro{\pollcenterX}{\centerX+\pollX}
\pgfmathsetmacro{\pollcenterY}{\centerY+\pollY}
\pgfmathsetmacro{\pollcornerX}{\pollcenterX - \pollsize/2}
\pgfmathsetmacro{\pollcornerY}{\pollcenterY + \pollsize/2}
%
\pgfmathsetmacro{\pAcoordX}{\pollcenterX + \pAX}
\pgfmathsetmacro{\pAcoordY}{\pollcenterY - \pAY}
\pgfmathsetmacro{\pBcoordX}{\pollcenterX + \pBX}
\pgfmathsetmacro{\pBcoordY}{\pollcenterY - \pBY}
\pgfmathsetmacro{\pCcoordX}{\pollcenterX + \pCX}
\pgfmathsetmacro{\pCcoordY}{\pollcenterY - \pCY}
%
\pgfmathsetmacro{\deltameshn}{\pollcenterX-\pollcornerX}
\pgfmathsetmacro{\deltameshd}{\centerX}
\pgfmathsetmacro{\deltamesh}{(\pollcenterX-\pollcornerX)/\centerX}
\pgfmathsetmacro{\deltaframen}{\deltameshn * \deltameshn}
\pgfmathsetmacro{\deltaframed}{\deltameshd * \deltameshd}
\pgfmathsetmacro{\deltaframe}{\deltamesh * \deltamesh}
%
\Grid{0}{0}{\meshsize}{\meshsize}{\unit}{1pt}
\Grid{\pollcornerX*\unit}{\pollcornerY*\unit}{\pollsize}{\pollsize}{\unit}{3pt}
%
\ifdim \deltamesh pt = 1 pt
    \dimline[label style={above=0.5ex,}, line style = {line width=0.7},extension start length=+0.3,extension end length=+0.3] {(\pollcornerX*\unit,1)}{(\pollcenterX*\unit,1)}{\huge $\delta^{k}_\mathrm{poll}=1$};
    
    \dimline[label style={above=0.5ex,}, line style = {line width=0.7},extension start length=+0.3,extension end length=+0.3] {(\meshsize*\unit - 1*\unit,1)}{(\meshsize*\unit,1)}{\huge $\delta^{k}_\mathrm{mesh}=1$};
\else
    \dimline[label style={above=0.5ex,}, line style = {line width=0.7},extension start length=+0.3,extension end length=+0.3] {(\pollcornerX*\unit,0.5)}{(\pollcenterX*\unit,0.5)}{\huge $\delta^{k}_\mathrm{poll}=\addfraction{\deltameshn}{\deltameshd}$};
    
    \dimline[label style={above=0.5ex,}, line style = {line width=0.7},extension start length=+0.3,extension end length=+0.3] {(\meshsize*\unit - 1*\unit,0.5)}{(\meshsize*\unit,0.5)}{\huge $\delta^{k}_\mathrm{mesh}=\addfraction{\deltaframen}{\deltaframed}$};
\fi
    \node[align=center, minimum width = 100mm, above right] at (0,2) {\huge Poll \textbf{uncertain} \textbf{\color{red}failure}};
    \node[star, star points=8, minimum size = 6mm, fill=blue, label={[fill=white,above right,label distance=0.2cm]\huge $x_k$}] (center) at (\pollcenterX*\unit,\pollcenterY*\unit) {};
    \node[circle, minimum size = 1mm, fill=red, label={[fill=white,below,label distance=-0.5cm]\huge $p_4$}] (p1) at (\pAcoordX*\unit,\pAcoordY*\unit) {};
    \node[circle, minimum size = 1mm, fill=red, label={[fill=white,right,label distance=0.5cm]\huge $p_5$}] (p2) at (\pBcoordX*\unit,\pBcoordY*\unit) {};
    \node[circle, minimum size = 1mm, fill=red, label={[fill=white]\huge $p_6$}] (p3) at (\pCcoordX*\unit,\pCcoordY*\unit) {};
    \node[circle, minimum size = 2*1.21*\unit*20mm, fill=blue, opacity=0.5] (uncertainty) at (\pollcenterX*\unit,\pollcenterY*\unit) {};
\fi
%
\ifnum\frameopts < 6
    \def\unit{0.625}
    \def\meshsize{16}
    \def\pollX{0}
    \def\pollY{0}
    \def\pollsize{6}
    %
    \def\pAX{-\pollsize/2}
    \def\pAY{-1}
    \def\pBX{\pollsize/2}
    \def\pBY{\pollsize/2}
    \def\pCX{\pollsize/2}
    \def\pCY{-1}
\fi
%
\ifnum\frameopts=5
    % https://tex.stackexchange.com/a/183633
\newcommand{\Grid}[6]{%
    \def\startX{#1}
    \def\startY{#2}
    \def\maxX{#3}
    \def\maxY{#4}
    \def\size{#5}
    \def\borderthick{#6}
    %
    \draw[line width=\borderthick] (\size*0+\startX,\size*0+\startY) rectangle (\size*\maxX+\startX,\size*-\maxY+\startY);
    \foreach \x in {0,1,...,\maxX}{
        \draw (\size*\x+\startX,\size*0+\startY) -- (\size*\x+\startX,\size*-\maxY+\startY);
    }
    %
    \foreach \y in {0,1,...,\maxY}{
        \draw (\size*0+\startX,\size*-\y+\startY) -- (\size*\maxX+\startX,\size*-\y+\startY);
    }
}
%
\pgfmathsetmacro{\centerX}{\meshsize/2}
\pgfmathsetmacro{\centerY}{-\meshsize/2}
%
\pgfmathsetmacro{\pollcenterX}{\centerX+\pollX}
\pgfmathsetmacro{\pollcenterY}{\centerY+\pollY}
\pgfmathsetmacro{\pollcornerX}{\pollcenterX - \pollsize/2}
\pgfmathsetmacro{\pollcornerY}{\pollcenterY + \pollsize/2}
%
\pgfmathsetmacro{\pAcoordX}{\pollcenterX + \pAX}
\pgfmathsetmacro{\pAcoordY}{\pollcenterY - \pAY}
\pgfmathsetmacro{\pBcoordX}{\pollcenterX + \pBX}
\pgfmathsetmacro{\pBcoordY}{\pollcenterY - \pBY}
\pgfmathsetmacro{\pCcoordX}{\pollcenterX + \pCX}
\pgfmathsetmacro{\pCcoordY}{\pollcenterY - \pCY}
%
\pgfmathsetmacro{\deltameshn}{\pollcenterX-\pollcornerX}
\pgfmathsetmacro{\deltameshd}{\centerX}
\pgfmathsetmacro{\deltamesh}{(\pollcenterX-\pollcornerX)/\centerX}
\pgfmathsetmacro{\deltaframen}{\deltameshn * \deltameshn}
\pgfmathsetmacro{\deltaframed}{\deltameshd * \deltameshd}
\pgfmathsetmacro{\deltaframe}{\deltamesh * \deltamesh}
%
\Grid{0}{0}{\meshsize}{\meshsize}{\unit}{1pt}
\Grid{\pollcornerX*\unit}{\pollcornerY*\unit}{\pollsize}{\pollsize}{\unit}{3pt}
%
\ifdim \deltamesh pt = 1 pt
    \dimline[label style={above=0.5ex,}, line style = {line width=0.7},extension start length=+0.3,extension end length=+0.3] {(\pollcornerX*\unit,1)}{(\pollcenterX*\unit,1)}{\huge $\delta^{k}_\mathrm{poll}=1$};
    
    \dimline[label style={above=0.5ex,}, line style = {line width=0.7},extension start length=+0.3,extension end length=+0.3] {(\meshsize*\unit - 1*\unit,1)}{(\meshsize*\unit,1)}{\huge $\delta^{k}_\mathrm{mesh}=1$};
\else
    \dimline[label style={above=0.5ex,}, line style = {line width=0.7},extension start length=+0.3,extension end length=+0.3] {(\pollcornerX*\unit,0.5)}{(\pollcenterX*\unit,0.5)}{\huge $\delta^{k}_\mathrm{poll}=\addfraction{\deltameshn}{\deltameshd}$};
    
    \dimline[label style={above=0.5ex,}, line style = {line width=0.7},extension start length=+0.3,extension end length=+0.3] {(\meshsize*\unit - 1*\unit,0.5)}{(\meshsize*\unit,0.5)}{\huge $\delta^{k}_\mathrm{mesh}=\addfraction{\deltaframen}{\deltaframed}$};
\fi
    \node[align=center, minimum width = 100mm, above right] at (0,2) {\huge Poll \textbf{uncertain} \textbf{\color{red}failure}};
    \node[star, star points=8, minimum size = 6mm, fill=blue, label={[fill=white,above right,label distance=0.2cm]\huge $x_k$}] (center) at (\pollcenterX*\unit,\pollcenterY*\unit) {};
    \node[circle, minimum size = 1mm, fill=red, label={[fill=white,below,label distance=-0.5cm]\huge $p_7$}] (p1) at (\pAcoordX*\unit,\pAcoordY*\unit) {};
    \node[circle, minimum size = 1mm, fill=red, label={[fill=white,right,label distance=0.5cm]\huge $p_8$}] (p2) at (\pBcoordX*\unit,\pBcoordY*\unit) {};
    \node[circle, minimum size = 1mm, fill=red, label={[fill=white]\huge $p_9$}] (p3) at (\pCcoordX*\unit,\pCcoordY*\unit) {};
    \node[circle, minimum size = 2.5*1.41*\unit*20mm, fill=blue, opacity=0.5] (uncertainty) at (\pollcenterX*\unit,\pollcenterY*\unit) {};
\fi
%
\ifnum\frameopts < 7
    \def\unit{0.3125}
    \def\meshsize{32}
    \def\pollX{0}
    \def\pollY{0}
    \def\pollsize{8}
    %
    \def\pAX{-\pollsize/2}
    \def\pAY{-1}
    \def\pBX{\pollsize/2}
    \def\pBY{\pollsize/2}
    \def\pCX{\pollsize/2}
    \def\pCY{-1}
\fi
%
\ifnum\frameopts=6
    % https://tex.stackexchange.com/a/183633
\newcommand{\Grid}[6]{%
    \def\startX{#1}
    \def\startY{#2}
    \def\maxX{#3}
    \def\maxY{#4}
    \def\size{#5}
    \def\borderthick{#6}
    %
    \draw[line width=\borderthick] (\size*0+\startX,\size*0+\startY) rectangle (\size*\maxX+\startX,\size*-\maxY+\startY);
    \foreach \x in {0,1,...,\maxX}{
        \draw (\size*\x+\startX,\size*0+\startY) -- (\size*\x+\startX,\size*-\maxY+\startY);
    }
    %
    \foreach \y in {0,1,...,\maxY}{
        \draw (\size*0+\startX,\size*-\y+\startY) -- (\size*\maxX+\startX,\size*-\y+\startY);
    }
}
%
\pgfmathsetmacro{\centerX}{\meshsize/2}
\pgfmathsetmacro{\centerY}{-\meshsize/2}
%
\pgfmathsetmacro{\pollcenterX}{\centerX+\pollX}
\pgfmathsetmacro{\pollcenterY}{\centerY+\pollY}
\pgfmathsetmacro{\pollcornerX}{\pollcenterX - \pollsize/2}
\pgfmathsetmacro{\pollcornerY}{\pollcenterY + \pollsize/2}
%
\pgfmathsetmacro{\pAcoordX}{\pollcenterX + \pAX}
\pgfmathsetmacro{\pAcoordY}{\pollcenterY - \pAY}
\pgfmathsetmacro{\pBcoordX}{\pollcenterX + \pBX}
\pgfmathsetmacro{\pBcoordY}{\pollcenterY - \pBY}
\pgfmathsetmacro{\pCcoordX}{\pollcenterX + \pCX}
\pgfmathsetmacro{\pCcoordY}{\pollcenterY - \pCY}
%
\pgfmathsetmacro{\deltameshn}{\pollcenterX-\pollcornerX}
\pgfmathsetmacro{\deltameshd}{\centerX}
\pgfmathsetmacro{\deltamesh}{(\pollcenterX-\pollcornerX)/\centerX}
\pgfmathsetmacro{\deltaframen}{\deltameshn * \deltameshn}
\pgfmathsetmacro{\deltaframed}{\deltameshd * \deltameshd}
\pgfmathsetmacro{\deltaframe}{\deltamesh * \deltamesh}
%
\Grid{0}{0}{\meshsize}{\meshsize}{\unit}{1pt}
\Grid{\pollcornerX*\unit}{\pollcornerY*\unit}{\pollsize}{\pollsize}{\unit}{3pt}
%
\ifdim \deltamesh pt = 1 pt
    \dimline[label style={above=0.5ex,}, line style = {line width=0.7},extension start length=+0.3,extension end length=+0.3] {(\pollcornerX*\unit,1)}{(\pollcenterX*\unit,1)}{\huge $\delta^{k}_\mathrm{poll}=1$};
    
    \dimline[label style={above=0.5ex,}, line style = {line width=0.7},extension start length=+0.3,extension end length=+0.3] {(\meshsize*\unit - 1*\unit,1)}{(\meshsize*\unit,1)}{\huge $\delta^{k}_\mathrm{mesh}=1$};
\else
    \dimline[label style={above=0.5ex,}, line style = {line width=0.7},extension start length=+0.3,extension end length=+0.3] {(\pollcornerX*\unit,0.5)}{(\pollcenterX*\unit,0.5)}{\huge $\delta^{k}_\mathrm{poll}=\addfraction{\deltameshn}{\deltameshd}$};
    
    \dimline[label style={above=0.5ex,}, line style = {line width=0.7},extension start length=+0.3,extension end length=+0.3] {(\meshsize*\unit - 1*\unit,0.5)}{(\meshsize*\unit,0.5)}{\huge $\delta^{k}_\mathrm{mesh}=\addfraction{\deltaframen}{\deltaframed}$};
\fi
    \node[align=center, minimum width = 100mm, above right] at (0,2) {\huge Poll \textbf{certain} \textbf{\color{red}failure}};
    \node[star, star points=8, minimum size = 6mm, fill=blue, label={[fill=white,above right,label distance=0.2cm]\huge $x_k$}] (center) at (\pollcenterX*\unit,\pollcenterY*\unit) {};
    \node[circle, minimum size = 1mm, fill=red] (p1) at (\pAcoordX*\unit,\pAcoordY*\unit) {};
    \node[circle, minimum size = 1mm, fill=red] (p2) at (\pBcoordX*\unit,\pBcoordY*\unit) {};
    \node[circle, minimum size = 1mm, fill=red] (p3) at (\pCcoordX*\unit,\pCcoordY*\unit) {};
    \node[circle, minimum size = 2.5*1.41*\unit*20mm, fill=blue, opacity=0.5] (uncertainty) at (\pollcenterX*\unit,\pollcenterY*\unit) {};
\fi
%
\draw[-, color=red, line width=2pt, opacity=1.0] (center) -- (p1);
\draw[-, color=red, line width=2pt, opacity=1.0] (center) -- (p2);
\draw[-, color=red, line width=2pt, opacity=1.0] (center) -- (p3);
%
\node[align=center, minimum width = 100mm, above right] (ulabel) at (0,-12) {\huge $\left\{x:f^k_s - f^k_0 \in \mathcal{I}\right\}$};
%
\path[stealth-,blue!70,line width = 3pt] (uncertainty.south) edge [out=290, in=70] (ulabel.north);}%
				\end{tikzpicture}%
			}%
		\end{column}
	\end{columns}
	\vspace{-3em}
\end{frame}
\addtocounter{footnote}{-2}
%------------------------------------------------
\subsection{ABM Conclusion}
%------------------------------------------------
\begin{frame}[t,label=abm_7]
	\frametitle{Conclusion}
	\tikzstyle{background grid}=[draw, black!50,step=.5cm]
	%
	\uncover<1->{Formulated and solved public health policy-making problems}%
	\only<6->{{\color{white}\ifshowcitations\footpartcite{Ferguson2020}\fi}}\\
	%
	\begin{columns}[t] % The "c" option specifies centered vertical alignment while the "t" option is used for top vertical alignment
		\begin{column}{.42\textwidth} % Left column and width
			\begin{itemize}
				\item<1-> Identified a public health policy that favored
				\begin{itemize}
					\item<2-> High testing capacity $n_T$
					\item<3-> Large number of essential workers $n_E$
					\item<4-> Modest social distancing $S_D$
				\end{itemize}
				\item<5-> StoMADS outperformed GAs and NOMAD on active constraints
				\item<6-> Applicable to other large-scale agent-based models\footnotemark[1]
			\end{itemize}
		\end{column}
		%
		\begin{column}{.5\textwidth} % Left column and width
			\tikzstyle{background grid}=[draw, black!50,step=.5cm]
			\begin{tikzpicture}[remember picture, overlay]%[show background grid]
				\node [inner sep=0pt,above right, opacity=1.0]  at (0.0\textwidth,-0.75\textheight) (mobility) 
				{
					\only<-5>{\includegraphics[width=0.93\textwidth]{agent_based_models/trajecctory_results/M_compare_opt_2.pdf}}%
				};
				\only<-5>{
					\node[inner sep=0pt,align=flush center,above=\belowcaptionskip of mobility,text width=\linewidth]
					{\vspace{-0em}{
						\large best known solution
					}};
				}%
				% show origin
				% \fill (0,0) circle (2pt);
			\end{tikzpicture}%
			\begin{tikzpicture}[remember picture, overlay] %show background grid, 
				\node [inner sep=0pt,above right, opacity=1.0]  at (0.1\textwidth,-0.66\textheight) (abm) 
					{
						\only<6>{
							\begin{animateinline}[autoplay,width=0.9\textwidth]{8}
								\ifshowanimations
									\multiframe{48}{i=1+15}{%
										\includegraphics{agent_based_models/canada_ABM_anim/map_\i.png}
									}
								\else
									\multiframe{1}{i=721+0}{%
										\includegraphics{agent_based_models/canada_ABM/map_\i.png}
									}
								\fi
							\end{animateinline}%
						}%
						\only<7->{
							\includegraphics[width=0.9\textwidth]{agent_based_models/canada_ABM/map_721.png}
						}
					};
			\end{tikzpicture}%
		\end{column}
	
	\end{columns}
	\vspace{-3em}
\end{frame}
\addtocounter{footnote}{-1}
%------------------------------------------------
\subsection{COVID-19 introduction}
%------------------------------------------------
\begin{frame}[t]
	\frametitle{Background: COVID-19 forecasting targets}
	\tikzstyle{background grid}=[draw, black!50,step=.5cm]
	%
	Forecasting novel epidemics is a multidisciplinary field involving multiple \emph{targets} \ifshowcitations\footpartcite{Wu2021}\fi\\
	\emph{Inputs} for forecasting \emphasis{epidemic size:}\\
    %
    \begin{itemize}
        \item<2-> Growth rate indicators
        \item<3-> \only<5->{\emphasis}{Historical incidence rate data}
        \item<4-> \only<5->{\emphasis}{Serologic assays}\\
        \centering\includegraphics[width=0.23\textwidth]{machine_learning/targets/pcr.png}
    \end{itemize}
    %
	\tikzstyle{background grid}=[draw, black!50,step=.5cm]
	\begin{tikzpicture}[remember picture, overlay] %show background grid, 
		% Put the graphic inside a node. This makes it easy to place the
		% graphic and to draw on top of it. 
		% The above right option is used to place the lower left corner
		% of the image at the (0,0) coordinate. 
		\node [inner sep=0pt,above left, opacity=1.0]  at (0.99\textwidth,-0.0\textheight) (seq2seq) 
			{
				\only<1>{\includegraphics[width=0.45\textwidth]{machine_learning/targets/nowcasting_targets_1.pdf}}%
				\only<2>{\includegraphics[width=0.45\textwidth]{machine_learning/targets/nowcasting_targets_2.pdf}}%
                \only<3->{\includegraphics[width=0.45\textwidth]{machine_learning/targets/nowcasting_targets_3.pdf}}%
			};
		\only<1->{
			\node[inner sep=0pt,align=flush center,below=\belowcaptionskip of seq2seq,text width=\linewidth]
				{\vspace{-1em}{Nowcasting targets}};
		}%
		% show origin
		% \fill (0,0) circle (2pt);
	\end{tikzpicture}%
	%
	\vspace{-3em}
\end{frame}
\addtocounter{footnote}{-1}
%------------------------------------------------
\subsection{Hyperparameter tuning future directions}
%------------------------------------------------
\begin{frame}[t]
	\frametitle{Conclusion and future directions}
	\tikzstyle{background grid}=[draw, black!50,step=.5cm]
	%
	\only<1-5>{Model discovery and development facilitated by hyperparameter optimization}%
	\only<6->{StoMADS can be improved to solve a wide variety of HPO problems}%
	\only<5>{{\color{white}\ifshowcitations\footpartcite{Hay2021}\fi}}%
	\only<7->{{\color{white}\ifshowcitations\footpartcite{Lakhmiri2019}\textsuperscript{,}\footpartcite{Bergstra2011}\fi}}\\
	%
	\tikzstyle{background grid}=[draw, black!50,step=.5cm]
	\begin{tikzpicture}[remember picture, overlay] %show background grid, 
		% Put the graphic  inside a node. This makes it easy to place the
		% graphic and to draw on top of it. 
		% The above right option is used to place the lower left corner
		% of the image at the (0,0) coordinate. 
		\node [inner sep=0pt,above left, opacity=1.0]  at (1.01\textwidth,-0.73\textheight) (prediction) 
			{
				\only<3->{\includegraphics[width=0.5\textwidth]{machine_learning/predictions/model_predictions_unseen_SVR_mean_Ct_daily_cases_G12.pdf}}%
			};
		% show origin
		% \fill (0,0) circle (2pt);
	\end{tikzpicture}%
	
	\begin{columns}[c] % The "c" option specifies centered vertical alignment while the "t" option is used for top vertical alignment
		\begin{column}{.5\textwidth} % Left column and width
			\vspace{-0.0em}
			% Optimization problem
			\begin{itemize}\itemsep0em
			\only<-5>{
				\item<2-5> Cycle Threshold (Ct) is a useful feature for incidence projection
				\item<3-5> Model that generalizes well on unseen data
    			\item<4-5> Simpler models perform well when historical data is limited
				\item<5> Works well on other datasets\footnotemark[1]
			}
			\item<6-> Can be used to meet deployment targets
				\begin{exampleblock}{Objective and constraints}
					\vspace{-1.0em}%
					\begin{equation*}
						\begin{aligned}
							& \underset{\mathbf{x}}{\text{min}}
							& & f(\mathbf{x}) = \mathbb{E}_{\Theta}\left[{f}_{\Theta}(\mathbf{x}) = \mathrm{error}_\mathrm{CV}\right]\\
							& \text{subject to}
							& & {c}(\mathbf{x}) = \mathrm{\small inference~time} - \mathrm{\small threshold} \le 0\\
							& \text{where}
							& & \Theta\mathrm{:realizations}
						\end{aligned}
					\end{equation*}
				\end{exampleblock}
				\item<7-> Should be benchmarked against HyperNOMAD\footnotemark[1], Bayesian optimization\footnotemark[2]
				\item<8-> Mixed variable version is needed
			\end{itemize}
			\only<-5>{\vspace{-10em}}
		\end{column}
		%
		\begin{column}{.5\textwidth} % Left column and width
		\end{column}
	
	\end{columns}
	%
	\vspace{-3em}
\end{frame}
\addtocounter{footnote}{-2}
%------------------------------------------------