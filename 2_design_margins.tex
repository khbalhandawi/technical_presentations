%%%%%%%%%%%%%%%%%%%%%%%%%%%%%%%%%%%%%%%%%%%%%%%%%%%%%%%
%%        ANSWERING THE RESEARCH QUESTIONS           %%
%%%%%%%%%%%%%%%%%%%%%%%%%%%%%%%%%%%%%%%%%%%%%%%%%%%%%%%
\section{Change absorption through excess and buffer}
%------------------------------------------------
\begin{frame}[c,noframenumbering]
	\centering
	% \setlength\fboxsep{0pt}
	\begin{titleblock}{}
		~\\%
		{\centering\LARGE Using design margins to absorb change in requirements \ifshowcitations\footnotemark[1]\fi\\}%
		~\\%
	\end{titleblock}
	{\color{white}\ifshowcitations\footpartcite{Alhandawi2021}[1]\fi}
\end{frame}
\addtocounter{footnote}{-1}
%------------------------------------------------
\begin{frame}[t,label=ap2_2]
	\frametitle{Quantifying capability, buffer, and excess}
	\vspace{-1em}
	\uncover<2->{We focus on one 2D projection}%
	~\\
	\begin{columns}[t] % The "c" option specifies centered vertical alignment while the "t" option is used for top vertical alignment
		\begin{column}{.5\textwidth} % Left column and width
			\small
			\vspace{-1em}
			\begin{itemize}
				\item<1-> \textit{Failure criterion}: Factor of safety $n_{\textrm{safety}}(\mathbf{T})$ \uncover<2->{represented using isocontours}
				\item<3-> \textit{Capability}: Loads that \emphasis{can} be satisfied
				\item<4-> \textit{Requirements}: Loads that \emphasis{must} be satisfied
				\item<5-> \textit{Buffer}: Portion of requirement \emphasis{satisfied}
				\item<6-> \textit{Danger}: Portion of requirement \emphasis{not satisfied}
				\item<7-> \textit{Excess}: \emphasis{Unused} capability not needed for satisfying requirement
				\item<8-> \textit{Reliability}: $\smash{\mathbb{P}(\mathbf{T} \in C)}$ estimated using Monte-Carlo integration
			\end{itemize}
		\end{column}
	
		\begin{column}{.45\textwidth} % Left column and width
			\begin{figure}
				\only<1>{\includegraphics[height=5.0cm]{design_margins/DM_geometry/TRS_isometric_failure_domain.pdf}}%
				\only<2>{\includegraphics[height=5.6cm]{design_margins/DM_parameter_space/109_req_1_thermal_out_nominal_RS_2D_pi_0.pdf}}%
				\only<3>{\includegraphics[height=5.6cm]{design_margins/DM_parameter_space/109_req_1_thermal_out_nominal_RS_2D_pi_1.pdf}}%
				\only<4>{\includegraphics[height=5.6cm]{design_margins/DM_parameter_space/109_req_1_thermal_out_nominal_RS_2D_pi_2.pdf}}%
				\only<5>{\includegraphics[height=5.6cm]{design_margins/DM_parameter_space/109_req_1_thermal_out_nominal_RS_2D_pi_3.pdf}}%
				\only<6>{\includegraphics[height=5.6cm]{design_margins/DM_parameter_space/109_req_1_thermal_out_nominal_RS_2D_pi_4.pdf}}%
				\only<7>{\includegraphics[height=5.6cm]{design_margins/DM_parameter_space/109_req_1_thermal_out_nominal_RS_2D_pi_5.pdf}}%
				\only<8>{\includegraphics[height=5.6cm]{design_margins/DM_parameter_space/109_req_1_thermal_out_nominal_RS_2D_pi_6.pdf}}%
				\only<1>{
					\vspace{-0.2em}
					\caption{Failure domain}
				}%
			\end{figure}
		\end{column}
		
	\end{columns}
	
\end{frame}
%------------------------------------------------
\begin{frame}[t,label=ap2_3]
	\frametitle{Considering redesign when buffer is insufficient}
	\vspace{-1em}
	\only<1>{Going back to our baseline design we can calculate these metrics as follows:}%
	\only<2>{Consider depositing more material}%
	\only<3>{Reliability increases}%
	\only<4->{However, {\color{red} requirements can change as well!}}%
	~\\
	\begin{columns}[t] % The "c" option specifies centered vertical alignment while the "t" option is used for top vertical alignment
		\begin{column}{.45\textwidth} % Left column and width
			\only<1>{$c=1$,$\mathbf{D}=\left[1,2,4\right]$}%
			\only<2->{$c=1$,$\mathbf{D}=\left[1,2,4,{\color{red}0}\right]$}%
			\begin{figure}
				\only<1-2>{\includegraphics[height=5.0cm]{design_margins/DM_geometry/TRS_isometric_7.pdf}}%
				\only<3->{\includegraphics[height=5.0cm]{design_margins/DM_geometry/TRS_isometric_8.pdf}}%
				\vspace{-0.2em}
				\caption{Turbine Rear Structure (TRS)}
			\end{figure}
			\vspace{-2em}
			\begin{table}[h!]
				\centering
				% \renewcommand{\arraystretch}{1.0}% Wider
				\footnotesize\addtolength{\tabcolsep}{-2pt}
				\begin{tabular}{ccc}
					\bf Reliability & \bf Excess & \bf Weight\\
					\only<1-2>{0.3089 & 0.529 & 13.9\\}%
					\only<3>{0.759({\color{darkgreen}$\uparrow 145\%$}) & 0.856({\color{red}$\uparrow 62\%$}) & 18.5($\color{red}\uparrow 33\%$)\\}
					\only<4>{0.943({\color{darkgreen}$\uparrow 24\%$}) & 0.754({\color{darkgreen}$\downarrow 12\%$}) & 18.5\\}
					\only<5>{1.00({\color{darkgreen}$\uparrow 6\%$}) & 0.871({\color{red}$\uparrow 15\%$}) & 18.5\\}
				\end{tabular}
			\end{table}
		\end{column}
		
		\begin{column}{.45\textwidth} % Left column and width
			\centering
			~\\%
			~\\%
			\vspace{-1em}
			\begin{figure}
				\only<1-2>{\includegraphics[height=5.6cm]{design_margins/DM_parameter_space/109_req_1_thermal_out_nominal_RS_2D_pi_6.pdf}}%
				\only<3>{\includegraphics[height=5.6cm]{design_margins/DM_parameter_space/110_req_1_thermal_out_nominal_RS_2D_pi_6.pdf}}%
				\only<4>{\includegraphics[height=5.6cm]{design_margins/DM_parameter_space/110_reqs_2D/110_req_15_thermal_out_RS_2D_pi_6.pdf}}%
				\only<5>{\includegraphics[height=5.6cm]{design_margins/DM_parameter_space/110_reqs_2D/110_req_11_thermal_out_RS_2D_pi_6.pdf}}%
				\label{fig:2DPspace}
			\end{figure}
		\end{column}
		
	\end{columns}
	
\end{frame}

%%%%%%%%%%%%%%%%%%%%%%%%%%%%%%%%%%%%%%%%%%%%%%%%%%%%%%%
%%         APPLYING THE RESEARCH QUESTIONS           %%
%%%%%%%%%%%%%%%%%%%%%%%%%%%%%%%%%%%%%%%%%%%%%%%%%%%%%%%

%------------------------------------------------
\subsection{Optimizing when and how much change is needed}
%------------------------------------------------
\begin{frame}[t]
	\frametitle{Optimizing when and how much change is needed}
	\vspace{-0.5em}
	Formulate optimization problem to minimize total excess while being reliable
	
	\begin{columns}[t] % The "c" option specifies centered vertical alignment while the "t" option is used for top vertical alignment
		
		\begin{column}{.42\textwidth} % Left column and width
			\vspace{-1.5em}
			% Optimization problem
			\begin{alertblock}{Objective and constraints}
				\vspace{-1.5em}
				\begin{equation*}
					\begin{aligned}
						& {\text{minimize}}
						& & \text{excess}(c,\mathbf{D};\text{requirement arc})\\
						& \text{subject to}
						& & reliability(c,\mathbf{D};\text{requirement arc})\\
						& \text{over}
						& & \text{The set of feasible design decisions}\\
						& & & \text{(404 possible combinations)}
					\end{aligned}
				\end{equation*}
			\end{alertblock}
			\vspace{-0.5em}
			\uncover<3->{
				% Variables
				\begin{exampleblock}{Design variables}
					\vspace{-0.5em}
					\begin{equation*}
						\begin{aligned}
							& & & c \in \left\{"wavy","hatched","tubular"\right\}\\
							& & & \mathbf{D} = \left[\text{decision}_1,\text{decision}_2,\cdots\right]\\
							& & & \text{decision} \in \left\{-1,0,1,2,3,4\right\}\\
						\end{aligned}
					\end{equation*}
				\end{exampleblock}
			}
			\vspace{-0.5em}
		\end{column}

		\begin{column}{.5\textwidth} % Right column and width
			\vspace{-1em}
			\tikzstyle{background grid}=[draw, black!50,step=.5cm]
			
			\begin{tikzpicture}%[show background grid]
			% Put the graphic inside a node. This makes it easy to place the
			% graphic and to draw on top of it. 
			% The above right option is used to place the lower left corner
			% of the image at the (0,0) coordinate. 
				\uncover<-5>{
					\node [inner sep=0pt,above right, opacity=0.6]  at (0, 1) (objective)
						{
							\only<2->{\includegraphics[height=5.0cm]{design_margins/DM_parameter_space/110_reqs_2D/110_req_11_thermal_out_RS_2D_pi_6.pdf}}%
						};
					\node [inner sep=0pt,above right, opacity=0.7, fill=white]  at (4.0, -.75) (constraint)
						{
							\only<3->{\includegraphics[height=4.0cm]{design_margins/DM_geometry/TRS_isometric_8.pdf}}%
						};
				}
				% show origin
				% \fill (0,0) circle (2pt);
				% define destination coordinates
			\end{tikzpicture}

		\end{column}

	\end{columns}
\end{frame}