%%%%%%%%%%%%%%%%%%%%%%%%%%%%%%%%%%%%%%%%%%%%%%%%%%%%%%%
%%                   Introduction                    %%
%%%%%%%%%%%%%%%%%%%%%%%%%%%%%%%%%%%%%%%%%%%%%%%%%%%%%%%
%------------------------------------------------
\section{Introduction}
%------------------------------------------------
\begin{frame}[t]
	\frametitle{The challenge}
	
	\begin{itemize}
		\uncover<1->{\item This is a Turbine Rear Structure (TRS) under thermal loading}
		\uncover<2->{\item Over its lifetime, design requirements can change}
	\end{itemize}
	
	\vspace{-2em}
	\begin{columns}[t] % The "c" option specifies centered vertical alignment while the "t" option is used for top vertical alignment
			
		\begin{column}{.4\textwidth} % Left column and width
			\begin{figure}
				\only<0>{
					\includegraphics[height=5cm]{scalability/TRS_overview_pictures/1a_TRS_isometric_1.pdf}
				}%
				\only<1>{
					\includegraphics[height=5cm]{scalability/TRS_overview_pictures/1a_TRS_isometric_2.pdf}
				}%
				\only<2->{
					\includegraphics[height=5cm]{scalability/TRS_overview_pictures/1a_TRS_isometric_3.pdf}
				}%
			\end{figure}
			\vspace{-3em}
		\end{column}
	
		\begin{column}{.5\textwidth} % Left column and width
	
			\begin{figure}
				\centering\includegraphics[height=5cm]{scalability/trent_900.pdf}
				\caption{An aeroengine}
			\end{figure}%
				
		\end{column}
	\end{columns}
\end{frame}
%------------------------------------------------
\begin{frame}[t]
	\frametitle{The opportunity}
	
	\begin{itemize}
		\item Disposal or redesign can be sub-optimal options
		\uncover<2->{\item Additive remanufacturing may be a better option to extend useful lifetime}
	\end{itemize}
	
	\vspace{-2em}
	\begin{columns}[t] % The "c" option specifies centered vertical alignment while the "t" option is used for top vertical alignment
			
		\begin{column}{.4\textwidth} % Left column and width
			\begin{figure}
				\includegraphics[height=5cm]{scalability/TRS_overview_pictures/1a_TRS_isometric_3.pdf}
			\end{figure}
			\vspace{-3em}
		\end{column}
	
		\begin{column}{.5\textwidth} % Left column and width
	
			\begin{figure}
				\only<1>{
					\vspace{2em}
					\centering\includegraphics[height=4cm]{scalability/dispose_guy.jpg}
				}%
				\only<2>{
					\centering
					\begin{animateinline}[autoplay,height=5cm]{5}
						\ifshowanimations
							\multiframe{30}{i=1+6}{%
								\includegraphics{scalability/TEMP_GIF/frame_\i.png}
							}
						\else
							\multiframe{1}{i=180+0}{%
								\includegraphics{scalability/TEMP_GIF/frame_\i.png}
							}
						\fi
					\end{animateinline}
					\caption{AM Remanufacturing of TRS}
				}%
			\end{figure}%
				
		\end{column}
	\end{columns}
\end{frame}
%------------------------------------------------
% https://tex.stackexchange.com/a/136166
\tikzset{
  transparent/.style={opacity=0.4},
  untransparent on/.style={alt={#1{}{transparent}}},
  alt/.code args={<#1>#2#3}{%
    \alt<#1>{\pgfkeysalso{#2}}{\pgfkeysalso{#3}} % \pgfkeysalso doesn't change the path
  },
  invisible/.style={opacity=0.0},
  visible on/.style={alt={#1{}{invisible}}},
  alt/.code args={<#1>#2#3}{%
    \alt<#1>{\pgfkeysalso{#2}}{\pgfkeysalso{#3}} % \pgfkeysalso doesn't change the path
  },
}
\begin{frame}[t]
	\newcommand{\yshift}{0.0cm}
	\frametitle{Change absorption through scalability}
    %
    \tikzstyle{background grid}=[draw, black!50,step=.5cm]
    \begin{tikzpicture}[remember picture, overlay] %show background grid, 
        \node [inner sep=0pt,above right, opacity=1.0]  at (7cm,-5cm+\yshift) (original) 
        {
            \includegraphics[height=3.5cm]{scalability/intro_definitions/TRS_isometric_A.pdf}
        };
        \node [inner sep=0pt,above right, opacity=1.0]  at (7cm,-5cm+\yshift) (loads) 
        {
            \includegraphics[height=3.5cm]{scalability/intro_definitions/arrows_1.pdf}
        };
		\node [inner sep=0pt,above right, opacity=1.0, visible on=<2-4>]  at (7cm,-5cm+\yshift) (loads2) 
        {
            \includegraphics[height=3.5cm]{scalability/intro_definitions/arrows_2.pdf}
        };
        \only<4->{
            \node [inner sep=0pt,above right, untransparent on=<5>]  at (12cm,-3cm+\yshift) (trsA) 
            {
                \includegraphics[height=3.5cm]{scalability/intro_definitions/TRS_isometric_B.pdf}
            };
            \node [inner sep=0pt,above right, untransparent on=<5>]  at (12cm,-7cm+\yshift) (trsB)
            {
                \includegraphics[height=3.5cm]{scalability/intro_definitions/TRS_isometric_C.pdf}
            };%
            \draw[,->,thick,untransparent on=<5->] (original.east) to [in=180,out=0] (trsA.west);
            \draw[,->,thick,untransparent on=<5->] (original.east) to [in=180,out=0] (trsB.west);
        }%
		\node[%
        visible on=<5>,
            draw=green,
            fill=green!10,
            cloud,
            line width=1mm,
            font=\fontfamily{ppl}\fontsize{0.5cm}{0.5cm}\selectfont
        ] at (11cm,-5.25cm+\yshift) {\cmark};
        \node[%
            visible on=<5>,
            draw=red,
            fill=red!10,
            cloud,
            line width=1mm,
            font=\fontfamily{ppl}\fontsize{0.5cm}{0.5cm}\selectfont
        ] at (11cm,-1.25cm+\yshift) {\xmark};
        % show origin
        % \fill (0,0) circle (2pt);
    \end{tikzpicture}%
    %
	\begin{columns}[c]
		\begin{column}{.5\textwidth} % Left column and width
			~\\
			~\\
			~\\
           	When the applied \only<1>{loads change}\only<2->{\emphasis{loads change}}:\\%
			%
			\uncover<3->{A design can be \emphasis{changed}:}
			%
			\begin{itemize}
				\item<4-> \textit{Scalability}: the ability to change to accommodate stricter requirements \ifshowcitations\footnotemark[1]\fi
				\item<5-> Successful \textit{remanufacturing}: enabled by scalability of components \ifshowcitations\footnotemark[2]\fi
			\end{itemize}
			%
		\end{column}
		\begin{column}{.5\textwidth} % Left column and width
		\end{column}
	\end{columns}
	%
	{\color{white}
	\ifshowcitations
		\only<4->{\footpartcite{Ross2008}[1]}
		\only<5->{\footpartcite{Xing2007}[2]}
	\fi
	}
\end{frame}
\addtocounter{footnote}{-2}

%%%%%%%%%%%%%%%%%%%%%%%%%%%%%%%%%%%%%%%%%%%%%%%%%%%%%%%
%%                   Design problem                  %%
%%%%%%%%%%%%%%%%%%%%%%%%%%%%%%%%%%%%%%%%%%%%%%%%%%%%%%%
%------------------------------------------------
\section{Parametric optimization problem formulation}
% \frame{\tableofcontents[
% 	currentsection, 
% 	hideallsubsections
% 	]}
%------------------------------------------------
\begin{frame}[c,noframenumbering]
	\centering
	% \setlength\fboxsep{0pt}
	\begin{titleblock}{}
		~\\%
		{\centering\LARGE Change absorption through scalability and remanufacturing \ifshowcitations\footnotemark[1]\fi\\}%
		~\\%
	\end{titleblock}
	{\color{white}\ifshowcitations\footpartcite{Alhandawi2020}[1]\fi}
\end{frame}
\addtocounter{footnote}{-1}
%------------------------------------------------
\subsection{Obtain set of parametric optimal designs using surrogate}
%------------------------------------------------
\begin{frame}[t]
\frametitle{Parametric design optimization problem formulation}
%
Remanufacturing variables are included in the design optimization problems and changing requirements are modeled by parameters $\mathbf{p}$ \\~\\

\vspace{-0.5em}
\begin{columns}[t] % The "c" option specifies centered vertical alignment while the "t" option is used for top vertical alignment
	\column{.4\textwidth} % Left column and width
		\vspace{-1em}
		% Optimization problem
		\begin{alertblock}{Design optimization problem}
			\vspace{-1.5em}
			\begin{equation*}
				\begin{aligned}
					& \underset{\mathbf{x}}{\text{minimize}}
					& & {f}(\mathbf{x};\mathbf{p}) = -n_{\textrm{safety}}(\mathbf{x};P_{\textrm{load}})\\
					& \text{subject to}
					& & {g_1}(\mathbf{x};\mathbf{p}) = x_3 + x_1 - W_{\textrm{total}} \le 0\\
					&&& {g_2}(\mathbf{x};\mathbf{p}) = T_m - T_{\textrm{deposit}} \le 0\\
				\end{aligned}
			\end{equation*}
		\end{alertblock}
		% Parameters
		\vspace{-1em}
		\only<1->{
		\begin{exampleblock}{Changing requirements}
			\vspace{-1.5em}
			\begin{equation*}
				\begin{aligned}
					& & & W_{\textrm{total}} = \textrm{width of base part}\\
					& & & T_{m} = \textrm{Melting temperature of deposit}\\
					& & & P_{\textrm{load}} = \textrm{Pressure load on outer casing}\\
				\end{aligned}
			\end{equation*}
		\end{exampleblock}
		}%
		
	\column{.6\textwidth} % Right column and width
		
		\begin{figure}
			\includegraphics[height=5cm]{scalability/TRS_overview_pictures/1b_TRS_overview_3.pdf}
			\caption{TRS cross-section}
		\end{figure}
	\end{columns}

\end{frame}

%%%%%%%%%%%%%%%%%%%%%%%%%%%%%%%%%%%%%%%%%%%%%%%%%%%%%%%
%%              Parametric optimization              %%
%%%%%%%%%%%%%%%%%%%%%%%%%%%%%%%%%%%%%%%%%%%%%%%%%%%%%%%
%------------------------------------------------
\section{Scalability Assessment}
% \frame{\tableofcontents[
% 	currentsection, 
% 	hideallsubsections
% 	]}
%------------------------------------------------
\subsection{Conditions for scalability}
%------------------------------------------------
\begin{frame}[t]
\frametitle{Conditions for scalability}
%
\vspace{-0.5em}
\begin{columns}[t] % The "c" option specifies centered vertical alignment while the "t" option is used for top vertical alignment
	\column{.4\textwidth} % Left column and width
		\vspace{-1em}
		% Optimization problem
		\begin{alertblock}{Design optimization problem}
			\vspace{-1.5em}
			\begin{equation*}
				\begin{aligned}
					& \underset{\mathbf{x}}{\text{minimize}}
					& & {f}(\mathbf{x};\mathbf{p}) = -n_{\textrm{safety}}(\mathbf{x};{P_{\textrm{load}}})\\
					& \text{subject to}
					& & {g_1}(\mathbf{x};\mathbf{p}) = x_3 + x_1 - {W_{\textrm{total}}} \le 0\\
					&&& {g_2}(\mathbf{x};\mathbf{p}) = {T_m} - T_{\textrm{deposit}} \le 0\\
				\end{aligned}
			\end{equation*}
		\end{alertblock}
		% Parameters
		\vspace{-1em}
		\begin{exampleblock}{Changing requirements}
			\vspace{-1.5em}
			\begin{equation*}
				\begin{aligned}
					& & & W_{\textrm{total}} = \textrm{width of base part}\\
					& & & T_{m} = \textrm{Melting temperature of deposit}\\
					& & & P_{\textrm{load}} = \textrm{Pressure load on outer casing}\\
				\end{aligned}
			\end{equation*}
		\end{exampleblock}
		
	\column{.6\textwidth} % Right column and width
		Let us consider one variable %({\color{red}$x_3$}) 
  and two parameters 
  %({\color{red}$p_2$} and {\color{red}$p_3$})
  \\
		For an optimal design to be scalable, the product of 
		\begin{itemize}
			\item<2-> the \emphasis{direction of change} 
			\item<5-> the \emphasis{change effect}, and 
			\item<6-> the variable \emphasis{monotonicity} 
		\end{itemize}
 \uncover<6->{must be greater or equal to zero}
		\uncover<13->{~\\Our full example: {\color{red}$4$} variables and {\color{red}$3$} parameters}

		\newcommand{\xshift}{5.6cm}
		\newcommand{\yshift}{-2cm}

		\begin{tikzpicture}[remember picture,overlay]
				
			\only<6>{
				\node (textscalable) at ( [xshift = \xshift, yshift = -0.1cm] current page.center)
				{\textbf{Scalable design!}};
			}%
			\only<12>{
				\node (textscalable) at ( [xshift = \xshift, yshift = -0.1cm] current page.center)
				{\textbf{{\color{red} Not} a Scalable design!}};
			}%

			\only<1>{
				\fixedpicture{\xshift}{\yshift}{3.6cm}{scalability/scalability_example/11_gradient_demo_1.pdf}
			}% empty contours
			\only<2>{
				\fixedpicture{\xshift}{\yshift}{3.6cm}{scalability/scalability_example/11_gradient_demo_2.pdf}
			}% gradient vector only
			\only<3>{
				\fixedpicture{\xshift}{\yshift}{3.6cm}{scalability/scalability_example/11_gradient_demo_3.pdf}
			}% gradient vector + component A
			\only<4>{
				\fixedpicture{\xshift}{\yshift}{3.6cm}{scalability/scalability_example/11_gradient_demo_4.pdf}
			}% gradient vector + component A + B
			\only<5-6>{
				\fixedpicture{\xshift}{\yshift}{3.6cm}{scalability/scalability_example/11_gradient_demo_5.pdf}
			}% gradient vector + component A + B + change quadrant (scalable)
			\only<7>{
				\fixedpicture{\xshift}{\yshift}{3.6cm}{scalability/scalability_example/11_gradient_demo_6.pdf}
			}% empty contours
			\only<8>{
				\fixedpicture{\xshift}{\yshift}{3.6cm}{scalability/scalability_example/11_gradient_demo_7.pdf}
			}% gradient vector only
			\only<9>{
				\fixedpicture{\xshift}{\yshift}{3.6cm}{scalability/scalability_example/11_gradient_demo_8.pdf}
			}% gradient vector + component A
			\only<10>{
				\fixedpicture{\xshift}{\yshift}{3.6cm}{scalability/scalability_example/11_gradient_demo_9.pdf}
			}% gradient vector + component A + B
			\only<11-12>{
				\fixedpicture{\xshift}{\yshift}{3.6cm}{scalability/scalability_example/11_gradient_demo_10.pdf}
			}% gradient vector + component A + B + change quadrant (non scalable)
		\end{tikzpicture}


	\end{columns}
	\only<-12>{~\\~\\}%
	$
		\small
		\only<-12>{
			\uncover<5->{
				\overbrace{
					\begin{bmatrix}
						\color<3->{red}-1 & 0\\ 
						0 & \color<3->{red}-1
					\end{bmatrix}
				}^{\mathbf{N}=\textrm{Change effect}}
				\times
			}%
		}% example matrix
		\only<13->{
			\overbrace{
				\begin{bmatrix}
					\color{red}n_1 &  0 & 0\\ 
					0 & \color{red}n_2 & 0\\ 
					0 &  0 & \color{red}n_3
				\end{bmatrix}
			}^{\mathbf{N}=\textrm{Change effect}}
			\times
		} % generalized matrix
		%---------------------------------
		\only<-12>{
			\uncover<2->{
				\overbrace{
					\begin{bmatrix}
						\uncover<3->{{^{\partial x_3^*}/_{\partial p_2}}} \\ 
						\uncover<4->{{^{\partial x_3^*}/_{\partial p_3}}}
					\end{bmatrix}
				}^{\mathbf{J}=\textrm{Direction of change}}
			}
		}% example matrix
		\only<13->{
			\uncover<2->{
				\overbrace{
					\begin{bmatrix}
						{^{\partial x_1^*}/_{\partial p_1}} & {^{\partial x_2^*}/_{\partial p_1}} & {^{\partial x_3^*}/_{\partial p_1}} & {^{\partial x_4^*}/_{\partial p_1}} \\ 
						{^{\partial x_1^*}/_{\partial p_2}} & {^{\partial x_2^*}/_{\partial p_2}} & {^{\partial x_3^*}/_{\partial p_2}} & {^{\partial x_4^*}/_{\partial p_2}} \\ 
						{^{\partial x_1^*}/_{\partial p_3}} & {^{\partial x_2^*}/_{\partial p_3}} & {^{\partial x_3^*}/_{\partial p_3}} & {^{\partial x_4^*}/_{\partial p_3}}
					\end{bmatrix}
				}^{\mathbf{J}=\textrm{Direction of change}}
			}
		}% generalized matrix
		%---------------------------------
		\only<-12>{
			\uncover<6->{
				\times
				\overbrace{
					\begin{bmatrix}
						\color<4->{red}1
					\end{bmatrix} 
				}^{\mathbf{M}=\textrm{Monotonicity}}
				\ge \mathbf{0}
			}%
		}% example matrix
		\only<13->{
			\times
			\overbrace{
				\begin{bmatrix}
					\color{red}m_1 & 0 & 0 & 0 \\ 
					0 & \color{red}m_2 & 0 & 0 \\ 
					0 & 0 & \color{red}m_3 & 0 \\ 
					0 & 0 & 0 & \color{red}m_4
				\end{bmatrix} 
			}^{\mathbf{M}=\textrm{Monotonicity}}
			\ge \mathbf{0}
		}% generalized matrix
	$
\end{frame}

%%%%%%%%%%%%%%%%%%%%%%%%%%%%%%%%%%%%%%%%%%%%%%%%%%%%%%%
%%               Scalability assessment              %%
%%%%%%%%%%%%%%%%%%%%%%%%%%%%%%%%%%%%%%%%%%%%%%%%%%%%%%%
%------------------------------------------------
\section{Practical scalability results}
% \frame{\tableofcontents[
% 	currentsection, 
% 	hideallsubsections
% 	]}
%------------------------------------------------
\subsection{Response surface of parameter space to obtain scalable optimal designs}
%------------------------------------------------
\begin{frame}[t]
\frametitle{Response surface of parametric optimal solutions and scalable designs}
\begin{itemize}
	\item Computing the components of $\mathbf{J}(\mathbf{p})$ can be prohibitively expensive 
 %in engineering design
	\item<2-> Response surface $\hat{\mathbf{x}^*}(\mathbf{p})$ can be obtained from parametric optimal designs $\mathbf{X}^*= \left\{\mathbf{x}^*(\mathbf{p}_1)~\mathbf{x}^*(\mathbf{p}_2)~\cdots~\mathbf{x}^*(\mathbf{p}_k)\right\}$ \\
	(we use kernel smoothing)
 %is used for the response surface
	\item<3-> Jacobian $\mathbf{J}(\mathbf{p})$ of $\hat{\mathbf{x}^*}(\mathbf{p})$ estimated from the derivatives of the kernel basis functions
	\item<4-> Scalable optimal designs given by $\mathbf{N}\mathbf{J}(\mathbf{p})^{\mathrm T}\mathbf{M} \ge \mathbf{0}$. %This is the \emphasis{set of designs} we are looking for.
\end{itemize}

\begin{figure}
	\only<2-3>{\includegraphics[width=0.9\textwidth]{scalability/parameter_space/8_PS_f_1.pdf}}%
	\only<4->{\includegraphics[width=0.9\textwidth]{scalability/parameter_space/8_PS_f_3.pdf}}%
	%\caption{\only<2->{2D projections of the parameter space  \only<5>{with {\color{red}sescalable regions} overlayed}}
\end{figure}

\end{frame}

%NEW
%------------------------------------------------
\subsection{Scalable optimal designs in the design space}
%------------------------------------------------
\begin{frame}[t]
\frametitle{Map scalable optimal design set to design space}
% \only<1>{
% 	Scalable optimal designs mapped back to design space using response surface
% }%
% \only<2->{This is the result we are looking for! \only<4>{\color{red}A manageable set of scalable solutions}}%
% \only<1>{
% \begin{tikzpicture}[remember picture,overlay]
% 	\node (SBDresult) at ([xshift=0.5\textwidth, yshift =-0.05\textwidth]current page.west) %or: (current page.center)
% 		{\includegraphics[width=0.9\textwidth]{scalability/SBD_results/6_SBD_results_f_9.pdf}};
% \end{tikzpicture}
% }%
% \only<2>{
\begin{tikzpicture}[remember picture,overlay]
	\node (SBDresult) at ([xshift=0.5\textwidth, yshift =-0.05\textwidth]current page.west) %or: (current page.center)
		{\includegraphics[width=0.9\textwidth]{scalability/SBD_results/6_SBD_results_f_10.pdf} };
\end{tikzpicture}
%}%
% \only<3>{
% \begin{tikzpicture}[remember picture,overlay]
% 	\node (SBDresult) at ([xshift=0.5\textwidth, yshift =-0.05\textwidth]current page.west) %or: (current page.center)
% 		{\includegraphics[width=0.9\textwidth]{scalability/SBD_results/6_SBD_results_f_11.pdf}};
% \end{tikzpicture}
% }%
% \only<4>{
% \begin{tikzpicture}[remember picture,overlay]
% 	\node (SBDresult) at ([xshift=0.5\textwidth, yshift =-0.05\textwidth]current page.west) %or: (current page.center)
% 		{\includegraphics[width=0.9\textwidth]{scalability/SBD_results/6_SBD_results_f_12.pdf}};
% \end{tikzpicture}
% }%

\end{frame}

