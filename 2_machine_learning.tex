%==============================================================================%
%% Mathematical symbols
\newcommand{\Ti}{\mbox{$T_1$}}
\newcommand{\nh}{\mbox{$n_\text{hidden}$}}
\newcommand{\Pd}{\mbox{$P_\text{dropout}$}}
\newcommand{\nla}{\mbox{$n_\text{layers}$}}
\newcommand{\Pt}{\mbox{$P_\text{teacher}$}}
\newcommand{\lr}{\mbox{$l_\text{rate}$}}
\newcommand{\bs}{\mbox{$b_\text{size}$}}
\newcommand{\R}{\mbox{$\lambda$}}
\newcommand{\e}{\mbox{$\epsilon$}}
\newcommand{\etol}{\mbox{$\epsilon_\text{tol}$}}
\newcommand{\fs}{\mbox{$f_\text{sample}$}}
\newcommand{\ff}{\mbox{$f_\text{features}$}}
\newcommand{\D}{\mbox{$D$}}
\newcommand{\ns}{\mbox{$n_\text{stages}$}}
\newcommand{\nd}{\mbox{$n_\text{degree}$}}

\newcommand{\To}{\mbox{$T_2$}}
\newcommand{\np}{\mbox{$n_\text{patience}$}}
\newcommand{\nepoch}{\mbox{$n_\text{epochs}$}}
\newcommand{\nbest}{\mbox{$n_\text{epochs}^\text{best}$}}
%==============================================================================%
%------------------------------------------------
\section{Hyperparameter tuning of machine learning problems}
%------------------------------------------------
\begin{frame}[c,noframenumbering]
	\centering
	% \setlength\fboxsep{0pt}
	\begin{titleblock}{}
		~\\%
		{\centering\LARGE COVID-19 incidence now-casting using machine learning\\}%
		~\\%
	\end{titleblock}
\end{frame}
%------------------------------------------------
\subsection{Overview of problem}
%------------------------------------------------
\begin{frame}[t]
	\frametitle{COVID-19 incidence forecasting}
	\tikzstyle{background grid}=[draw, black!50,step=.5cm]
	%
	How to \emph{design} a machine learning model for forecasting a time-series?\\
	%
	\tikzstyle{background grid}=[draw, black!50,step=.5cm]
	\begin{tikzpicture}[remember picture, overlay] %show background grid, 
		% Put the graphic inside a node. This makes it easy to place the
		% graphic and to draw on top of it. 
		% The above right option is used to place the lower left corner
		% of the image at the (0,0) coordinate. 
		\node [inner sep=0pt,above right, opacity=1.0]  at (0.0\textwidth,-0.72\textheight) (raw data) 
			{
				\only<1>{\includegraphics[width=0.43\textwidth]{machine_learning/raw_data/animation_0.pdf}}%
				\only<2>{\includegraphics[width=0.43\textwidth]{machine_learning/raw_data/animation_1.pdf}}%
				\only<3>{\includegraphics[width=0.43\textwidth]{machine_learning/raw_data/animation_2.pdf}}%
				\only<4->{\includegraphics[width=0.43\textwidth]{machine_learning/raw_data/animation_3.pdf}}%
			};
		\node [inner sep=0pt,above left, opacity=1.0]  at (0.99\textwidth,-0.65\textheight) (seq2seq) 
			{
				\only<5>{\includegraphics[width=0.55\textwidth]{machine_learning/model_1.pdf}}%
				\only<6>{\includegraphics[width=0.55\textwidth]{machine_learning/model_2.pdf}}%
				\only<7>{\includegraphics[width=0.55\textwidth]{machine_learning/model_3.pdf}}%
			};
		\only<5->{
			\node[inner sep=0pt,align=flush center,below=\belowcaptionskip of seq2seq,text width=\linewidth]
				{\vspace{-1em}{Seq2Seq model}};
		}%
		\node [inner sep=0pt, opacity=1.0]  at (0.43\textwidth,-0.15\textheight) (ct) {};%
		\node [inner sep=0pt, opacity=1.0]  at (0.43\textwidth,-0.50\textheight) (cases) {};%
		\only<-6>{
			\node [inner sep=0pt, opacity=1.0]  at (0.65\textwidth,-0.50\textheight) (input) {};%
		}%
		\only<7->{
			\node [inner sep=0pt, opacity=1.0]  at (0.60\textwidth,-0.60\textheight) (input) {};%
		}%
		% show origin
		% \fill (0,0) circle (2pt);
	\end{tikzpicture}%
	%
	\begin{tikzpicture}[overlay]
		\only<5->{\path[->,magenta,thick] (ct) edge [out=0, in=270] (input);}
		\only<5->{\path[->,magenta,thick] (cases) edge [out=0, in=270] (input);}
	\end{tikzpicture}%
	%
	\vspace{-3em}
\end{frame}
%------------------------------------------------
\subsection{Hyperparameter tuning}
%------------------------------------------------
\begin{frame}[t]
    \frametitle{Training the model}
	%
	There are several challenges associated with hyperparameter optimization\\
	%
    \begin{columns}[t] % The "c" option specifies centered vertical alignment while the "t" option is used for top vertical alignment
		
        \begin{column}{.5\textwidth} % Left column and width
        \only<1->{
			\begin{itemize}
				\item The number of epochs can be tuned using \textit{early stopping}%
				\item This is a form of \textit{regularization} to reduce over-fitting%
			\end{itemize}
        }%
        \only<3->{
			~~~However,\\
			\uncover<4->{
				~~~1) Other hyperparameters to tune
				\begin{itemize}
					\item Dropout%
					\item Training batch size%
					\item Activation function(s)%
				\end{itemize}
			}%
            \uncover<5->{~~~2) Training can be \emphasis{expensive}}\\
            \uncover<6->{~~~3) Backpropagation is \emphasis{stochastic}}
        }%
    
        \end{column}
    
        \begin{column}{.5\textwidth} % Left column and width
    
            \vspace{-1.5em}
            \begin{figure}
                \centering
                \only<1>{\includegraphics[width=0.9\textwidth]{machine_learning/training/epoch_0.pdf}}%
                \only<2>{
					\begin{animateinline}[autoplay,width=0.9\textwidth]{8}
						\ifshowanimations
							\multiframe{52}{i=0+5}{%
								\includegraphics{machine_learning/training_anim/epoch_\i.pdf}
							}
						\else
							\multiframe{1}{i=262+0}{%
								\includegraphics{machine_learning/training/epoch_\i.pdf}
							}
						\fi
					\end{animateinline}%
                }%
                \only<3->{\includegraphics[width=0.9\textwidth]{machine_learning/training/final_model.pdf}}%
                \vspace{-0.75em}
                \caption{Effect of number of epochs on testing loss}
            \end{figure}%
        \end{column}
    \end{columns}
        
\end{frame}
%------------------------------------------------
\begin{frame}[t]
	\frametitle{Hyperparameter tuning}
	\tikzstyle{background grid}=[draw, black!50,step=.5cm]
	%
	We can use StoMADS to solve such hyperparameter optimization problems \ifshowcitations\footpartcite{Khalil2021}\fi\\
	%
	\tikzstyle{background grid}=[draw, black!50,step=.5cm]
	\begin{tikzpicture}[remember picture, overlay] %show background grid, 
		% Put the graphic inside a node. This makes it easy to place the
		% graphic and to draw on top of it. 
		% The above right option is used to place the lower left corner
		% of the image at the (0,0) coordinate. 
		\node [inner sep=0pt,above right, opacity=1.0]  at (-0.01\textwidth,-0.7\textheight) (error) 
			{
				\only<1>{\includegraphics[width=0.5\textwidth]{machine_learning/box_plots/boxplot_input_dim_0.pdf}}%
				\only<2>{\includegraphics[width=0.5\textwidth]{machine_learning/box_plots/boxplot_input_dim_1.pdf}}%
				\only<3>{\includegraphics[width=0.5\textwidth]{machine_learning/box_plots/boxplot_input_dim_2.pdf}}%
				\only<4>{\includegraphics[width=0.5\textwidth]{machine_learning/box_plots/boxplot_input_dim_3.pdf}}%
				\only<5>{\includegraphics[width=0.5\textwidth]{machine_learning/box_plots/boxplot_input_dim_4.pdf}}%
				\only<6>{\includegraphics[width=0.5\textwidth]{machine_learning/box_plots/boxplot_input_dim_5.pdf}}%
				\only<7>{\includegraphics[width=0.5\textwidth]{machine_learning/box_plots/boxplot_input_dim_6.pdf}}%
				\only<8>{\includegraphics[width=0.5\textwidth]{machine_learning/box_plots/boxplot_input_dim_opt.pdf}}%
				\only<9>{\includegraphics[width=0.5\textwidth]{machine_learning/box_plots/boxplot_hid_dim_opt.pdf}}%
				\only<10>{\includegraphics[width=0.5\textwidth]{machine_learning/box_plots/boxplot_dropout_opt.pdf}}%
			};
		\node [inner sep=0pt,above left, opacity=1.0]  at (1.01\textwidth,-0.7\textheight) (prediction) 
			{
				\only<1>{\includegraphics[width=0.5\textwidth]{machine_learning/models/model_input_dim_0.pdf}}%
				\only<2>{\includegraphics[width=0.5\textwidth]{machine_learning/models/model_input_dim_1.pdf}}%
				\only<3>{\includegraphics[width=0.5\textwidth]{machine_learning/models/model_input_dim_2.pdf}}%
				\only<4>{\includegraphics[width=0.5\textwidth]{machine_learning/models/model_input_dim_3.pdf}}%
				\only<5>{\includegraphics[width=0.5\textwidth]{machine_learning/models/model_input_dim_4.pdf}}%
				\only<6>{\includegraphics[width=0.5\textwidth]{machine_learning/models/model_input_dim_5.pdf}}%
				\only<7>{\includegraphics[width=0.5\textwidth]{machine_learning/models/model_input_dim_6.pdf}}%
				\only<8>{\includegraphics[width=0.5\textwidth]{machine_learning/models/final_model_input_dim.pdf}}%
				\only<9>{\includegraphics[width=0.5\textwidth]{machine_learning/models/final_model_hid_dim.pdf}}%
				\only<10->{\includegraphics[width=0.5\textwidth]{machine_learning/models/final_model_dropout.pdf}}%
			};
		% show origin
		% \fill (0,0) circle (2pt);
	\end{tikzpicture}%
	%
	\begin{columns}[t] % The "c" option specifies centered vertical alignment while the "t" option is used for top vertical alignment
		\begin{column}{.42\textwidth} % Left column and width
			\vspace{-2.0em}
			% Optimization problem
			\uncover<11->{
				\begin{exampleblock}{Objective and constraints}
					\vspace{-0.0em}
					\begin{equation*}
						\begin{aligned}
							& \underset{\mathbf{x}}{\text{min}}
							& & f(\mathbf{x}) = \mathbb{E}_{\Theta}\left[{f}_{\Theta}(\mathbf{x}) = \mathrm{error}_\mathrm{CV}\right]\\
							& \text{where}
							& & \Theta\mathrm{:realizations}
						\end{aligned}
					\end{equation*}
				\end{exampleblock}
			}%
			\vspace{-0.5em}
			\uncover<11->{
				% Variables
				\begin{alertblock}{Design variables ($\mathbf{x}$)}
					\vspace{-0.0em}
						\begin{itemize}\itemsep0em
							\item $T_1:$ Input dimension
							\item $n_\text{hidden}:$ Number of hidden neurons
							\item $P_\text{dropout}:$ Probability of dropout, etc.
						\end{itemize}
				\end{alertblock}
			}%
			\vspace{-0.5em}
			\uncover<11->{
				% Parameters
				\begin{blueblock}{Randomly seeded parameters}
					\vspace{-0.0em}
					\begin{itemize}\itemsep0em
						\item Initial weights
						\item Gradient descent steps
					\end{itemize}
				\end{blueblock}
			}%
		\end{column}
		%
		\begin{column}{.5\textwidth} % Left column and width
		\end{column}
	
	\end{columns}
	%
	\vspace{-3em}
\end{frame}
\addtocounter{footnote}{-1}
%------------------------------------------------
\subsection{Results}
\subsubsection{Optimal hyperparameters}
%------------------------------------------------
\begin{frame}[t]
	\frametitle{Results: Optimal hyperparameters}
	\tikzstyle{background grid}=[draw, black!50,step=.5cm]
	%
	Optimal hyperparameters for the \emph{Seq2Seq} model:\\
	%
	\tikzstyle{background grid}=[draw, black!50,step=.5cm]
	\begin{tikzpicture}[remember picture, overlay] %show background grid, 
		% Put the graphic inside a node. This makes it easy to place the
		% graphic and to draw on top of it. 
		% The above right option is used to place the lower left corner
		% of the image at the (0,0) coordinate. 
		\node [inner sep=0pt,above left, opacity=1.0]  at (1.01\textwidth,-0.73\textheight) (prediction) 
			{\includegraphics[width=0.5\textwidth]{machine_learning/predictions/model_predictions_test_S2S_mean_Ct_daily_cases.pdf}};
		% show origin
		% \fill (0,0) circle (2pt);
	\end{tikzpicture}%
	%
	\begin{columns}[t] % The "c" option specifies centered vertical alignment while the "t" option is used for top vertical alignment
		\begin{column}{.42\textwidth} % Left column and width
			\vspace{-2.0em}
            % Column widths
            \newcommand{\ocwb}{3.7cm}
            \newcommand{\ocwc}{1cm}
            \newcommand{\ocwd}{1.2cm}
            \newcommand{\ocwe}{3cm}
            %
            \begin{table}[h!]
                \centering
                \footnotesize
                \renewcommand{\arraystretch}{1.5}% Wider
                \begin{tabular}{L{\ocwb}C{\ocwc}C{\ocwd}} \toprule
                    \multicolumn{2}{c}{\bf Hyperparameter}	& \bf Value         \\ \toprule
                    Sliding window size  		& \Ti		&	6				\\
                    Number of hidden neurons	& \nh		&	1500			\\
                    Probability of dropout		& \Pd		&	0.8				\\
                    Number of hidden layers		& \nh		&	2				\\
                    Teacher forcing probability	& \Pt		&	0.3				\\
                    Learning rate 				& \lr		&	$1\times10^{-4}$\\
                    batch size 					& \bs		&	32				\\\hline
                \end{tabular}
            \end{table}
        \end{column}
		%
		\begin{column}{.5\textwidth} % Left column and width
		\end{column}
	
	\end{columns}
	%
	\vspace{-3em}
\end{frame}
%------------------------------------------------
\begin{frame}[t]
	\frametitle{Results: Optimal hyperparameters}
	\tikzstyle{background grid}=[draw, black!50,step=.5cm]
	%
	Optimal hyperparameters for the \emph{support vector machine regression (SVR)} model:\\
	%
	\tikzstyle{background grid}=[draw, black!50,step=.5cm]
	\begin{tikzpicture}[remember picture, overlay] %show background grid, 
		% Put the graphic inside a node. This makes it easy to place the
		% graphic and to draw on top of it. 
		% The above right option is used to place the lower left corner
		% of the image at the (0,0) coordinate. 
		\node [inner sep=0pt,above left, opacity=1.0]  at (1.01\textwidth,-0.73\textheight) (prediction) 
			{\includegraphics[width=0.5\textwidth]{machine_learning/predictions/model_predictions_test_SVR_mean_Ct_daily_cases.pdf}};
		% show origin
		% \fill (0,0) circle (2pt);
	\end{tikzpicture}%
	%
	\begin{columns}[t] % The "c" option specifies centered vertical alignment while the "t" option is used for top vertical alignment
		\begin{column}{.42\textwidth} % Left column and width
			\vspace{-2.0em}
            % Column widths
            \newcommand{\ocwb}{3.7cm}
            \newcommand{\ocwc}{1cm}
            \newcommand{\ocwd}{1.2cm}
            \newcommand{\ocwe}{3cm}
            %
            \begin{table}[h!]
                \centering
                \footnotesize
                \renewcommand{\arraystretch}{1.5}% Wider
                \begin{tabular}{L{\ocwb}C{\ocwc}C{\ocwd}} \toprule
                    \multicolumn{2}{c}{\bf Hyperparameter}				    & \bf Value             \\ \toprule
                    Sliding window size 						& \Ti	    &	6					\\
                    Ridge factor								& \R	    &	$1\times10^{-4}$	\\
                    Margin of tolerance							& \e	    &	$1\times10^{-2}$	\\
                    Stopping criteria tolerance					& \etol	    &	0.1					\\
                    Learning rate 								& \lr	    &	$1\times10^{-5}$	\\ \hline
                \end{tabular}
            \end{table}
            \uncover<2->{Support vector machine models have \emphasis{deterministic} performance}
        \end{column}
		%
		\begin{column}{.5\textwidth} % Left column and width
		\end{column}
	
	\end{columns}
	%
	\vspace{-3em}
\end{frame}
%------------------------------------------------
\subsubsection{Prospective validation}
%------------------------------------------------
\begin{frame}[t]
	\frametitle{Results: Prospective validation}
	\tikzstyle{background grid}=[draw, black!50,step=.5cm]
	%
	Performance of models on \emphasis{unseen} data (first 4 months of 2021):\\
	%
	\tikzstyle{background grid}=[draw, black!50,step=.5cm]
	\begin{tikzpicture}[remember picture, overlay] %show background grid, 
		% Put the graphic inside a node. This makes it easy to place the
		% graphic and to draw on top of it. 
		% The above right option is used to place the lower left corner
		% of the image at the (0,0) coordinate. 
		\node [inner sep=0pt,above left, opacity=1.0]  at (1.01\textwidth,-0.73\textheight) (prediction) 
			{
                \only<1>{\includegraphics[width=0.5\textwidth]{machine_learning/predictions/model_predictions_unseen_S2S_mean_Ct_daily_cases_truth.pdf}}%
                \only<2>{\includegraphics[width=0.5\textwidth]{machine_learning/predictions/model_predictions_unseen_S2S_mean_Ct_daily_cases.pdf}}%
                \only<3>{\includegraphics[width=0.5\textwidth]{machine_learning/predictions/model_predictions_unseen_SVR_mean_Ct_daily_cases.pdf}}%
            };
		% show origin
		% \fill (0,0) circle (2pt);
	\end{tikzpicture}%
	%
	\begin{columns}[t] % The "c" option specifies centered vertical alignment while the "t" option is used for top vertical alignment
		\begin{column}{.42\textwidth} % Left column and width
			\vspace{-2.0em}
            % Column widths
            \newcommand{\ocwa}{5cm}
            \newcommand{\ocwb}{1.6cm}
            \newcommand{\ocwc}{1.6cm}
            \newcommand{\ocwd}{1.6cm}
            %
            \begin{table}[h!]
                \centering
                \footnotesize
                \renewcommand{\arraystretch}{1.5}% Wider
                \begin{tabular}{L{\ocwa}C{\ocwd}} \toprule
                    \multicolumn{1}{c}{\bf Model}               & \multicolumn{1}{c}{\bf Test error}    \\ \toprule
                    \only<2>{\emphasis}{Seq2Seq}		        & \only<2>{\emphasis}{0.571}            \\
                    Long short term memory (LSTM) cell          & 0.326                                 \\
                    feedforward neural network	                & 0.255                                 \\
                    \only<3>{\emphasis}{Support vector machine} & \only<3>{\emphasis}{0.168}            \\
                    Gradient boosting 		                    & 1.444                                 \\
                    Linear regression 		                    & 0.160                                 \\ \bottomrule
                \end{tabular}
            \end{table}
        \end{column}
        %
		\begin{column}{.5\textwidth} % Left column and width
		\end{column}
	
	\end{columns}
	%
	\vspace{-3em}
\end{frame}
%------------------------------------------------
\begin{frame}[t]
	\frametitle{Results: Prospective validation}
	\tikzstyle{background grid}=[draw, black!50,step=.5cm]
	%
	Effect of increasing number of training days (Adding 1 month of data):\\
	%
	\tikzstyle{background grid}=[draw, black!50,step=.5cm]
	\begin{tikzpicture}[remember picture, overlay] %show background grid, 
		% Put the graphic inside a node. This makes it easy to place the
		% graphic and to draw on top of it. 
		% The above right option is used to place the lower left corner
		% of the image at the (0,0) coordinate. 
		\node [inner sep=0pt,above left, opacity=1.0]  at (1.01\textwidth,-0.73\textheight) (prediction) 
			{
                \only<1>{\includegraphics[width=0.5\textwidth]{machine_learning/predictions/model_predictions_unseen_S2S_mean_Ct_daily_cases.pdf}}%
                \only<2>{\includegraphics[width=0.5\textwidth]{machine_learning/predictions/model_predictions_unseen_S2S_mean_Ct_daily_cases_G12.pdf}}%
                \only<3>{\includegraphics[width=0.5\textwidth]{machine_learning/predictions/model_predictions_unseen_SVR_mean_Ct_daily_cases_G12.pdf}}%
            };
		% show origin
		% \fill (0,0) circle (2pt);
	\end{tikzpicture}%
	%
	\begin{columns}[t] % The "c" option specifies centered vertical alignment while the "t" option is used for top vertical alignment
		\begin{column}{.42\textwidth} % Left column and width
			\vspace{-2.0em}
            % Column widths
            \newcommand{\ocwa}{5cm}
            \newcommand{\ocwb}{1.6cm}
            \newcommand{\ocwc}{1.6cm}
            \newcommand{\ocwd}{1.6cm}
            %
            \begin{table}[h!]
                \centering
                \footnotesize
                \renewcommand{\arraystretch}{1.5}% Wider
                \begin{tabular}{L{\ocwa}C{\ocwd}} \toprule
                    \multicolumn{1}{c}{\bf Model}               & \multicolumn{1}{c}{\bf Test error}                                \\ \toprule
                    \only<1-2>{\emphasis}{Seq2Seq}		        & \only<1>{\emphasis{0.571}}\only<2->{\bf\color{darkgreen}{0.106}}   \\
                    Long short term memory (LSTM) cell          & 0.326                                                             \\
                    feedforward neural network	                & 0.255                                                             \\
                    \only<3>{\emphasis}{Support vector machine} & \only<1-2>{0.168}\only<3>{\bf\color{darkgreen}{0.140}}            \\
                    Gradient boosting 		                    & 1.444                                                             \\
                    Linear regression 		                    & 0.160                                                             \\ \bottomrule
                \end{tabular}
            \end{table}
        \end{column}
        %
		\begin{column}{.5\textwidth} % Left column and width
		\end{column}
	
	\end{columns}
	%
	\vspace{-3em}
\end{frame}
%------------------------------------------------
\subsection{UML diagram}
%------------------------------------------------
\begin{frame}[t]
	\frametitle{UML diagram of code}
    %
	\begin{figure}
        \centering
		\only<1>{\includegraphics[width=1.0\textwidth]{machine_learning/classes_hpo.pdf}}%
		\only<2>{\includegraphics[width=1.0\textwidth]{machine_learning/packages_hpo.pdf}}%
    \end{figure}%
\end{frame}
%------------------------------------------------