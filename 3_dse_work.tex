%==============================================================================%
%% Mathematical symbols
\definecolor{design}{HTML}{D0A31D}
\definecolor{spec}{HTML}{FF7575}
\definecolor{margin}{HTML}{CC99FF}
\definecolor{perf}{HTML}{87C061}
\newcommand\design[1]{{\color{design}#1}}
\newcommand\spec[1]{{\color{spec}#1}}
\newcommand\margin[1]{{\color{margin}#1}}
\newcommand\perf[1]{{\color{perf}#1}}
%==============================================================================%
%------------------------------------------------
\section{Ongoing work}
%------------------------------------------------
\begin{frame}[c,noframenumbering]
	\centering
	% \setlength\fboxsep{0pt}
	\begin{titleblock}{}
		~\\%
		{\centering\LARGE Ongoing work: Design space exploration\\}%
		~\\%
	\end{titleblock}
\end{frame}
%------------------------------------------------
\subsection{Introduction on DSE}
%------------------------------------------------
\begin{frame}
	\frametitle{Design space exploration at the conceptual level}
	
	\only<1>{Consider a functional model (FM) of a product (conceptual design)}%
	\only<2>{\emph{hyperlinks}\textsuperscript{1} used in multi-domain modelling}%
	\only<3>{to link FM $\to$ margin value analysis\textsuperscript{2}}%
	\only<4->{different concepts have different preliminary designs}%
	{\color{white}\ifshowcitations\footpartcite{Ahmad2013}\textsuperscript{,}\footpartcite{Brahma2020}\fi}%
	
	\begin{columns}[t] % The "c" option specifies centered vertical alignment while the "t" option is used for top vertical alignment
			
		\begin{column}{.5\textwidth} % Left column and width
	
			\centering\textbf{conceptual design}
			\begin{figure}
				\only<1>{\includegraphics[height=5cm]{design_space_exploration/Intro/efm_example_1.pdf}}%
				\only<2>{\includegraphics[height=5cm]{design_space_exploration/Intro/efm_example_2.pdf}}%
				\only<3-5>{\includegraphics[height=5cm]{design_space_exploration/Intro/efm_example_3.pdf}}%
				\only<6->{\includegraphics[height=5cm]{design_space_exploration/Intro/efm_example_4.pdf}}%
			\end{figure}
			%
			\tikzstyle{background grid}=[draw, black!50,step=.5cm]
			\begin{tikzpicture}[remember picture, overlay, shift={(2,6)}] %show background grid, 
				\node [draw=black, line width=0.5mm, inner sep=0pt,above left, opacity=1.0]  at (-0.35\textwidth,-0.45\textheight) (trsA) 
					{
						\only<3->{\includegraphics[width=0.15\textwidth]{design_space_exploration/Intro/TRS_alt_1.pdf}}%
					};
				\node [draw=black, line width=0.5mm, inner sep=0pt,above left, opacity=1.0]  at (-0.15\textwidth,-0.45\textheight) (trsB) 
					{
						\only<6->{\includegraphics[width=0.15\textwidth]{design_space_exploration/Intro/TRS_alt_3.pdf}}%
					};
				% show origin
				% \fill (0,0) circle (2pt);
			\end{tikzpicture}%
			%
		\end{column}
	
		\vrule{}
	
		\begin{column}{.5\textwidth} % Left column and width
	
			\centering\textbf{preliminary design}
	
			\begin{figure}
				\uncover<2->{
					\only<-2>{\includegraphics[height=5cm]{design_space_exploration/Intro/man_example_1.pdf}}%
				}%
				\only<3-5>{\includegraphics[height=5cm]{design_space_exploration/Intro/man_example_2.pdf}}%
				\only<6->{\includegraphics[height=5cm]{design_space_exploration/Intro/man_example_3.pdf}}%
			\end{figure}
			%
			\tikzstyle{background grid}=[draw, black!50,step=.5cm]
			\begin{tikzpicture}[remember picture, overlay, shift={(2,6)}] %show background grid, 
				\node [draw=black, line width=0.1mm, fill=white!20, inner sep=0pt,above left, opacity=1.0]  at (-0.4\textwidth,-0.65\textheight) (loadA) 
					{
						\only<5->{\includegraphics[width=0.35\textwidth]{design_space_exploration/Intro/load_cases_1.pdf}}%
					};
				\node [draw=black, line width=0.1mm, fill=white!20, inner sep=0pt,above left, opacity=1.0]  at (-0.4\textwidth,-0.15\textheight) (loadB) 
					{
						\only<7->{\includegraphics[width=0.35\textwidth]{design_space_exploration/Intro/load_cases_2.pdf}}%
					};
				\node [draw=black, line width=0.1mm, fill=white!20, inner sep=1pt,above left, opacity=1.0]  at (0.22\textwidth,-0.65\textheight) (manuA) 
					{
						\only<4->{\includegraphics[width=0.35\textwidth]{design_space_exploration/TEMP_GIF/frame_180.png}}%
					};
				% show origin
				% \fill (0,0) circle (2pt);

				\path (-3,-5) coordinate (B1 A)
					  (-4.5,-8) coordinate (B3 A);

			\end{tikzpicture}%
			%
		\end{column}
	\end{columns}
	
	\vspace{-2em}
\end{frame}
\addtocounter{footnote}{-2}
%------------------------------------------------
\begin{frame}
	\frametitle{Design space exploration at the conceptual level}
	
	Multi-domain modelling (MDM) matrix for change propagation management can be used to represent hyperlinks \footnotemark[1]
	\begin{figure}
		\includegraphics[height=5cm]{design_space_exploration/Intro/MDM_example.pdf}
	\end{figure}
	
	\vspace{-2em}
	{\color{white}
		\footpartcite{Koh2012}[1]
		\addtocounter{footnote}{-1}
	}
	
\end{frame}
%------------------------------------------------
\subsection{Example problem: Design of TRS strut}
%------------------------------------------------
\begin{frame}[t,label=ap2_2]
	\frametitle{Overview of turbine rear frame design}
	\vspace{-1em}
	\only<1-2>{The turbine rear structure (TRS) sits aft of the turbine}%
	\only<3>{We consider the design of a single strut}%
	\only<4>{Variable thermal loads are expected}%
	\only<5-7>{Which cause thermal expansion \uncover<6-7>{and stress on the strut}}%
	\only<8>{The entire problem can be represented compactly as an MDM}%
	~\\
	\begin{columns}[t] % The "c" option specifies centered vertical alignment while the "t" option is used for top vertical alignment
			
		\begin{column}{.45\textwidth} % Left column and width
			\footnotesize
			Design parameters:
			\design{
				\uncover<3->{
				\begin{itemize}\itemsep0em 
					\item \only<8>{$D1$} vane width $w$
					\item \only<8>{$D2$} vane height $h$
					\item \only<8>{$D3$} lean angle $\theta$
					\item \only<8>{$D4$} Material - yield stress $\sigma_y$ (off-the shelf)
				\end{itemize}
			}}%
			Input Specifications:
			\spec{
				\uncover<5->{
				\begin{itemize}\itemsep0em 
					\item \only<8>{$S1$} Nacelle temperature $T_1$ ($\downarrow$)
					\item \only<8>{$S2$} Gas surface temperature $T_2$ ($\uparrow$)
				\end{itemize}
			}}%
			Design margins:
			\margin{
				\uncover<6->{
				\begin{itemize}\itemsep0em 
					\item \only<8>{$E1$} compression $\sigma_y - \sigma_a \ge 0$
					\item \only<8>{$E2$} bending $\sigma_y - \sigma_m \ge 0$
					\item \only<8>{$E3$} buckling $F_\text{buckling} - F_a \ge 0$
				\end{itemize}
			}}%
			Performance parameters:
			\perf{
				\uncover<7->{
				\begin{itemize}\itemsep0em 
					\item \only<8>{$P1$}  Weight
				\end{itemize}
			}}%
		\end{column}
	
		\begin{column}{.4\textwidth} % Left column and width
			\begin{figure}
				\only<1>{
					\includegraphics[width=6.0cm]{design_space_exploration/TRS/engine.png}
					\vspace{-0.2em}
					\caption{Trent 900 aeroengine}
				}%
				\only<2>{
					\includegraphics[width=5.0cm]{design_space_exploration/TRS/rear_view.pdf}
					\vspace{-0.2em}
					\caption{Turbine rear structure}
				}%
				\only<3>{
					\includegraphics[width=7.0cm]{design_space_exploration/TRS/cross_section_view.pdf}
					\vspace{-0.2em}
					\caption{Turbine rear structure: cross-sectional view}
				}%
				\only<4>{
					\includegraphics[width=6.0cm]{design_space_exploration/TRS/expansion_1.pdf}
					\vspace{-0.2em}
					\caption{Thermal expansion}
				}%
				\only<5>{
					\includegraphics[width=6.0cm]{design_space_exploration/TRS/expansion_2.pdf}
					\vspace{-0.2em}
					\caption{Thermal expansion}
				}%
				\only<6-7>{
					\includegraphics[width=7.0cm]{design_space_exploration/TRS/loads.pdf}
					\vspace{-0.2em}
					\caption{Load cases}
				}%
				\only<8>{
					\vspace{-1.2em}
					\includegraphics[width=6.7cm]{design_space_exploration/TRS/MDM_strut.pdf}
					\vspace{-0.2em}
					\caption{Mulit-domain matrix of strut example}
				}%
			\end{figure}
		\end{column}
		
	\end{columns}
	
\end{frame}
%------------------------------------------------
\subsection{Applying the margin value method}
%------------------------------------------------
\subsubsection{Calculating impact on performance}
%------------------------------------------------
\begin{frame}[t,label=ap2_2]
	\frametitle{Calculating impact of excess margin on performance}
	\vspace{-1em}
	Need to construct a response surface%
	~\\
	\begin{columns}[t] % The "c" option specifies centered vertical alignment while the "t" option is used for top vertical alignment
			
		\begin{column}{.45\textwidth} % Left column and width
			\footnotesize
			Design parameters:
			\design{
				\begin{itemize}\itemsep0em 
					\item vane width $w$
					\item vane height $h$
					\item lean angle $\theta$
					\item Material - yield stress $\sigma_y$ (off-the shelf)
				\end{itemize}
			}%
			Input Specifications:
			\spec{
				\begin{itemize}\itemsep0em 
					\item Nacelle temperature $T_1$ ($\downarrow$)
					\item Gas surface temperature $T_2$ ($\uparrow$)
				\end{itemize}
			}%
			Design margins:
			\margin{
				\begin{itemize}\itemsep0em 
					\item compression $\sigma_y - \sigma_a \ge 0$
					\item bending $\sigma_y - \sigma_m \ge 0$
					\item buckling $F_\text{buckling} - F_a \ge 0$
				\end{itemize}
			}%
			Performance parameters:
			\perf{
				\begin{itemize}\itemsep0em 
					\item Weight
				\end{itemize}
			}%
		\end{column}
	
		\begin{column}{.4\textwidth} % Left column and width
			\footnotesize
			Need to find a function $f$ that translates \margin{excess} to \perf{performance}
			\renewcommand\baselinestretch{0.5}
			\only<1>{
				\small
				\begin{equation*}
					\begin{aligned}
						& \hat{\mathbf{p}} = f(\mathbf{e})\\
						& \mathbf{p} = \left[p_1,p_2,\cdots,p_j\right]^{\mathit{T}}\\
						& \mathbf{e} = \left[e_1,e_2,\cdots,e_m\right]^{\mathit{T}}
					\end{aligned}
				\end{equation*}
			}%
			\only<2>{
				\small
				\begin{equation*}
					\begin{aligned}
						& \hat{\mathbf{p}} = f(\mathbf{e})\\
						& \mathbf{p}_1^\mathrm{threshold} = \left[p_1,p_2,\cdots,p_j\right]^{\mathit{T}}\\
						& \mathbf{e} = \left[{\color{red}0},e_2,\cdots,e_m\right]^{\mathit{T}}
					\end{aligned}
				\end{equation*}
			}%
			\only<3>{
				\small
				\begin{equation*}
					\begin{aligned}
						& \hat{\mathbf{p}} = f(\mathbf{e})\\
						& \mathbf{p}_2^\mathrm{threshold} = \left[p_1,p_2,\cdots,p_j\right]^{\mathit{T}}\\
						& \mathbf{e} = \left[e_1,{\color{red}0},\cdots,e_m\right]^{\mathit{T}}
					\end{aligned}
				\end{equation*}
			}%
			\only<4>{
				\renewcommand{\kbldelim}{[}% Left delimiter
				\renewcommand{\kbrdelim}{]}% Right delimiter
				\[
				  \text{Impact} = \kbordermatrix{
										& \text{weight} \\
					\text{buckling} 	& 0.545  	\\
					\text{compression} 	& 0.111 	\\
					\text{bending} 		& -1.011
				  }
				\]
			}%

			\vspace{-1em}
			\begin{figure}
				\only<1>{
					\includegraphics[width=5.0cm]{design_space_exploration/MVM/IOP_1.pdf}
				}%
				\only<2>{
					\includegraphics[width=5.0cm]{design_space_exploration/MVM/IOP_2.pdf}
				}%
				\only<3>{
					\includegraphics[width=5.0cm]{design_space_exploration/MVM/IOP_3.pdf}
				}%
			\end{figure}
		\end{column}
		
	\end{columns}
	
\end{frame}
%------------------------------------------------
\subsubsection{Calculating change absorption capability}
%------------------------------------------------
\begin{frame}[t,label=ap2_2]
	\frametitle{Calculating deterioration (allowable change in specifications)}
	\vspace{-1em}
	We calculate the maximum allowable deterioration in the \spec{specifications}%
	~\\
	\begin{columns}[t] % The "c" option specifies centered vertical alignment while the "t" option is used for top vertical alignment
			
		\begin{column}{.45\textwidth} % Left column and width
			\footnotesize
			Design parameters:
			\design{
				\begin{itemize}\itemsep0em 
					\item vane width $w$
					\item vane height $h$
					\item lean angle $\theta$
					\item Material - yield stress $\sigma_y$ (off-the shelf)
				\end{itemize}
			}%
			Input Specifications:
			\spec{
				\begin{itemize}\itemsep0em 
					\item Nacelle temperature $T_1$ ($\uparrow$)
					\item Gas surface temperature $T_2$ ($\downarrow$)
				\end{itemize}
			}%
			Design margins:
			\margin{
				\begin{itemize}\itemsep0em 
					\item compression $\sigma_y - \sigma_a \ge 0$
					\item bending $\sigma_y - \sigma_m \ge 0$
					\item buckling $F_\text{buckling} - F_a \ge 0$
				\end{itemize}
			}%
			Performance parameters:
			\perf{
				\begin{itemize}\itemsep0em 
					\item Weight
				\end{itemize}
			}%
		\end{column}
	
		\begin{column}{.4\textwidth} % Left column and width
			Gradually change each input specifications until any margin is consumed
			\renewcommand\baselinestretch{0.5}
			\renewcommand{\kbldelim}{[}% Left delimiter
			\renewcommand{\kbrdelim}{]}% Right delimiter
			\only<1>{
				\[
				  \mathbf{s} = \kbordermatrix{
							&		\\
					T_1 	& 450  	\\
					T_2 	& 425 	\\
				  }
				\]
			}%
			\only<2>{
				\[
				  \mathbf{s} = \kbordermatrix{
					  						&			\\
					\color{red}T_1\uparrow 	& 454.5   	\\
					T_2						& 425 		\\
				  }
				\]
			}%
			\only<3>{
				\[
				  \mathbf{s} = \kbordermatrix{
					  							&			\\
					T_1 						& 450   	\\
					\color{red}T_2\downarrow	& 420.75 	\\
				  }
				\]
			}%
			\only<4>{
				\[
				  \mathbf{s}^\text{max} = \kbordermatrix{
					  				&			\\
					T_1\uparrow 	& 454.5   	\\
					T_2\downarrow	& 420.75 	\\
				  }
				\]
			}%
			\only<5>{
				\[
				  \text{deterioration} = \kbordermatrix{
					  				&			\\
					T_1				& 0.01  	\\
					T_2				& 0.01 		\\
				  }
				\]
			}%
			\only<1>{
				\[
				  \mathbf{e} = \kbordermatrix{
										&				\\
					\text{buckling} 	& 597.440		\\
					\text{compression} 	& 72.6904696	\\
					\text{bending} 		& 451.9918707	\\
				  }
				\]
			}%
			\only<2-3>{
				\[
				  \mathbf{e} = \kbordermatrix{
										&				\\
					\text{buckling} 	& 488.404		\\
					\text{compression} 	& \color{red}0	\\
					\text{bending} 		& 450.4889005	\\
				  }
				\]
			}%
		\end{column}
		
	\end{columns}
	
\end{frame}
%------------------------------------------------
\begin{frame}[t,label=ap2_2]
	\frametitle{Calculating deterioration (allowable change in specifications)}
	\vspace{-1em}
	\only<1>{We calculate the change absorption capability as a function of deterioration}%
	~\\
	\begin{columns}[t] % The "c" option specifies centered vertical alignment while the "t" option is used for top vertical alignment
			
		\begin{column}{.45\textwidth} % Left column and width
			\footnotesize
			Design parameters:
			\design{
				\begin{itemize}\itemsep0em 
					\item vane width $w$
					\item vane height $h$
					\item lean angle $\theta$
					\item Material - yield stress $\sigma_y$ (off-the shelf)
				\end{itemize}
			}%
			Input Specifications:
			\spec{
				\begin{itemize}\itemsep0em 
					\item Nacelle temperature $T_1$ ($\downarrow$)
					\item Gas surface temperature $T_2$ ($\uparrow$)
				\end{itemize}
			}%
			Design margins:
			\margin{
				\begin{itemize}\itemsep0em 
					\item compression $\sigma_y - \sigma_a \ge 0$
					\item bending $\sigma_y - \sigma_m \ge 0$
					\item buckling $F_\text{buckling} - F_a \ge 0$
				\end{itemize}
			}%
			Performance parameters:
			\perf{
				\begin{itemize}\itemsep0em 
					\item Weight
				\end{itemize}
			}%
		\end{column}
	
		\begin{column}{.4\textwidth} % Left column and width
			Change absorption is calculated using the following equation:
			\renewcommand\baselinestretch{0.5}
			\begin{equation*}
				\text{absorption}_{mi} = \dfrac{t^\text{new}_{mi} - t^\text{nominal}_{mi}}{t^\text{nominal}_{mi}\times\text{deterioration}_i}
			\end{equation*}
			for each margin node $m$ and specification $i$
			\renewcommand{\kbldelim}{[}% Left delimiter
			\renewcommand{\kbrdelim}{]}% Right delimiter
			\only<1>{
				\[
				  \text{absorption} = \kbordermatrix{
										& T_1	& T_2	\\
					\text{buckling} 	&2.79	& 3.87	\\
					\text{compression} 	&1.98	& 3.04	\\
					\text{bending} 		&2.58	& 3.66	\\
				  }
				\]
			}%
		\end{column}
		
	\end{columns}
	
\end{frame}
%------------------------------------------------
\subsubsection{Calculating change absorption capability}
%------------------------------------------------
\begin{frame}[t,label=ap2_2]
	\frametitle{Constructing the margin value plot (MVP)}
	\vspace{-1em}
	{Using ``impact'' and ``absorption'' we can construct a tradespace (MVP)}%
	~\\
	\begin{columns}[t] % The "c" option specifies centered vertical alignment while the "t" option is used for top vertical alignment
			
		\begin{column}{.45\textwidth} % Left column and width
			\footnotesize
			Design parameters:
			\design{
				\begin{itemize}\itemsep0em 
					\only<1-2>{
						\item vane width
						\item vane height
						\item lean angle
					}%
					\only<3>{
						\item vane width $w = 6.78$
						\item vane height $h = 125.60$
						\item lean angle $\theta = 14.6$
					}%
					\only<4>{
						\item vane width $w = 11.5$
						\item vane height $h = 113.33$
						\item lean angle $\theta = 18.0$
					}%
					\only<5->{
						\item vane width $w = 5.42$
						\item vane height $h = 106.14$
						\item lean angle $\theta = 15.7$
					}%
					\item Material - yield stress $\sigma_y$ (off-the shelf)
				\end{itemize}
			}%
			Input Specifications:
			\spec{
				\begin{itemize}\itemsep0em 
					\item Nacelle temperature $T_1$ ($\downarrow$)
					\item Gas surface temperature $T_2$ ($\uparrow$)
				\end{itemize}
			}%
			Design margins:
			\margin{
				\begin{itemize}\itemsep0em 
					\item compression $\sigma_y - \sigma_a \ge 0$
					\item bending $\sigma_y - \sigma_m \ge 0$
					\item buckling $F_\text{buckling} - F_a \ge 0$
				\end{itemize}
			}%
			Performance parameters:
			\perf{
				\begin{itemize}\itemsep0em 
					\item Weight
				\end{itemize}
			}%
		\end{column}
	
		\begin{column}{.4\textwidth} % Left column and width
			\only<-2>{\uncover<2->{We aggregate across each \perf{performance} and \spec{specification}}}%
			\only<3->{%
				Design A: $d = 2.70$\\
				\uncover<4->{Design B: $d = 3.13$}\\
				\uncover<5->{Design C: $d = -3.77$}
			}%
			\renewcommand\baselinestretch{0.5}
			\renewcommand{\kbldelim}{[}% Left delimiter
			\renewcommand{\kbrdelim}{]}% Right delimiter
			\only<1>{
				\[
				  	\text{Impact} = \kbordermatrix{
										& \text{weight} \\
					\text{buckling} 	& 0.545  		\\
					\text{compression} 	& 0.111 		\\
					\text{bending} 		& -1.011		\\
				  }
				\]
				\[
				  	\text{absorption} = \kbordermatrix{
										& T_1	& T_2	\\
					\text{buckling} 	&2.79	& 3.87	\\
					\text{compression} 	&1.98	& 3.04	\\
					\text{bending} 		&2.58	& 3.66	\\
				  }
				\]
			}%
			\only<2>{
				\[
				  	\text{Impact} = \kbordermatrix{
										&  			\\
					\text{buckling} 	& 0.545		\\
					\text{compression} 	& 0.111		\\
					\text{bending} 		& -1.011	\\
				  }
				\]
				\[
				  	\text{absorption} = \kbordermatrix{
										& 		\\
					\text{buckling} 	&3.33	\\
					\text{compression} 	&2.51	\\
					\text{bending} 		&3.12	\\
				  }
				\]
			}%
			\only<3->{
				\begin{figure}
					\only<3>{
						\includegraphics[width=6.0cm]{design_space_exploration/MVP/MVP_1.pdf}
					}%
					\only<4>{
						\includegraphics[width=6.0cm]{design_space_exploration/MVP/MVP_2.pdf}
					}%
					\only<5->{
						\includegraphics[width=6.0cm]{design_space_exploration/MVP/MVP_3.pdf}
					}%
				\end{figure}
			}%
		\end{column}
		
	\end{columns}
\end{frame}
%------------------------------------------------
\subsection{Implementation}
%------------------------------------------------
\begin{frame}[fragile,t]
	\frametitle{Pseudo-code of MVM}
    %
	\vspace{-1em}
	\begin{algorithm}[H] % must use [H]
		\tiny
		\DontPrintSemicolon % Some LaTeX compilers require you to use \dontprintsemicolon instead
		\KwIn{
			set of possible choices $\mathcal{D}$, chosen design $\mathbf{c}\in\mathcal{D}$,
			the MAN given by analytic functions, models, and surrogates,
			$n_\mathrm{samples}$, Joint PDF $F_\mathbf{S}(\mathbf{s})$
		}
		\KwOut{estimated mean impact and change absorption $\bar{\bar{\mathbf{a}}}$, $\bar{\bar{\mathbf{i}}}$}
		%
		\textbf{[0] Initialization}: 
			set counter $i_\mathrm{sample} \gets 0$, initialize aggregate absorption and impact $\bar{\bar{\mathbf{a}}} \in \mathbb{R}^e$ and $\bar{\bar{\mathbf{i}}} \in \mathbb{R}^e$\;
		%
		\textbf{[1] Monte-carlo simulation}: 
			draw sample input specifications $\mathbf{S}\in\mathbb{R}^n \sim F_\mathbf{S}(\mathbf{s})$\;
		%
		\textbf{[2] Calculate threshold, decided, excess, and performance values}: 
			$\mathbf{t} = \mathbf{f}(\mathbf{S},\mathbf{c})$, 
			$\mathbf{d} = \mathbf{g}(\mathbf{c})$, 
			$\mathbf{e} = \left(\mathbf{t} - \mathbf{d}\right){\mathbf{i}^\mathrm{excess}}^\mathsf{T}$, 
			$\mathbf{p} =  \mathbf{h}(\mathbf{S},\mathbf{c})$\;
		%
		\textbf{[3] Calculate allowable deterioration}\;
		\qquad loop over input specifications \For{$k = 1, 2, ..., n$} {
			\qquad perform a line search along the $k$th direction to identify the root of: 
				$\min{((\mathbf{f}(\mathbf{x},\mathbf{c}) - \mathbf{d}){\mathbf{i}^\mathrm{excess}}^\mathsf{T})} = \mathbf{0}$, $s_k' \gets x^*$, 
				$s_k^\mathrm{limit} \gets s_k'$\;
		}
		%
		\textbf{[4] Calculate margin value metrics}\;
		\qquad initialization tensors and vectors, 
			$\mathbf{I} \in \mathbb{R}^{m\times e} \gets \mathbf{0}$ and $\mathbf{A} \in \mathbb{R}^{n\times e} \gets \mathbf{0}$, 
			$\mathbf{v} \in \mathbb{R}^{n} \gets \mathbf{0}$, $\bar{\mathbf{a}} \in \mathbb{R}^{e} \gets \mathbf{0}$, and $\bar{\mathbf{i}} \in \mathbb{R}^{e} \gets \mathbf{0}$\;
		\qquad loop over margin nodes \For{$i = 1, 2, ..., e$} {  
			\qquad \textbf{[a] Calculate impact on performance}\;
			\qquad $\mathbf{d}^\mathrm{threshold} \gets \mathbf{d}$, $d_i^\mathrm{threshold} \gets t_i$, 
				$\mathbf{c}_i^\mathrm{threshold} = \mathbf{g}^{-1}(\mathbf{d}^\mathrm{threshold})$, 
				$\mathbf{p}_i^\mathrm{threshold} = \mathbf{h}(\mathbf{S},\mathbf{c}_i^\mathrm{threshold})$\;
			\qquad loop over performance parameters \For{$j = 1, 2, ..., m$} {
				\qquad $i_{i,j} \gets (p_j - p_{j,i}^\mathrm{threshold})i_j^p/{p_{j,i}^\mathrm{threshold}}$\;
			}
			\qquad \textbf{[b] Calculate change absorption capability}\;
			\qquad loop over input specifications \For{$k = 1, 2, ..., n$} {
				\qquad calculate deterioration $v_k=(s_k^\mathrm{limit} - s_k)/{s_k}$\;
				\qquad $a_{i,k} \gets (f_i(\mathbf{s}_k^\mathrm{limit},\mathbf{c}) - f_i(\mathbf{S},\mathbf{c}))/(f_i(\mathbf{S},\mathbf{c})v_k)$\;
			}
			\qquad \textbf{[c] Aggregate metrics}: 
				$\bar{a}_i \gets \frac{1}{n}\sum_{k=1}^{n} a_{i,k}$, 
				$\bar{i}_i \gets \frac{1}{m}\sum_{k=1}^{m} i_{i,j}$
		}
		%
		\textbf{[5] sum of monte-carlo samples}: 
			$\bar{\bar{\mathbf{a}}} \gets \bar{\bar{\mathbf{a}}} + \bar{\mathbf{a}}$, 
			$\bar{\bar{\mathbf{i}}} \gets \bar{\bar{\mathbf{i}}} + \bar{\mathbf{i}}$\;
		%
		\textbf{[6] Termination}, 
		\If{$i_\mathrm{sample} < n_\mathrm{samples}$} {
			set $i_\mathrm{sample} \gets i_\mathrm{sample}+1$, go to \textbf{[1]}\;
			otherwise $\bar{\bar{\mathbf{a}}} \gets \bar{\bar{\mathbf{a}}}/n_\mathrm{samples}$ and $\bar{\bar{\mathbf{i}}} \gets \bar{\bar{\mathbf{i}}}/n_\mathrm{samples}$, \textbf{stop}
		}
	\end{algorithm}
\end{frame}
%------------------------------------------------
\begin{frame}[t,noframenumbering]
	\frametitle{UML diagram of \texttt{mvmlib}\ifshowcitations\footpartcite{Alhandawi2022}\fi}
    %
	\begin{figure}
        \centering
		\only<1>{\includegraphics[width=1.0\textwidth]{design_space_exploration/classes_mvmlib.pdf}}%
		\only<2>{\includegraphics[width=0.6\textwidth]{design_space_exploration/packages_mvmlib.pdf}}%
    \end{figure}%
\end{frame}
\addtocounter{footnote}{-1}
%------------------------------------------------