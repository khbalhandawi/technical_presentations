%%%%%%%%%%%%%%%%%%%%%%%%%%%%%%%%%%%%%%%%%%%%%%%%%%%%%%%
%%      Supplementary Content: Design margins        %%
%%%%%%%%%%%%%%%%%%%%%%%%%%%%%%%%%%%%%%%%%%%%%%%%%%%%%%%
%------------------------------------------------
\begin{frame}[c,noframenumbering]
	\centering
	% \setlength\fboxsep{0pt}
	\begin{titleblock}{}
		~\\%
		{\centering\LARGE Supplementary content: Design margin allocation\\}%
		~\\%
	\end{titleblock}
\end{frame}
%------------------------------------------------
\subsection{Thermal loadcase explained}
%------------------------------------------------
\begin{frame}[t,label=ap2_1]
	\frametitle{Thermal loadcase explained}
	\vspace{-1em}
	\only<1>{Let us revisit the previous example of the TRS}%
	\only<2>{Consider four thermal loads $\mathbf{T} = \left[T_1,T_2,T_3,T_4\right]$ that can change}%
	\only<3-8>{A stiffener pattern can be deposited on the outercasing}%
	~\\
	\begin{columns}[t] % The "c" option specifies centered vertical alignment while the "t" option is used for top vertical alignment
			
		\begin{column}{.45\textwidth} % Left column and width
			\only<1-3>{\uncover<3>{"wavy" concept: $c=0${\color{white},$\mathbf{D}=\left[1\right]$}}}%
			\only<4>{"hatched" concept: $c=1${\color{white},$\mathbf{D}=\left[1\right]$}}%
			\only<5>{"tubular" concept:$c=2${\color{white},$\mathbf{D}=\left[1\right]$}}%
			\only<6>{$c=1$, Desposit steps: $\mathbf{D}=\left[1\right]$}%
			\only<7>{$c=1$, Desposit steps: $\mathbf{D}=\left[1,2\right]$}%
			\only<8->{$c=1$, Desposit steps: $\mathbf{D}=\left[1,2,4\right]$}%
			\begin{figure}
				\only<1-2>{\includegraphics[height=5.0cm]{design_margins/DM_geometry/TRS_isometric_1.pdf}}%
				\only<3>{\includegraphics[height=5.0cm]{design_margins/DM_geometry/TRS_isometric_2.pdf}}%
				\only<4>{\includegraphics[height=5.0cm]{design_margins/DM_geometry/TRS_isometric_3.pdf}}%
				\only<5>{\includegraphics[height=5.0cm]{design_margins/DM_geometry/TRS_isometric_4.pdf}}%
				\only<6>{\includegraphics[height=5.0cm]{design_margins/DM_geometry/TRS_isometric_5.pdf}}%
				\only<7>{\includegraphics[height=5.0cm]{design_margins/DM_geometry/TRS_isometric_6.pdf}}%
				\only<8->{\includegraphics[height=5.0cm]{design_margins/DM_geometry/TRS_isometric_7.pdf}}%
				\vspace{-0.2em}
				\caption{Turbine Rear Structure (TRS)}
			\end{figure}
		\end{column}
	
		\begin{column}{.4\textwidth} % Left column and width
			\begin{figure}
				\only<2->{
					\includegraphics[height=5.0cm]{design_margins/DM_geometry/thermal_loadcase.pdf}
					\vspace{-0.2em}
					\caption{Thermal loads}
				}%
			\end{figure}
		\end{column}
		
	\end{columns}
	
\end{frame}
%------------------------------------------------
\subsection{Dynamically allocating margins}
%------------------------------------------------
\begin{frame}[t,label=ap2_4]
	\frametitle{Changing requirements}
	\vspace{-1em}
	\only<-10>{Let us consider 6 extremely different requirement scenarios\\}%
	\only<11>{The {\color{blue}blue} decisions exceed the required thresholds\\}%
	\only<12>{The {\color{red}red} decisions violate the thresholds but have less excess relative to the {\color{blue}blue} decisions\\}%
	\only<13>{The {\color{darkgreen}green} decisions have more reliability but a lot of excess\\}%
	\uncover<3-10>{Let us also consider 2 different design decisions}%
	\begin{columns}[t] % The "c" option specifies centered vertical alignment while the "t" option is used for top vertical alignment
		
		\begin{column}{.45\textwidth} % Left column and width
			\only<1-10>{
				\uncover<4->{
					$c=1$,$\mathbf{D}=\left[\uncover<4->{{\color<4>{red} 2}},%
					\uncover<5->{{\color<5>{red} 1}},%
					\uncover<6->{\only<-6>{{\color<6>{red} -1}}\only<7->{{\color<7>{red} 0}}},%
					\uncover<8->{{\color<8>{red} 4}},\uncover<9->{{\color<9>{red} -1}},%
					\uncover<10->{{\color<10>{red}3}}\right]$
				}
			}%
			\only<11->{
				\only<11>{$c=1$,$\mathbf{D}=\left[2, 1, 0, 4, -1, 3\right]$}%
				\only<12>{$c=1$,$\mathbf{D}=\left[2, 1, -1, -1, 0, -1\right]$}%
				\only<13>{$c=1$,$\mathbf{D}=\left[4, 1, 0, 2, -1, 3\right]$}%
			}%
			\uncover<2-10>{$\mathbf{R}=\left[{\color<4>{red} Req_{1}},%
										  {\color<5>{red} Req_{2}},%
										  {\color<6-7>{red} Req_{3}},%
										  {\color<8>{red} Req_{4}},%
										  {\color<9>{red} Req_{5}},%
										  {\color<10>{red} Req_{6}}\right]$}%
			% \vspace{-1em}
				\begin{figure}
						\only<1>{\includegraphics[height=5.0cm]{design_margins/Stagespace/2D_stagespace_res_th_only_no_line.pdf}}%	
						\only<2>{\includegraphics[height=5.0cm]{design_margins/Stagespace/2D_stagespace_res_th_only.pdf}}%	
						\only<3>{\includegraphics[height=5.0cm]{design_margins/Stagespace/D1_stagespace_res_1.pdf}}%
						\only<4>{\includegraphics[height=5.0cm]{design_margins/Stagespace/D1_stagespace_res_2.pdf}}%
						\only<5>{\includegraphics[height=5.0cm]{design_margins/Stagespace/D1_stagespace_res_3.pdf}}%
						\only<6>{\includegraphics[height=5.0cm]{design_margins/Stagespace/D1_stagespace_res_4.pdf}}%
						\only<7>{\includegraphics[height=5.0cm]{design_margins/Stagespace/D3_stagespace_res_4.pdf}}%
						\only<8>{\includegraphics[height=5.0cm]{design_margins/Stagespace/D3_stagespace_res_5.pdf}}%
						\only<9>{\includegraphics[height=5.0cm]{design_margins/Stagespace/D3_stagespace_res_6.pdf}}%
						\only<10>{\includegraphics[height=5.0cm]{design_margins/Stagespace/D3_stagespace_res_7.pdf}}%
						\only<11-12>{\includegraphics[height=5.0cm]{design_margins/Stagespace/2D_stagespace_res.pdf}}%
						\only<13>{\includegraphics[height=5.0cm]{design_margins/Stagespace/3D_stagespace_res.pdf}}%
					\label{fig:stagespace} 
					\vspace{-1.5em}
					\caption{Reliability at each epoch}
				\end{figure}
		\end{column}
	
		\begin{column}{.45\textwidth} % Left column and width
			\centering
			\vspace{-2.4em}
			\only<-10>{
					\color{white}{
					$\boldsymbol{\mu}=\left[ 0.5 ,~ 0.5 ,~ 0.5 ,~ 0.5 \right]$\\%
					$\boldsymbol{\sigma}=\left[ 0.1875375 ,~ 0.125 ,~ 0.125 ,~ 0.1875 \right]$%
				}
			}
			\only<11->{
				~\\%
				Total excess = \only<11>{4.58}\only<12>{{\color{darkgreen} 3.91 $\downarrow$}}%
							   \only<13>{{\color{red} 5.16 $\uparrow$}}\\%
				~\\%
				Reliability: \only<11>{{\color{darkgreen} Above threshold}}%
							 \only<12>{{\color{red} Below threshold}}%
							 \only<13>{{\color{darkgreen} Above threshold}}%
			}
			\begin{figure}
				\only<4>{\includegraphics[height=5.6cm]{design_margins/DM_parameter_space/146_reqs_2D/146_req_36_thermal_out_RS_2D_pi_6.pdf}}%
				\only<5>{\includegraphics[height=5.6cm]{design_margins/DM_parameter_space/163_reqs_2D/163_req_50_thermal_out_RS_2D_pi_6.pdf}}%
				\only<6>{\includegraphics[height=5.6cm]{design_margins/DM_parameter_space/163_reqs_2D/163_req_1_thermal_out_RS_2D_pi_6.pdf}}%
				\only<7>{\includegraphics[height=5.6cm]{design_margins/DM_parameter_space/164_reqs_2D/164_req_1_thermal_out_RS_2D_pi_6.pdf}}%
				\only<8>{\includegraphics[height=5.6cm]{design_margins/DM_parameter_space/167_reqs_2D/167_req_46_thermal_out_RS_2D_pi_6.pdf}}%
				\only<9>{\includegraphics[height=5.6cm]{design_margins/DM_parameter_space/167_reqs_2D/167_req_13_thermal_out_RS_2D_pi_6.pdf}}%
				\only<10>{\includegraphics[height=5.6cm]{design_margins/DM_parameter_space/168_reqs_2D/168_req_31_thermal_out_RS_2D_pi_6.pdf}}%
				\only<11>{\includegraphics[height=5.0cm]{design_margins/Stagespace/D3_stagespace_obj.png}}%
				\only<12>{\includegraphics[height=5.0cm]{design_margins/Stagespace/D1_stagespace_obj.png}}%
				\only<13>{\includegraphics[height=5.0cm]{design_margins/Stagespace/D2_stagespace_obj.png}}%
				\label{fig:stagespace_obj}
				\only<11->{
					\vspace{-1.5em}
					\caption{Excess at each epoch}
				}
			\end{figure}
		\end{column}
		
	\end{columns}
	
\end{frame}
%------------------------------------------------
\subsection{Optimizing when and how much change is needed}
%------------------------------------------------
\begin{frame}[t]
	\frametitle{Optimizing when and how much change is needed}
	\vspace{-0.5em}
	Formulate optimization problem to minimize total excess while being reliable
	
	\begin{columns}[t] % The "c" option specifies centered vertical alignment while the "t" option is used for top vertical alignment
		
		\begin{column}{.42\textwidth} % Left column and width
			\vspace{-1.5em}
			% Optimization problem
			\begin{alertblock}{Objective and constraints}
				\vspace{-1.5em}
				\begin{equation*}
					\begin{aligned}
						\uncover<2->{
							& {\text{minimize}}
							& & \text{excess}(c,\mathbf{D};\text{requirement arc})\\
						}
						\uncover<3->{
							& \text{subject to}
							& & reliability(c,\mathbf{D};\text{requirement arc})\\
						}	
						\uncover<4->{
							& \text{over}
							& & \text{The set of feasible design decisions}\\
							& & & \text{(404 possible combinations)}
						}
					\end{aligned}
				\end{equation*}
			\end{alertblock}
			\vspace{-0.5em}
			\uncover<4->{
				% Variables
				\begin{exampleblock}{Design variables}
					\vspace{-0.5em}
					\begin{equation*}
						\begin{aligned}
							& & & c \in \left\{"wavy","hatched","tubular"\right\}\\
							& & & \mathbf{D} = \left[\text{decision}_1,\text{decision}_2,\cdots\right]\\
							& & & \text{decision} \in \left\{-1,0,1,2,3,4\right\}\\
						\end{aligned}
					\end{equation*}
				\end{exampleblock}
			}
			\vspace{-0.5em}
			\uncover<6->{
				% Parameters
				\begin{block}{Requirement arcs}
					\vspace{-0.5em}
					\begin{equation*}
						\begin{aligned}
							& & & \text{We can randomly generate}\\
							& & & \text{10000 different requirement arcs}
						\end{aligned}
					\end{equation*}
				\end{block}
			}
		\end{column}

		\begin{column}{.5\textwidth} % Right column and width
			\vspace{-1em}
			\tikzstyle{background grid}=[draw, black!50,step=.5cm]
			
			\begin{tikzpicture}%[show background grid]
			% Put the graphic inside a node. This makes it easy to place the
			% graphic and to draw on top of it. 
			% The above right option is used to place the lower left corner
			% of the image at the (0,0) coordinate. 
				\uncover<-5>{
					\node [inner sep=0pt,above right, opacity=1.0]  at (0, 1) (objective)
						{
							\only<2->{\includegraphics[height=4.0cm]{design_margins/Stagespace/opt_stagespace_obj.png}}%
						};
					\node [inner sep=0pt,above right, opacity=0.7]  at (2.5, -1.75) (constraint)
						{
							\only<3>{\includegraphics[height=4.0cm]{design_margins/Stagespace/2D_stagespace_res_th_only_no_line.pdf}}%
							\only<4->{\includegraphics[height=4.0cm]{design_margins/Stagespace/opt_stagespace_res.pdf}}%
						};
				}
				\uncover<4->{
					\node[draw, align=center, fill=white, inner sep=2pt,above right, opacity=1]  at (0.35, 2) {
							Solving the problem yields optimal decisions\\% needs another set of braces
							{\uncover<5->{$\downarrow$}}\\%
							{\uncover<5->{optimal designs}}\\%
							{\uncover<5->{$\left\{c^{*},\mathbf{D}^{*}\right\}$}}%
						};
					\uncover<7->{
						\draw (0,1.5) rectangle (8,5) node at (4.,4.75) {Repeat for different requirement arcs};
					}
				}
				% show origin
				% \fill (0,0) circle (2pt);
				% define destination coordinates
			\end{tikzpicture}

		\end{column}

	\end{columns}
\end{frame}
%------------------------------------------------
\subsection{Set-based solution of parametric optimization problem}
%------------------------------------------------
\begin{frame}[t]
	\frametitle{Set-based solution of parametric optimization problem}
	\vspace{-0.5em}
	\begin{nscenter}
		\only<1>{We solve the problem for $\mathit{n = {\color{red} 100}}$ requirement arcs}%
		\only<2>{We solve the problem for $\mathit{n = {\color{red} 1000}}$ requirement arcs}%
		\only<3>{We solve the problem for $\mathit{n = {\color{red} 100000}}$ requirement arcs}%
	\end{nscenter}%

	\begin{columns}[t] % The "c" option specifies centered vertical alignment while the "t" option is used for top vertical alignment
		
		\begin{column}{.42\textwidth} % Left column and width
			\vspace{-1.5em}
			% Optimization problem
			\begin{alertblock}{Objective and constraints}
				\vspace{-1.5em}
				\begin{equation*}
					\begin{aligned}
							& {\text{minimize}}
							& & \text{excess}(c,\mathbf{D};\text{requirement arc})\\
							& \text{subject to}
							& & reliability(c,\mathbf{D};\text{requirement arc})\\
							& \text{over}
							& & \text{The set of feasible design decisions}\\
							& & & \text{(404 possible combinations)}
					\end{aligned}
				\end{equation*}
			\end{alertblock}
			\vspace{-0.5em}
			% Variables
			\begin{exampleblock}{Design variables}
				\vspace{-0.5em}
				\begin{equation*}
					\begin{aligned}
						& & & c \in \left\{"wavy","hatched","tubular"\right\}\\
						& & & \mathbf{D} = \left[\text{decision}_1,\text{decision}_2,\cdots\right]\\
						& & & \text{decision} \in \left\{-1,0,1,2,3,4\right\}\\
					\end{aligned}
				\end{equation*}
			\end{exampleblock}
			\vspace{-0.5em}
			% Parameters
			\begin{block}{Requirement arcs}
				\vspace{-0.5em}
				\begin{equation*}
					\begin{aligned}
						& & & \text{We can randomly generate}\\
						& & & \text{10000 different requirement arcs}
					\end{aligned}
				\end{equation*}
			\end{block}
		\end{column}

		\begin{column}[t]{.5\textwidth} % Right column and width

			\centering
			\tikzstyle{background grid}=[draw, black!50,step=.5cm]
			\begin{tikzpicture}%[show background grid]
			% Put the graphic inside a node. This makes it easy to place the
			% graphic and to draw on top of it. 
			% The above right option is used to place the lower left corner
			% of the image at the (0,0) coordinate. 
				\node [inner sep=0pt,above right, opacity=1] (histogram)
					{
						\only<1>{\includegraphics[height=5.0cm]{design_margins/Histograms/histogram_DOE_E_1.pdf}}%
						\only<2>{\includegraphics[height=5.0cm]{design_margins/Histograms/histogram_DOE_E_10.pdf}}%
						\only<3>{\includegraphics[height=5.0cm]{design_margins/Histograms/histogram_DOE_E_28.pdf}}%
					};
			\end{tikzpicture}

		\end{column}

	\end{columns}
\end{frame}
%------------------------------------------------
\subsection{Tradespace exploration}
%------------------------------------------------
\begin{frame}[t]
	\frametitle{Tradespace exploration}
	\vspace{-0.5em}
	On a tradespace, we plot the feasible designs, %
	\uncover<3->{the {\color{purple}Pareto front}}\uncover<4->{, and {\color{red}the optimal designs}}%

	\begin{columns}[t] % The "c" option specifies centered vertical alignment while the "t" option is used for top vertical alignment
		
		\begin{column}[t]{.42\textwidth} % Left column and width

			% Use a background grid to make it easier to find coordinates
			% When the coordinates have been found, remove the 
			% 'show background grid' option. 
			\tikzstyle{background grid}=[draw, black!50,step=.5cm]
			\begin{tikzpicture}%[show background grid]
				% Put the graphic inside a node. This makes it easy to place the
				% graphic and to draw on top of it. 
				% The above right option is used to place the lower left corner
				% of the image at the (0,0) coordinate.
				\uncover<2->{
					\node [inner sep=0pt,above right, opacity=1.0] (histogram)
						{
							\only<1-2>{\includegraphics[height=5.0cm]{design_margins/tradespace/tradespace_pareto_D.pdf}}%
							\only<3>{\includegraphics[height=5.0cm]{design_margins/tradespace/tradespace_pareto_DP.pdf}}%
							\only<4>{\includegraphics[height=5.0cm]{design_margins/tradespace/tradespace_pareto_DPS.pdf}}%
							\only<5>{\includegraphics[height=5.0cm]{design_margins/tradespace/tradespace_pareto_PS.pdf}}%
						};
					\node[inner sep=0pt,align=flush center,above=\belowcaptionskip of histogram,text width=\linewidth]
						{\vspace{-1em}{\large Tradespace}};
					% show origin
					%\fill (0,0) circle (2pt);
				}%

			\end{tikzpicture}


		\end{column}

		\begin{column}[t]{.5\textwidth} % Right column and width
			% Use a background grid to make it easier to find coordinates
			% When the coordinates have been found, remove the 
			% 'show background grid' option. 
			\tikzstyle{background grid}=[draw, black!50,step=.5cm]
			\begin{tikzpicture}%[show background grid]
				% Put the graphic inside a node. This makes it easy to place the
				% graphic and to draw on top of it. 
				% The above right option is used to place the lower left corner
				% of the image at the (0,0) coordinate. 
				\node [inner sep=0pt,above right, opacity=1.0] (histogram)
					{
						\includegraphics[height=5.0cm]{design_margins/Histograms/histogram_DOE_E_28.pdf}%
					};
				\node[inner sep=0pt,align=flush center,above=\belowcaptionskip of histogram,text width=\linewidth]
					{\vspace{-1em}{\large 
						Histogram (optimal designs)
					}};
				% show origin
				%\fill (0,0) circle (2pt);
				
			\end{tikzpicture}

		\end{column}

	\end{columns}

\end{frame}
%------------------------------------------------
\begin{frame}[t]
	\frametitle{Tradespace exploration: Robustness}
	\vspace{-0.5em}
	We can check the percentage of the 10000 requirements arcs satisfied by each feasible design decision to identify the most {\color{blue} Robust solutions}
	\vspace{-1.0em}
	\begin{columns}[t] % The "c" option specifies centered vertical alignment while the "t" option is used for top vertical alignment
		
		\begin{column}[t]{.42\textwidth} % Left column and width

			% Use a background grid to make it easier to find coordinates
			% When the coordinates have been found, remove the 
			% 'show background grid' option. 
			\tikzstyle{background grid}=[draw, black!50,step=.5cm]
			\begin{tikzpicture}%[show background grid]
				% Put the graphic inside a node. This makes it easy to place the
				% graphic and to draw on top of it. 
				% The above right option is used to place the lower left corner
				% of the image at the (0,0) coordinate.
				\node [inner sep=0pt,above right, opacity=1.0] (histogram)
					{
						\includegraphics[height=5.0cm]{design_margins/tradespace/tradespace_pareto_PSR.pdf}%
					};
				\node[inner sep=0pt,align=flush center,above=\belowcaptionskip of histogram,text width=\linewidth]
					{\vspace{-1em}{\large Tradespace}};
				% show origin
				%\fill (0,0) circle (2pt);
			\end{tikzpicture}


		\end{column}

		\begin{column}[t]{.5\textwidth} % Right column and width
			% Use a background grid to make it easier to find coordinates
			% When the coordinates have been found, remove the 
			% 'show background grid' option. 
			\tikzstyle{background grid}=[draw, black!50,step=.5cm]
			\begin{tikzpicture}%[show background grid]
				% Put the graphic inside a node. This makes it easy to place the
				% graphic and to draw on top of it. 
				% The above right option is used to place the lower left corner
				% of the image at the (0,0) coordinate. 
				\node [inner sep=0pt,above right, opacity=1.0] (histogram)
					{
						\includegraphics[height=5.0cm]{design_margins/Histograms/histogram_DOE_R.pdf}%
					};
				\node[inner sep=0pt,align=flush center,above=\belowcaptionskip of histogram,text width=\linewidth]
					{\vspace{-1em}{\large 
						Histogram (robust designs)
					}};
				% show origin
				%\fill (0,0) circle (2pt);
				
			\end{tikzpicture}

		\end{column}

	\end{columns}

\end{frame}
%------------------------------------------------
\begin{frame}[t]
	\frametitle{Tradespace exploration: Flexibility}
	\vspace{-0.5em}
	The filtered outdegree is the number of design alternatives offered by each decision

	\begin{columns}[t] % The "c" option specifies centered vertical alignment while the "t" option is used for top vertical alignment
		
		\begin{column}[t]{.42\textwidth} % Left column and width

			% Use a background grid to make it easier to find coordinates
			% When the coordinates have been found, remove the 
			% 'show background grid' option. 
			\tikzstyle{background grid}=[draw, black!50,step=.5cm]
			\begin{tikzpicture}%[show background grid]
				% Put the graphic inside a node. This makes it easy to place the
				% graphic and to draw on top of it. 
				% The above right option is used to place the lower left corner
				% of the image at the (0,0) coordinate.
				\node [inner sep=0pt,above right, opacity=1.0] (histogram)
					{
						\includegraphics[height=5.0cm]{design_margins/tradespace/tradespace_pareto_PSRF.pdf}%
					};
				\node[inner sep=0pt,align=flush center,above=\belowcaptionskip of histogram,text width=\linewidth]
					{\vspace{-1em}{\large Tradespace}};
				% show origin
				%\fill (0,0) circle (2pt);
			\end{tikzpicture}


		\end{column}

		\begin{column}[t]{.5\textwidth} % Right column and width
			% Use a background grid to make it easier to find coordinates
			% When the coordinates have been found, remove the 
			% 'show background grid' option. 
				\tikzstyle{background grid}=[draw, black!50,step=.5cm]
				\begin{tikzpicture}%[show background grid]
					% Put the graphic inside a node. This makes it easy to place the
					% graphic and to draw on top of it. 
					% The above right option is used to place the lower left corner
					% of the image at the (0,0) coordinate. 
					\node [inner sep=0pt,above right, opacity=1.0] (histogram)
						{
							\includegraphics[height=5.0cm]{design_margins/Histograms/histogram_DOE_F.pdf}%
						};
					\node[inner sep=0pt,align=flush center,above=\belowcaptionskip of histogram,text width=\linewidth]
						{\vspace{-1em}{\large 
							Histogram (flexible designs)
						}};
					% show origin
					%\fill (0,0) circle (2pt);
					
				\end{tikzpicture}

		\end{column}

	\end{columns}

\end{frame}

%------------------------------------------------
\begin{frame}[t]
	\frametitle{Tradespace exploration: Comparison}
	\vspace{-0.5em}
	We use \textbf{convex hulls} to draw some conclusions about the obtained solution sets

	\begin{columns}[t] % The "c" option specifies centered vertical alignment while the "t" option is used for top vertical alignment
		
		\begin{column}[t]{.42\textwidth} % Left column and width

			% Use a background grid to make it easier to find coordinates
			% When the coordinates have been found, remove the 
			% 'show background grid' option. 
			\tikzstyle{background grid}=[draw, black!50,step=.5cm]
			\begin{tikzpicture}%[show background grid]
				% Put the graphic inside a node. This makes it easy to place the
				% graphic and to draw on top of it. 
				% The above right option is used to place the lower left corner
				% of the image at the (0,0) coordinate.
				\node [inner sep=0pt,above right, opacity=1.0] (histogram)
					{
						\only<1>{\includegraphics[height=5.0cm]{design_margins/tradespace/tradespace_pareto_PSRF_CH.pdf}}%
						\only<2->{\includegraphics[height=5.0cm]{design_margins/tradespace/tradespace_pareto_PSRF_CH_NP.pdf}}%
					};
				\node[inner sep=0pt,align=flush center,above=\belowcaptionskip of histogram,text width=\linewidth]
					{\vspace{-1em}{\large Tradespace}};
				% show origin
				%\fill (0,0) circle (2pt);

				\path (2.49,2.875) coordinate (flexible)
						(3.95,4.2) coordinate (SBD)
						(5.4,4.75) coordinate (robust);

			\end{tikzpicture}


		\end{column}

		\begin{column}[t]{.5\textwidth} % Right column and width
			% Use a background grid to make it easier to find coordinates
			% When the coordinates have been found, remove the 
			% 'show background grid' option. 
			\begin{itemize}
				\uncover<3->{\item \tikz[na] \coordinate (robust_bullet); The \textbf{\color{blue} robust} solutions favor {\color{red} heavier $\uparrow$} designs with {\color{darkgreen} high capabilities $\uparrow$}}\\
				~\\
				\uncover<5->{\item \tikz[na] \coordinate (SBD_bullet); The \textbf{\color{red} optimized} set-based solutions balance weight with capability}\\
				~\\
				\uncover<4->{\item \tikz[na] \coordinate (flexible_bullet); The \textbf{\color{darkgreen} flexible} solutions favor {\color{darkgreen} light $\downarrow$} designs with {\color{red} low capabilities $\downarrow$}}
			\end{itemize}

		\end{column}

	\end{columns}

	% define overlays
	% Note the use of the overlay option. This is required when 
	% you want to access nodes in different pictures.
	\begin{tikzpicture}[overlay]
		\only<3->{\path[->,magenta,thick] (robust) edge [out=0, in=180] (robust_bullet);}
		\only<4->{\path[->,magenta,thick] (flexible) edge [out=0, in=180] (flexible_bullet);}
		\only<5->{\path[->,magenta,thick] (SBD) edge [out=0, in=180] (SBD_bullet);}
	\end{tikzpicture}

\end{frame}